\chapter*{Introduction}
\addcontentsline{toc}{chapter}{Introduction}
\markboth{Introduction}{Introduction}


Il existe déjà de multiples ouvrages traitant de S-Plus ou \textsf{R}.
Dans la majorité des cas, toutefois, ces deux logiciels sont présentés
dans le cadre d'applications statistiques spécifiques. Le présent
ouvrage se concentre plutôt sur l'apprentissage du langage de
programmation sous-jacent aux diverses fonctions statistiques, le S.

Cette seconde édition est principalement une réorganisation du contenu
de la première édition. Les chapitres sur la régression linéaire et
les séries chronologiques ont été éliminés et les annexes C et D
ont été déplacées dans le corps du document. Nous avons également
corrigé plusieurs coquilles suite à la révision effectuée par
Mme~Mireille Côté.

Le texte est la somme de notes et d'exercices de cours donnés par
l'auteur à l'École d'actuariat de l'Université Laval. Les six premiers
chapitres, qui constituent le c{\oe}ur du document, proviennent d'une
partie d'un cours où l'accent est mis sur l'apprentissage d'un
(deuxième) langage de programmation par des étudiants de premier cycle
en sciences actuarielles. Les applications numériques et statistiques
de S-Plus et \textsf{R} présentées aux chapitres \ref{optimisation},
\ref{rng} et \ref{simulation} sont étudiées plus tard dans le cursus
universitaire.

Les cours d'introduction au langage S sont donnés à raison de une
heure par semaine de cours magistral suivie de deux heures en
laboratoire d'informatique. C'est ce qui explique la structure des six
premiers chapitres: les éléments de théorie, contenant peu voire aucun
exemple, sont présentés en rafale en classe. Puis, lors des séances de
laboratoire, les étudiantes et étudiants sont appelés à lire et
exécuter les exemples se trouvant à la fin des chapitres. Chaque
section d'exemples couvre l'essentiel des concepts présentés dans le
chapitre et les complémente souvent. L'étude de ces sections fait donc
partie intégrante de l'apprentissage du langage S.

Le texte des sections d'exemples est disponible en format électronique
dans le site Internet
\begin{quote}
  \url{http://vgoulet.act.ulaval.ca/intro_S}
\end{quote}

Certains exemples et exercices trahissent le premier public de ce
document: on y fait à l'occasion référence à des concepts de base de
la théorie des probabilités et des mathématiques financières. Les
contextes actuariels demeurent néanmoins peu nombreux et ne devraient
généralement pas dérouter le lecteur pour qui ces notions sont moins
familières.

Les chapitres \ref{optimisation} (fonctions d'optimisation), \ref{rng}
(générateurs de nombres aléatoires) et \ref{simulation} (planification
d'une simulation en S) sont structurés de manière plus classique,
notamment parce que le texte y est en prose.

Le texte prend parti en faveur de l'utilisation de GNU Emacs et du
mode ESS pour l'édition de code S.  Les annexes contiennent de
l'information sur l'utilisation de S-Plus et \textsf{R} avec cet
éditeur.

Dans la mesure du possible, cet ouvrage tâche de présenter les
environnements S-Plus et \textsf{R} en parallèle, en soulignant leurs
différences s'il y a lieu. Les informations propres à S-Plus ou à
\textsf{R} sont d'ailleurs signalées en marge par les marques «S$+$»
et «\textsf{R}», respectivement. Étant donné la nette préférence de
l'auteur pour \textsf{R}, les divers extraits de code ont généralement
été exécutés avec ce moteur S.

À moins d'erreurs et d'omissions (que les lecteurs sont invités à nous
faire connaître), les informations données à propos de S-Plus sont
exactes pour les versions 6.1 (Linux et Windows), 6.2 Student Edition
(Windows) et 7.0 (Linux et Windows). Pour \textsf{R}, la version 2.4.1
(Linux et Windows), soit la plus récente lors de la rédaction, a été
utilisée comme référence.

On notera enfin que cet ouvrage n'a aucune prétention d'exhaustivité.
C'est ce qui explique les nombreux renvois au livre de \cite{MASS},
plus complet.

L'auteur tient à remercier M.~Mathieu Boudreault pour sa collaboration
dans la rédaction des exercices.

\begin{flushright}
  \itshape
  Vincent Goulet \url{<vincent.goulet@act.ulaval.ca>} \\
  Québec, janvier 2007
\end{flushright}


%%% Local Variables:
%%% mode: latex
%%% TeX-master: "introduction_programmation_S"
%%% coding: utf-8
%%% End:
