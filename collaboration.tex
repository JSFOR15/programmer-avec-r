%%% Copyright (C) 2017 Vincent Goulet
%%%
%%% Ce fichier fait partie du projet «Programmer avec R»
%%% http://github.com/vigou3/programmer-avec-r
%%%
%%% Cette création est mise à disposition selon le contrat
%%% Attribution-Partage dans les mêmes conditions 4.0
%%% International de Creative Commons.
%%% http://creativecommons.org/licenses/by-sa/4.0/

\chapter{Travail collaboratif}
\label{chap:collaboration}

\begin{objectifs}
\item Employer les normes de programmation reconnues en matière de
  segmentation du code, de style et de documentation.
\item Créer une version locale d'un projet informatique hébergé dans
  un dépôt utilisant le système de gestion de version Git.
\item Créer, utiliser et supprimer une branche dans un projet avec
  Git.
\item Publier des ajouts et des modifications à un projet avec Git.
\item Rendre disponibles des ajouts et des modifications à une
  communauté par le biais d'un serveur Git.
\end{objectifs}

De manière générale, le développement et la maintenance de code
informatique repose sur la contribution de plusieurs personnes. En
effet, il est plutôt rare, dans le milieu professionnel, d'être appelé
à concevoir un programme informatique à partir d'une page blanche et
en complète autarcie, c'est-à-dire sans que quiconque n'ait à
interagir avec le code à un stade ou à un autre. Une grande part du
travail de programmation consiste à corriger, à mettre à jour ou à
améliorer du code existant. Dans ce contexte, l'adhésion à un certain
nombre de normes et de bonnes pratiques permet de faciliter le travail
de tous les intervenants et de réduire les risques d'erreurs. Ce
chapitre présente quelques unes de ces bonnes pratiques à adopter en
matière de style de programmation, de présentation du code et de
documentation.

Autre élément fondamental lors de travail collaboratif: le partage et
l'échange des contributions de chacun. Comment rendre le travail de
Marianne disponible à Alexandre? Comment s'assurer qu'une modification
apportée dans un fichier par Alexandre n'écrase celle sur laquelle
Marianne travaille depuis plusieurs heures? Comment identifier à coup
sûr et automatiquement la plus récente version d'un fichier? Ce genre
de questions, auxquelles quiconque a déjà travaillé en équipe sur un
projet a déjà été confronté, les informaticiens y ont répondu depuis
de très nombreuses années avec les systèmes de gestion de versions.
Nous offrons, dans ce chapitre, une courte introduction à ces outils
de développement incontournables avant de diriger les lecteurs vers la
documentation officielle du système de gestion le plus utilisé dans le
monde en ce moment: Git.


\section{Style}
\label{sec:collaboration:style}

Il en va du code informatique comme de la prose: si le style peut
varier d'un auteur à l'autre, l'œuvre doit toujours être à la fois
agréable à lire et facile à comprendre. Bref, les meilleurs
programmeurs préfèrent la \emph{lisibilité} de leur code aux effets de
style qui n'auraient pour seul mérite d'afficher leur maitrise du
langage.

Tout programmeur devrait constamment garder en tête les trois
objectifs suivants en effectuant son travail: simplicité, clarté,
concision. Ces objectifs entrent souvent en conflit les uns avec les
autres! Tout l'art de la bonne programmation consiste donc à trouver
un juste équilibre entre les trois pôles.

\citet{Kernighan:practice:1999,Oualline:C:1997,Kernighan:style:1978}
proposent d'excellents chapitres sur le style en programmation. Nous
se saurions être aussi complet que ces auteurs établis. Néanmoins,
nous vous incitons à porter une attention particulière aux quelques
points de style livrés en vrac, ci-dessous.

\begin{itemize}
\item Utilisez des noms de variables significatifs. Ne soyez pas ce
  collègue qui nomme les variables d'un programme \code{x}, \code{xx}
  et \code{xxx} (cas vécu). Attention, toutefois, de ne pas pousser le
  concept trop loin. Ici comme ailleurs, la clarté peut provenir de la
  concision; la terminologie\footnote{%
    Vous remarquerez que nous préférons utiliser l'anglais pour les
    noms d'objets, question d'uniformité avec les identificateurs du
    langage. Chose certaine, évitez à tout prix les accents dans les
    noms d'objets.}
  \begin{Schunk}
\begin{Verbatim}
xlen <- length(x)
\end{Verbatim}
  \end{Schunk}
  est aussi claire que
  \begin{Schunk}
\begin{Verbatim}
length_of_x <- length(x)
\end{Verbatim}
  \end{Schunk}
  et bien plus simple à utiliser au fil d'un programme.

  Certains noms d'objets sans réelle signification sont tellement
  usuels qu'il est contre-productif de leur préférer des versions plus
  explicites. Pensons, ici, à \code{x} comme premier argument d'une
  fonction R ou à \code{i}, \code{j} et \code{k} comme compteurs dans
  les boucles \code{for}.

  Quant à la mise en forme des longs noms, plusieurs styles
  s'affrontent: \code{variable.name}, \code{variable\_name},
  \code{variableName}, \code{VariableName}, etc. Assurez-vous
  simplement de suivre, le cas échéant, le standard en vigueur dans
  votre équipe de travail et, par-dessus tout, soyez constant. Notre
  préférence, qui concorde avec le code source de R, va aux noms
  d'objets courts et entièrement en minuscules.
  %
\item Dès qu'elles sont disponibles, utilisez les fonctions internes
  de R au lieu de reprogrammer certaines procédures. Non seulement
  bénéficierez-vous de l'optimisation des fonctions internes, mais
  votre code gagnera également en lisibilité. Comparez
  \begin{Schunk}
\begin{Verbatim}
sum(x)/length(x)
\end{Verbatim}
  \end{Schunk}
  à
  \begin{Schunk}
\begin{Verbatim}
mean(x)
\end{Verbatim}
  \end{Schunk}
  %
\item Connaitre sur le bout des doigts la priorité des opérateurs du
  \autoref{tab:bases:operateurs}, c'est bien, mais rendre explicite
  l'ordre des opérations dans une expression à l'aide de parenthèses,
  c'est mieux. À ce propos, \citet{Oualline:C:1997} ramène la
  quinzaine de règles de priorité des opérations (du langage C) à
  seulement deux:
  \begin{enumerate}
  \item La multiplication et la division précèdent l'addition et la
    soustraction.
  \item Placer tout le reste entre parenthèses.
  \end{enumerate}
  %
\item Évitez les expressions logiques complexes, surtout celles
  reposant sur la double négation. Par exemple, pour exécuter une
  expression si un vecteur contient des données manquantes, la
  condition
  \begin{Schunk}
\begin{Verbatim}
if (any(is.na(x)))
\end{Verbatim}
  \end{Schunk}
  est beaucoup plus facile à déchiffrer que la version logiquement
  équivalente
  \begin{Schunk}
\begin{Verbatim}
if (!all(!is.na(x)))
\end{Verbatim}
  \end{Schunk}
  En revanche, s'il s'agit plutôt d'exécuter une expression quand un
  vecteur ne contient pas de données manquantes, alors
  \begin{Schunk}
\begin{Verbatim}
if (all(!is.na(x)))
\end{Verbatim}
  \end{Schunk}
  est plus simple que
  \begin{Schunk}
\begin{Verbatim}
if (!any(is.na(x)))
\end{Verbatim}
  \end{Schunk}

  De plus, le conseil précédent sur la priorité des opérations est
  particulièrement indiqué avec les opérations logiques. Sauriez-vous
  confirmer, sans consulter le \autoref{tab:bases:operateurs}, l'ordre
  des opérations dans l'expression logique suivante?\footnote{%
    C'est \code{(!p) | (q \& r)}.}
  \begin{Schunk}
\begin{Verbatim}
!p | q & r
\end{Verbatim}
  \end{Schunk}
  %
\item Utilisez les fonctions d'application
  (\autoref{chap:application}) plutôt que des boucles explicites. Une
  expression utilisant une fonction d'application est plus concise et
  plus simple à décoder. Comparez
  \begin{Schunk}
\begin{Verbatim}
z <- numeric(n)
for (i in seq_len(n))
    z[i] <- mean(x[[i]])
\end{Verbatim}
  \end{Schunk}
  et
  \begin{Schunk}
\begin{Verbatim}
z <- sapply(x, mean)
\end{Verbatim}
  \end{Schunk}

  Encore ici, évitez de pousser la logique trop loin. Si une boucle
  est plus naturelle et simple à comprendre qu'une fonction
  d'application, optez pour la boucle. Dans la même veine, une
  fonction d'application \icode{sapply} à l'intérieur d'une autre
  fonction \code{sapply}, ce n'est généralement ni plus efficace, ni
  plus simple à déchiffrer qu'une double boucle \icode{for}.
  %
\item Utilisez \icode{return} uniquement pour provoquer la sortie
  anticipée d'une fonction, habituellement à l'intérieur d'une clause
  \code{if}. En d'autres termes, \code{return} n'a pas sa place, en R,
  à la toute fin d'une fonction.
  %
\item Adoptez la
  \link{https://fr.wikipedia.org/wiki/Philosophie_d\%27Unix}{philosophie
    Unix}, notamment le précepte qui appelle à créer des programmes
  qui effectuent une seule chose et qui le font bien. Lorsqu'une
  fonction devient «longue» --- cela dépend du contexte, mais
  généralement dès une vingtaine de lignes en R --- il convient de la
  scinder en plusieurs blocs logiques.
\end{itemize}




\section{Commentaires}
\label{sec:collaboration:commentaires}


\section{Présentation du code}
\label{sec:collaboration:presentation}

Si tous conviennent que l'adoption d'un style propre et uniforme
favorise le développement et la lecture de code, il existe plusieurs
chapelles dans le monde des programmeurs quant à la «bonne façon» de
présenter et, surtout, d'indenter le code informatique.

Par exemple, Emacs\index{Emacs} reconnaît et supporte les styles de
codage suivants, entre autres:
\begin{center}
  \begin{minipage}[t]{9cm}
    C++/Stroustrup
    \hfill
    \begin{minipage}[t]{5cm}
\begin{verbatim}
for (i in 1:10)
{
    expression
}
\end{verbatim}
    \end{minipage}
  \end{minipage}
  \vspace{\baselineskip}

  \begin{minipage}[t]{9cm}
    K\&R (1TBS)
    \hfill
    \begin{minipage}[t]{5cm}
\begin{verbatim}
for (i in 1:10){
     expression
}
\end{verbatim}
    \end{minipage}
  \end{minipage}
  \vspace{\baselineskip}

  \begin{minipage}[t]{9cm}
    Whitesmith
    \hfill
    \begin{minipage}[t]{5cm}
\begin{verbatim}
for (i in 1:10)
     {
     expression
     }
\end{verbatim}
    \end{minipage}
  \end{minipage}
  \vspace{\baselineskip}

  \begin{minipage}[t]{9cm}
    GNU
    \hfill
    \begin{minipage}[t]{5cm}
\begin{verbatim}
for (i in 1:10)
  {
    expression
  }
\end{verbatim}
    \end{minipage}
  \end{minipage}
\end{center}

\begin{itemize}
\item Pour des raisons générales de lisibilité et de popularité, le
  style C++, avec les accolades sur leurs propres lignes et une
  indentation de quatre (4) espaces est considéré comme standard pour
  la programmation en R.
% \item La version de GNU~Emacs distribuée par l'auteur est déjà
%   configurée pour utiliser ce style de codage.
\item Consulter la documentation de votre éditeur de texte pour savoir
  s'il est possible de configurer le niveau d'indentation. La plupart des
  bons éditeurs pour programmeurs le permettent.
\item Surtout, éviter de ne pas du tout indenter le code.
\end{itemize}


\section{Gestion des versions}
\label{sec:collaboration:git}



%%% Local Variables:
%%% mode: latex
%%% TeX-engine: xetex
%%% TeX-master: "programmer-avec-r"
%%% coding: utf-8
%%% End:
