\subsection{Fonction \code{uniroot}}
\label{optimisation:fonctions:uniroot}

La fonction \Fonction{uniroot} recherche la racine\index{racine!d'une
  fonction}\index{fonction!racine} d'une fonction dans un intervalle.
C'est donc la fonction de base pour trouver la solution (unique) de
l'équation $f(x) = 0$ dans un intervalle déterminé.


\subsection{Fonction \code{optimize}}
\label{optimisation:fonctions:optimize}

La fonction \Fonction{optimize} recherche le
minimum\index{minimum!local}\index{fonction!minimum local} local (par
défaut) ou le maximum\index{maximum!local}\index{fonction!maximum
  local} local d'une fonction dans un intervalle donné.


\subsection{Fonction \code{nlm}}
\label{optimisation:fonctions:nlm}

La fonction \Fonction{nlm} minimise une fonction non
linéaire\index{minimum!fonction non linéaire}\index{fonction!minimum}
sur un nombre arbitraire de paramètres.


\subsection{Fonction \code{nlminb}}
\label{optimisation:fonctions:nlminb}

La fonction \Fonction{nlminb} est similaire à \fonction{nlm}, sauf
qu'elle permet de spécifier des bornes inférieure ou supérieure
pour les paramètres. Attention, toutefois: les arguments de la
fonction ne sont ni les mêmes, ni dans le même ordre que ceux de
\code{nlm}.


\subsection{Fonction \code{optim}}
\label{optimisation:fonctions:optim}

La fonction \Fonction{optim}\index{fonction!optimisation} est l'outil
d'optimisation tout usage de R. À ce titre, la fonction est souvent
utilisée par d'autres fonctions. Elle permet de choisir parmi
plusieurs algorithmes d'optimisation différents et, selon l'algorithme
choisi, de fixer des seuils minimum ou maximum aux paramètres à
optimiser.


\subsection{\code{polyroot}}
\label{optimisation:fonctions:polyroot}

En terminant, un mot sur \code{polyroot()}, qui n'est pas à proprement
parler une fonction d'optimisation, mais qui pourrait être utilisée
dans ce contexte. La fonction \Fonction{polyroot} calcule toutes les
racines\index{racine!d'un polynôme} (complexes) du polynôme
$\sum_{i=0}^n a_i x^i$. Le premier argument est le vecteur des
coefficients $a_0, a_1, \dots, a_n$, dans cet ordre.

%%% Local Variables:
%%% mode: latex
%%% TeX-engine: xetex
%%% TeX-master: "introduction_programmation_r"
%%% coding: utf-8
%%% End:
