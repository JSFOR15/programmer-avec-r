\documentclass{beamer}
  \usepackage[T1]{fontenc}
  \usepackage[latin1]{inputenc}
  \usepackage[francais]{babel}
  \usepackage{icomma}
  \usepackage{vgmath,actu,amsmath,amsthm,url}
  \usepackage{palatino,mathpazo}
  \usepackage{textcomp,fourier-orns,mathabx}
  \usepackage{Sweave-sans-ae}
  \usepackage{threeparttable,expdlist}

  %%% Param�tres de beamer
  \usetheme{Rochester}
  \setbeamercovered{transparent}

  %%% Titre
  \title{Introduction � la programmation en S}
  \author{Vincent Goulet}
  \institute{�cole d'actuariat \\ Universit� Laval}
  \date{}

  %%% D�sactivation de certaines commandes
  \newcommand{\R}{}
  \newcommand{\Splus}{}
  \newcommand{\Index}[1]{}
  \newcommand{\indexargument}[1]{}
  \newcommand{\Indexargument}[1]{}
  \newcommand{\indexattribut}[1]{}
  \newcommand{\Indexattribut}[1]{}
  \newcommand{\indexclasse}[1]{}
  \newcommand{\Indexclasse}[1]{}
  \newcommand{\indexfonction}[1]{}
  \newcommand{\Indexfonction}[1]{}
  \newcommand{\indexmode}[1]{}
  \newcommand{\Indexmode}[1]{}
  \newcommand{\indexobjet}[1]{}
  \newcommand{\Indexobjet}[1]{} 
  \newcommand{\indexemacs}[1]{}
  \newcommand{\indexess}[1]{}
  
  %%% Styles pour les environnements d'exemples et al.
%   \theoremstyle{plain}
% %  \newtheorem{ex}{Exemple}[chapter]
%   \theoremstyle{remark}
%   \newtheorem*{rem}{Remarque}
%   \theoremstyle{definition}
%   \newtheorem*{astuce}{Astuce}
%   \newenvironment{ex}{\begin{example}}{\end{example}}
%   \newenvironment{sol}{\begin{proof}[Solution]}{\end{proof}}

  %%% Environnement pour les listes de commandes
  \newenvironment{ttscript}[1]{%
    \begin{list}{}{%
        \setlength{\labelsep}{1.5ex}
        \settowidth{\labelwidth}{\code{#1}}
        \setlength{\leftmargin}{\labelwidth}
        \addtolength{\leftmargin}{\labelsep}
        \setlength{\parsep}{0.5ex plus0.2ex minus0.2ex}
        \setlength{\itemsep}{0.3ex}
        \renewcommand{\makelabel}[1]{##1\hfill}}}
    {\end{list}}

  %%% Environnement pour les listes de structures de contr�le
  \newenvironment{struclist}{%
    \begin{description}[\breaklabel\setlabelstyle{\mdseries\ttfamily}%
      \setleftmargin{\parindent}]}
    {\end{description}}

  %%% Styles pour les noms de fonctions, code, etc.
  \newcommand{\code}[1]{\texttt{#1}}
  \newcommand{\attribut}[1]{\code{#1}}
  \newcommand{\Attribut}[1]{\code{#1}}
  \newcommand{\argument}[1]{\code{#1}}
  \newcommand{\Argument}[1]{\code{#1}}
  \newcommand{\classe}[1]{\code{#1}}
  \newcommand{\Classe}[1]{\code{#1}}
  \newcommand{\fonction}[1]{\code{#1}}
  \newcommand{\Fonction}[1]{\code{#1}}
  \renewcommand{\mode}[1]{\code{#1}}
  \newcommand{\Mode}[1]{\code{#1}}
  \newcommand{\objet}[1]{\code{#1}}
  \newcommand{\Objet}[1]{\code{#1}}
  \newcommand{\emacs}[1]{\code{#1}}
  \newcommand{\ess}[1]{\code{#1}}

\begin{document}

\begin{frame}
  \titlepage
\end{frame}

\lecture{Pr�sentation}{cours 1}
\part{PR�SENTATION DU LANGAGE S}

\begin{frame}
  \partpage
\end{frame}

\begin{frame}
  \frametitle{Sommaire}
  \tableofcontents
\end{frame}

\section{Le langage S}

\begin{frame}
  \frametitle{Le langage S}

  Le S est un langage pour �programmer avec des donn�es� d�velopp�
  chez Bell Laboratories (anciennement propri�t� de AT\&T, maintenant
  de Lucent Technologies).

  \begin{itemize}
  \item<2-> Pas seulement un �autre� environnement statistique, mais
    bien un langage de programmation complet et autonome.
  \item<3-> Inspir� de plusieurs langages, dont l'APL et le Lisp:
    \begin{itemize}
    \item interpr�t� (et non compil�);
    \item sans d�claration obligatoire des variables;
    \item bas� sur la notion de vecteur;
    \item particuli�rement puissant pour les applications
      math�matiques et statistiques (et donc actuarielles).
    \end{itemize}
  \end{itemize}
\end{frame}


\section{Les moteurs S}

\begin{frame}
  \frametitle{Les moteurs S}

  Il existe quelques �moteurs� ou dialectes du langage S.

  \begin{itemize}[<+->]
  \item Le plus connu est S-Plus, un logiciel commercial de Insightful
    Corporation. (Bell Labs octroie � Insightful la licence exclusive
    de leur syst�me S.)
  \item \textsf{R}, ou GNU S, est une version libre (\emph{Open
      Source}) �\emph{not unlike S}�.
  \end{itemize}

  S-Plus et \textsf{R} constituent tous deux des environnements
  int�gr�s de manipulation de donn�es, de calcul et de pr�paration de
  graphiques.
\end{frame}


\section[Documentation]{O� trouver de la documentation}

\begin{frame}
  \frametitle{O� trouver de la documentation}

  \begin{itemize}[<+->]
  \item S-Plus est livr� avec quatre livres, mais aucun ne s'av�re
    vraiment utile pour apprendre le langage S.
  \item Plusieurs livres --- en versions papier ou �lectronique,
    gratuits ou non --- ont �t� publi�s sur S-Plus et/ou \textsf{R}.
    On trouvera des listes exhaustives dans les sites de Insightful et
    du projet \textsf{R}.
  \end{itemize}
\end{frame}


\section[Interfaces]{Interfaces pour S-Plus et R}

\begin{frame}
  \frametitle{Interfaces pour S-Plus et R}

  S-Plus et \textsf{R} sont d'abord et avant tout des applications en
  ligne de commande (\texttt{sqpe.exe} et \texttt{rterm.exe} sous
  Windows).
  \pause

  \begin{itemize}[<+->]
  \item S-Plus poss�de toutefois une interface graphique �labor�e
    permettant d'utiliser le logiciel sans trop conna�tre le langage
    de programmation.
  \item \textsf{R} dispose �galement d'une interface graphique
    rudimentaire sous Windows et Mac OS.
  \item L'�dition s�rieuse de code S b�n�ficie cependant grandement
    d'un bon �diteur de texte.
  \end{itemize}
\end{frame}

\begin{frame}
  \begin{itemize}[<+->]
  \item � la question 6.2 de la foire aux questions (FAQ) de
    \textsf{R}, �Devrais-je utiliser \textsf{R} � l'int�rieur de
    Emacs?�, la r�ponse est: �Oui, d�finitivement.�
  \item Nous partageons cet avis, aussi apprendra-t-on � utiliser
    S-Plus ou \textsf{R} � l'int�rieur de GNU Emacs avec le mode
    ESS.
  \item Autre option: WinEdt (partagiciel) avec l'ajout R-WinEdt.
  \end{itemize}
\end{frame}


\section[Emacs et ESS]{Installation de Emacs avec ESS}

\begin{frame}
  \frametitle{Installation de Emacs avec ESS}

  \begin{itemize}[<+->]
  \item Pour une installation simplifi�e de Emacs et ESS, consulter le
    site Internet

    \url{http://vgoulet.act.ulaval.ca/pub/emacs/}
      
    On y trouve une version modifi�e de GNU Emacs et des instructions
    d'installation d�taill�es.
  \item L'annexe A du document d'accompagnement pr�sente les plus
    importantes commandes � conna�tre pour utiliser Emacs et le mode
    ESS.
  \end{itemize}
\end{frame}


\section[D�marrer et quitter]{D�marrer et quitter S-Plus ou R}

\begin{frame}
  \frametitle{D�marrer et quitter S-Plus ou R}

  \begin{itemize}[<+->]
  \item Pour d�marrer \textsf{R} \R � l'int�rieur de Emacs:

    \texttt{M-x R RET}

    puis sp�cifier un dossier de travail. Une console \textsf{R} est
    ouverte dans un \emph{buffer} nomm� \texttt{*R*}.
  \item Pour d�marrer S-Plus sous Windows, consulter l'annexe B du
    document d'accompagnement.
  \item Pour quitter, deux options sont disponibles:
    \begin{enumerate}[<+->]
    \item Taper \fonction{q()} � la ligne de commande.
    \item Dans Emacs, faire \ess{C-c C-q}. ESS va alors s'occuper de
      fermer le processus S ainsi que tous les \emph{buffers} associ�s
      � ce processus.
    \end{enumerate}
  \end{itemize}
\end{frame}


\section{Strat�gies de travail}

\begin{frame}
  \frametitle{Strat�gies de travail}

  Il existe principalement deux fa�ons de travailler avec S-Plus et
  \textsf{R}.

  \begin{enumerate}
  \item Le code est virtuel et les objets sont r�els.
  \item Le code est r�el et les objets sont virtuels.
  \end{enumerate}
\end{frame}

\begin{frame}
  \frametitle{Code virtuel, objets r�els}

  \begin{itemize}
  \item C'est l'approche qu'encouragent les interfaces graphiques,
    mais aussi la moins pratique � long terme.
  \item On entre des expressions directement � la ligne de commande
    pour les �valuer imm�diatement.
  \item Les objets cr��s au cours d'une session de travail sont
    sauvegard�s.
  \item Par contre, le code utilis� pour cr�er ces objets est perdu
    lorsque l'on quitte S-Plus ou \textsf{R}, � moins de sauvegarder
    celui-ci dans des fichiers.
  \end{itemize}
\end{frame}

\begin{frame}
  \frametitle{Code r�el, objets virtuels}

  \begin{itemize}
  \item C'est l'approche que nous favoriserons.
  \item Le travail se fait essentiellement dans des fichiers de script
    (de simples fichiers de texte) dans lesquels sont sauvegard�es les
    expressions (parfois complexes!) et le code des fonctions
    personnelles.
  \item Les objets sont cr��s au besoin en ex�cutant le code.
  \end{itemize}
\end{frame}

\begin{frame}
  Emacs permet ici de passer efficacement des fichiers de script �
  l'ex�cution du code:
  \begin{enumerate}[<+->]
  \item D�marrer un processus S-Plus (\texttt{M-x Sqpe}) ou \textsf{R}
    (\texttt{M-x R}) et sp�cifier le dossier de travail.
  \item Ouvrir un fichier de script avec \ess{C-x C-f}. Pour cr�er un
    nouveau fichier, ouvrir un fichier n'existant pas.
  \item Positionner le curseur sur une expression et faire \ess{C-c
      C-n} pour l'�valuer.
  \item Le r�sultat appara�t dans le \emph{buffer} \texttt{*S+6*} ou
    \texttt{*R*}.
  \end{enumerate}
\end{frame}


\section[Gestion des projets]{Gestion des projets ou environnements de
  travail}

\begin{frame}
  \frametitle{Gestion des projets ou environnements de travail}

  S-Plus et \textsf{R} ont une mani�re diff�rente, mais tout aussi
  particuli�re de sauvegarder les objets cr��s au cours d'une session
  de travail.
  \pause

  \begin{itemize}[<+->]
  \item Tous deux doivent travailler dans un dossier et non avec des
    fichiers individuels.
  \item Dans S-Plus, \Splus tout objet cr�� au cours d'une session de
    travail est sauvegard� de fa�on permanente sur le disque dur dans
    le sous-dossier \texttt{\_\_Data} du dossier de travail.
  \item Dans \textsf{R}, \R les objets cr��s sont conserv�s en m�moire
    jusqu'� ce que l'on quitte l'application ou que l'on enregistre le
    travail avec la commande \fonction{save.image()}. L'environnement
    de travail (\emph{workspace}) est alors sauvegard� dans le fichier
    \texttt{.RData} dans le dossier de travail.
  \end{itemize}
\end{frame}

\begin{frame}
  Le dossier de travail est d�termin� au lancement de l'application.
  \begin{itemize}[<+->]
  \item Avec Emacs et ESS on doit sp�cifier le dossier de travail �
    chaque fois que l'on d�marre un processus S-Plus ou R.
  \item Les interfaces graphiques permettent �galement de sp�cifier le
    dossier de travail.
  \end{itemize}
\end{frame}


\section{Consulter l'aide en ligne}

\begin{frame}[fragile]
  Les rubriques d'aide des diverses fonctions disponibles dans S-Plus
  et \textsf{R} contiennent une foule d'informations ainsi que des
  exemples d'utilisation. Leur consultation est tout � fait
  essentielle.
  \pause

  \begin{itemize}[<+->]
  \item Pour consulter la rubrique d'aide de la fonction \code{foo},
    on peut entrer � la ligne de commande
\begin{Sinput}
> ?foo
\end{Sinput}
  \item Dans Emacs, \code{C-c C-v foo RET} ouvrira
    la rubrique d'aide de la fonction \code{foo} dans un nouveau
    \emph{buffer}.
  \end{itemize}
\end{frame}

%%% Local Variables:
%%% mode: latex
%%% TeX-master: "introduction_programmation_S_slides"
%%% End:


%\part{BASES DU LANGAGE S}

\begin{frame}
  \partpage
\end{frame}

\begin{frame}
  \frametitle{Sommaire}
  \tableofcontents
\end{frame}

\section{Commandes S}

\begin{frame}[fragile]
  \frametitle{Affectations et expressions}

  Toute commande S est soit une \emph{affectation}\index{affectation},
  soit une \emph{expression}\index{expression}.
  \begin{itemize}
  \item Normalement, une expression est imm�diatement �valu�e et le
    r�sultat est affich� � l'�cran:
\begin{Schunk}
\begin{Sinput}
> 2 + 3
\end{Sinput}
\begin{Soutput}
[1] 5
\end{Soutput}
\begin{Sinput}
> pi
\end{Sinput}
\begin{Soutput}
[1] 3.141593
\end{Soutput}
\begin{Sinput}
> cos(pi/4)
\end{Sinput}
\begin{Soutput}
[1] 0.7071068
\end{Soutput}
\end{Schunk}
  \end{itemize}
\end{frame}

\begin{frame}[fragile]
  \frametitle{Affectations et expressions}

  \begin{itemize}
  \item Lors d'une affectation, une expression est �valu�e, mais le
    r�sultat est stock� dans un objet (variable) et rien n'est affich�
    � l'�cran.
  \item Le symbole d'affectation est \Fonction{<-} (ou
    \Fonction{->}).
\begin{Schunk}
\begin{Sinput}
> a <- 5
> a
\end{Sinput}
\begin{Soutput}
[1] 5
\end{Soutput}
\begin{Sinput}
> b <- a
> b
\end{Sinput}
\begin{Soutput}
[1] 5
\end{Soutput}
\end{Schunk}
  \end{itemize}
\end{frame}


\begin{frame}[fragile]
  \frametitle{Deux symboles d'affectation � �viter}

  \begin{itemize}
  \item<+-> L'op�rateur \code{=}
    \begin{itemize}
    \item peut porter � confusion.
    \end{itemize}
  \item<+-> Le caract�re \code{\_}
    \begin{itemize}
    \item permis dans S-Plus, mais plus dans \textsf{R} depuis la version
      1.8.0
    \item emploi  fortement d�courag�
    \item rend le code difficile � lire
    \item dans le mode ESS de Emacs, taper ce caract�re g�n�re
      carr�ment \verb*| <- |.
    \end{itemize}
  \end{itemize}
\end{frame}

\begin{frame}[fragile=singleslide]
  \frametitle{Astuce} 

  Pour affecter le r�sultat d'un calcul dans un objet et en m�me temps
  voir ce r�sultat, placer l'affectation entre
  parenth�ses. \\[\baselineskip]

  L'op�ration d'affectation devient alors une nouvelle expression:
\begin{Schunk}
\begin{Sinput}
> (a <- 2 + 3)
\end{Sinput}
\begin{Soutput}
[1] 5
\end{Soutput}
\end{Schunk}
\end{frame}


\section{Conventions pour les noms d'objets}

\begin{frame}
  \frametitle{Caract�res permis dans les noms d'objets}

  \begin{itemize}[<+->]
  \item Les lettres a--z, A--Z
  \item Les chiffres 0--9
  \item Le point �.�
  \item �\code{\_}� est maintenant permis dans \textsf{R}, mais son
    utilisation est d�courag�e.
  \end{itemize}
\end{frame}

\begin{frame}
  \frametitle{R�gles pour les noms d'objets}

  \begin{itemize}
  \item Les noms d'objets ne peuvent commencer par un chiffre.
  \item<2-> Le S est sensible � la casse: \code{foo}, \code{Foo} et
    \code{FOO} sont trois objets distincts.
  \item<2-> Moyen simple d'�viter des erreurs li�es � la casse: employer
    seulement des lettres minuscules.
  \end{itemize}
\end{frame}

\begin{frame}
  \frametitle{Noms d�j� utilis�s et r�serv�s}

  \begin{itemize}[<+->]
  \item Certains noms sont utilis�s par le syst�me, aussi vaut-il
    mieux �viter de les utiliser. En particulier, �viter d'utiliser
    \begin{center}
      \code{c}, \code{q}, \code{t}, \code{C}, \code{D}, \code{I},
      \code{diff}, \code{length}, \code{mean}, \code{pi},
      \code{range}, \code{var}.
    \end{center}
  \item Certains mots sont r�serv�s pour le syst�me et il est interdit
    de les utiliser comme nom d'objet:
    \begin{center}
      \code{Inf}, \code{NA}, \code{NaN},
      \code{NULL} \\
      \code{break}, \code{else}, \code{for}, \code{function},
      \code{if}, \code{in}, \code{next}, \code{repeat}, \code{return},
      \code{while}.
    \end{center}
  \item Dans S-Plus 6.1 et plus, \Splus \code{T} et \objet{TRUE}
    (vrai), ainsi que \code{F} et \objet{FALSE} (faux) sont �galement
    des noms r�serv�s.
  \end{itemize}
\end{frame}

\begin{frame}[fragile]
  \frametitle{\texttt{TRUE} et \texttt{FALSE} dans R}

  \begin{itemize}[<+->]
  \item Dans \textsf{R}, \R les noms \code{TRUE} et \code{FALSE}
    sont �galement r�serv�s.
  \item Les variables \code{T} et \code{F} prennent par d�faut les
    valeurs \code{TRUE} et \code{FALSE}, respectivement, mais peuvent
    �tre r�affect�es.
\begin{Schunk}
\begin{Sinput}
> T
\end{Sinput}
\begin{Soutput}
[1] TRUE
\end{Soutput}
\end{Schunk}
\begin{Schunk}
\begin{Sinput}
> TRUE <- 3
\end{Sinput}
\end{Schunk}
\begin{Schunk}
\begin{Soutput}
Erreur dans TRUE <- 3 : membre gauche de
l'assignation (do_set) incorrect
\end{Soutput}
\end{Schunk}
\begin{Schunk}
\begin{Sinput}
> (T <- 3)
\end{Sinput}
\begin{Soutput}
[1] 3
\end{Soutput}
\end{Schunk}
  \end{itemize}
\end{frame}

\section{Les objets S}

\begin{frame}
  \frametitle{Tout est un objet}
  
  \begin{itemize}
  \item Tout dans le langage S est un objet, m�me les fonctions et les
    op�rateurs.
  \item Les objets poss�dent au minimum un \alert{mode} et une
    \alert{longueur}.
  \end{itemize}
\end{frame}

\begin{frame}[fragile]
  \frametitle{Mode et longueur}

  \begin{itemize}
  \item Le mode d'un objet est obtenu avec la fonction \Fonction{mode}.
\begin{Schunk}
\begin{Sinput}
> v <- c(1, 2, 5, 9)
> mode(v)
\end{Sinput}
\begin{Soutput}
[1] "numeric"
\end{Soutput}
\end{Schunk}
  \item La longueur d'un objet est obtenue avec la fonction
    \Fonction{length}.
\begin{Schunk}
\begin{Sinput}
> length(v)
\end{Sinput}
\begin{Soutput}
[1] 4
\end{Soutput}
\end{Schunk}
  \item Certains objets sont �galement dot�s d'un ou plusieurs
    \alert{attributs}.
  \end{itemize}
\end{frame}

\subsection{Modes et types de donn�es}


\begin{frame}
  \frametitle{Modes et types de donn�es }

  \begin{itemize}
  \item Le mode prescrit ce qu'un objet peut contenir.
  \item Un objet ne peut donc avoir qu'un seul mode.
  \item Modes disponibles en S:
  \end{itemize}
  \begin{center}
    \begin{tabular}{ll}
      \toprule
      \Mode{numeric}   & nombres r�els \\
      \Mode{complex}   & nombres complexes \\
      \Mode{logical}   & valeurs bool�ennes (vrai/faux) \\
      \Mode{character} & cha�nes de caract�res \\
      \Mode{function}  & fonction \\
      \Mode{list}      & donn�es quelconques \\
      \bottomrule
    \end{tabular}
  \end{center}
\end{frame}


\subsection{Longueur}


\begin{frame}[fragile]
  \frametitle{Longueur d'un objet}
  
  La longueur\Index{longueur} d'un objet est �gale au nombre
  d'�l�ments qu'il contient.

  \begin{itemize}[<2->]
  \item La longueur d'une cha�ne de caract�res est toujours 1. Un
    objet de mode \code{character} doit contenir plusieurs cha�nes
    de caract�res pour que sa longueur soit sup�rieure � 1.
\begin{Schunk}
\begin{Sinput}
> v <- "actuariat"
> length(v)
\end{Sinput}
\begin{Soutput}
[1] 1
\end{Soutput}
\begin{Sinput}
> v <- c("a", "c", "t", "u", "a", "r", "i", 
+     "a", "t")
> length(v)
\end{Sinput}
\begin{Soutput}
[1] 9
\end{Soutput}
\end{Schunk}
  \end{itemize}
\end{frame}

\begin{frame}[fragile]
  \frametitle{Objet vide}
  
  \begin{itemize}[<+->]
  \item Un objet peut �tre de longueur 0.
  \item Doit alors �tre interpr�t� comme un contenant
    vide.
\begin{Schunk}
\begin{Sinput}
> v <- numeric(0)
> length(v)
\end{Sinput}
\begin{Soutput}
[1] 0
\end{Soutput}
\end{Schunk}
  \end{itemize}
\end{frame}


\subsection{Attributs}

\begin{frame}
  \frametitle{D�finition et liste des attributs}
  
  \begin{itemize}
  \item �l�ments d'information additionnels li�s � cet objet.
  \item Attributs les plus fr�quemment rencontr�s:
  \end{itemize}
  \begin{center}
    \begin{tabular}{ll}
      \toprule
      \Attribut{class}    &
      affecte le comportement d'un objet \\
      \Attribut{dim}      &
      dimensions\index{dimension} des matrices et tableaux \\
      \Attribut{dimnames} &
      �tiquettes\index{etiquette@�tiquette} des dimensions des matrices
      et tableaux \\
      \Attribut{names}    &
      �tiquettes des �l�ments d'un objet \\
      \bottomrule
    \end{tabular}
  \end{center}
\end{frame}

\subsection{L'objet sp�cial \code{NA}}

\begin{frame}
  \frametitle{Repr�sentation des donn�es manquantes}

  \Objet{NA} est fr�quemment utilis� pour repr�senter les donn�es
  manquantes.
  \pause

  \begin{itemize}[<+->]
  \item Son mode est \Mode{logical}.
  \item Toute op�ration impliquant une donn�e \code{NA} a comme
    r�sultat \code{NA}.
  \item Certaines fonctions (\fonction{sum}, \fonction{mean}, par
    exemple), ont par cons�quent un argument \Argument{na.rm} qui,
    lorsque \code{TRUE}, �limine les donn�es manquantes avant de faire
    un calcul.
  \item La fonction \Fonction{is.na} permet de tester si les �l�ments
    d'un objet sont \code{NA} ou non.
  \end{itemize}
\end{frame}

\subsection{L'objet sp�cial \code{NULL}}

\begin{frame}
  \frametitle{Repr�sentation du vide}

  \Objet{NULL} repr�sente �rien�, ou le vide.
  \pause

  \begin{itemize}[<+->]
  \item Son mode est \Mode{NULL}.
  \item Sa longueur est 0.
  \item Diff�rent d'un objet vide:
    \begin{itemize}
    \item un objet de longueur 0 est un contenant vide;
    \item \code{NULL} est �pas de contenant�.
    \end{itemize}
  \item La fonction \Fonction{is.null} teste si un objet est
    \code{NULL} ou non.
  \end{itemize}
\end{frame}

\section{Vecteurs}

\begin{frame}[fragile]
  \frametitle{En S, tout est un vecteur}

  \begin{itemize}
  \item Dans un vecteur simple, tous les �l�ments doivent �tre du m�me
    mode.
  \item Il est possible (et souvent souhaitable) de donner une
    �tiquette � chacun des �l�ments d'un vecteur.
\begin{Schunk}
\begin{Sinput}
> (v <- c(a = 1, b = 2, c = 5))
\end{Sinput}
\begin{Soutput}
a b c
1 2 5
\end{Soutput}
\begin{Sinput}
> v <- c(1, 2, 5) 
> names(v) <- c("a", "b", "c") 
> v
\end{Sinput}
\begin{Soutput}
a b c
1 2 5
\end{Soutput}
\end{Schunk}
  \end{itemize}
\end{frame}

\begin{frame}
  \frametitle{Et comment cr�e-t-on ces vecteurs?}

  Les fonctions de base pour cr�er des vecteurs sont
  \begin{itemize}
  \item \Fonction{c} (concat�nation)
  \item \Fonction{numeric} (vecteur de mode \Mode{numeric})
  \item \Fonction{logical} (vecteur de mode \Mode{logical})
  \item \Fonction{character} (vecteur de mode \Mode{character}).
  \end{itemize}
\end{frame}

\begin{frame}[fragile]
  \frametitle{Indi�age d'un vecteur}

  \begin{itemize}[<+->]
  \item Se fait avec \fonction{[\ ]}.
  \item On peut extraire un �l�ment d'un vecteur par
    \begin{itemize}
    \item sa position ou
    \item son �tiquette, si elle existe (auquel cas cette approche est
      beaucoup plus s�re).
\begin{Schunk}
\begin{Sinput}
> v[3]
\end{Sinput}
\begin{Soutput}
c 
5 
\end{Soutput}
\begin{Sinput}
> v["c"]
\end{Sinput}
\begin{Soutput}
c 
5 
\end{Soutput}
\end{Schunk}
    \end{itemize}
  \end{itemize}
\end{frame}

\section{Matrices et tableaux}
\label{bases:matrices}

Une matrice\index{matrice} ou, de fa�on plus g�n�rale, un
tableau\index{tableau} (\emph{array}) n'est rien d'autre qu'un vecteur
dot� d'un attribut \attribut{dim}. � l'interne, une matrice est donc
stock�e sous forme de vecteur.

\begin{itemize}
\item La fonction de base pour cr�er des matrices est
  \Fonction{matrix}.
\item La fonction de base pour cr�er des tableaux est
  \Fonction{array}.
\item \alert{Important}: \warning les matrices et tableaux sont remplis
  en faisant d'abord varier la premi�re dimension, puis la seconde,
  etc. Pour les matrices, cela revient � remplir par colonne.
\begin{Schunk}
\begin{Sinput}
> matrix(1:6, nrow = 2, ncol = 3)
\end{Sinput}
\begin{Soutput}
     [,1] [,2] [,3]
[1,]    1    3    5
[2,]    2    4    6
\end{Soutput}
\begin{Sinput}
> matrix(1:6, nrow = 2, ncol = 3, byrow = TRUE)
\end{Sinput}
\begin{Soutput}
     [,1] [,2] [,3]
[1,]    1    2    3
[2,]    4    5    6
\end{Soutput}
\end{Schunk}
\item On extrait les �l�ments d'une matrice en pr�cisant leurs
  positions sous la forme (ligne, colonne) dans la matrice, ou encore
  leurs positions dans le vecteur sous-jacent.
\begin{Schunk}
\begin{Sinput}
> (m <- matrix(c(40, 80, 45, 21, 55, 32), 
+     nrow = 2, ncol = 3))
\end{Sinput}
\begin{Soutput}
     [,1] [,2] [,3]
[1,]   40   45   55
[2,]   80   21   32
\end{Soutput}
\begin{Sinput}
> m[1, 2]
\end{Sinput}
\begin{Soutput}
[1] 45
\end{Soutput}
\begin{Sinput}
> m[3]
\end{Sinput}
\begin{Soutput}
[1] 45
\end{Soutput}
\end{Schunk}
\item La fonction \Fonction{rbind} permet de fusionner verticalement
  deux matrices (ou plus) ayant le m�me nombre de colonnes.
\begin{Schunk}
\begin{Sinput}
> n <- matrix(1:9, nrow = 3)
> rbind(m, n)
\end{Sinput}
\begin{Soutput}
     [,1] [,2] [,3]
[1,]   40   45   55
[2,]   80   21   32
[3,]    1    4    7
[4,]    2    5    8
[5,]    3    6    9
\end{Soutput}
\end{Schunk}
\item La fonction \Fonction{cbind} permet de fusionner horizontalement
  deux matrices (ou plus) ayant le m�me nombre de lignes.
\begin{Schunk}
\begin{Sinput}
> n <- matrix(1:4, nrow = 2)
> cbind(m, n)
\end{Sinput}
\begin{Soutput}
     [,1] [,2] [,3] [,4] [,5]
[1,]   40   45   55    1    3
[2,]   80   21   32    2    4
\end{Soutput}
\end{Schunk}
\end{itemize}


\section{Listes}
\label{bases:listes}

Une liste\index{liste} est un type de vecteur sp�cial dont les
�l�ments peuvent �tre de n'importe quel mode, y compris le mode
\Mode{list} (ce qui permet d'embo�ter des listes).

\begin{itemize}
\item La fonction de base pour cr�er des listes est \Fonction{list}.
\item Il est g�n�ralement pr�f�rable de nommer les �l�ments d'une
  liste. Il est en effet plus simple et s�r d'extraire les �l�ments
  par leur �tiquette.
\item L'extraction\Index{indi�age!liste} des �l�ments d'une liste peut
  se faire de deux fa�ons:
  \begin{enumerate}
  \item avec des doubles crochets \fonction{[[\ ]]};
  \item par leur �tiquette avec \code{nom.liste\$etiquette.element}.
  \end{enumerate}
\item La fonction \Fonction{unlist} convertit une liste en un vecteur
  simple. Attention, cette fonction peut �tre destructrice si la
  structure de la liste est importante.
\end{itemize}


\section{\emph{Data frames}}
\label{bases:dataframes}

Les vecteurs, matrices, tableaux (\emph{arrays}) et listes sont les
types d'objets les plus fr�quemment utilis�s en S pour la
programmation de fonctions personnelles ou la simulation. L'analyse de
donn�es --- la r�gression lin�aire, par exemple --- repose toutefois
davantage sur les \emph{data frames}\Index{data frame}.

\begin{itemize}
\item Un \emph{data frame} est une liste de classe \classe{data.frame}
  dont tous les �l�ments sont de la m�me longueur.
\item G�n�ralement repr�sent� sous forme d'un tableau � deux
  dimensions (visuellement similaire � une matrice). Chaque �l�ment de
  la liste sous-jacente correspond � une colonne.
\item On peut donc obtenir les �tiquettes des colonnes avec la fonction
  \fonction{names} (ou \fonction{colnames} \R dans \textsf{R}).  Les
  �tiquettes des lignes sont quant � elles obtenues avec
  \fonction{row.names} (ou \fonction{rownames} dans \textsf{R}).
\item Plus g�n�ral qu'une matrice puisque les colonnes peuvent �tre de
  modes diff�rents (\Mode{numeric}, \Mode{complex}, \Mode{character}
  ou \Mode{logical}).
\item Peut �tre indic� � la fois comme une liste et comme une matrice.
\item Cr�� avec la fonction \Fonction{data.frame} ou
  \Fonction{as.data.frame} (pour convertir une matrice en \emph{data
    frame}, par exemple).
\item Les fonctions \fonction{rbind} et \fonction{cbind} peuvent �tre
  utilis�es pour ajouter des lignes ou des colonnes, respectivement.
\item On peut rendre les colonnes d'un \emph{data frame} (ou d'une
  liste) visibles dans l'espace de travail avec la fonction
  \Fonction{attach}, puis les masquer avec \Fonction{detach}.
\end{itemize}

Ce type d'objet est moins important lors de l'apprentissage du langage
de programmation.




\section{Indi�age}
\label{bases:indicage}

L'indi�age des vecteurs\Index{indi�age!vecteur} et
matrices\Index{indi�age!matrice} a d�j� �t� bri�vement pr�sent� aux
sections \ref{bases:vecteurs} et \ref{bases:matrices}. Cette section
contient plus de d�tails sur cette proc�dure des plus communes lors de
l'utilisation du langage S. On se concentre toutefois sur le
traitement des vecteurs. Se r�f�rer �galement � \citet[section
2.3]{MASS} pour de plus amples d�tails.

Il existe quatre fa�ons d'indicer un vecteur dans le langage S. Dans
tous les cas, l'indi�age se fait � l'int�rieur de crochets \Fonction{[\ ]}.
\begin{enumerate}
\item Avec un vecteur d'entiers positifs. Les �l�ments se trouvant aux
  positions correspondant aux entiers sont extraits du vecteur, dans
  l'ordre. C'est la technique la plus courante.
\begin{Schunk}
\begin{Sinput}
> letters[c(1:3, 22, 5)]
\end{Sinput}
\begin{Soutput}
[1] "a" "b" "c" "v" "e"
\end{Soutput}
\end{Schunk}
\item Avec un vecteur d'entiers n�gatifs. Les �l�ments se trouvant aux
  positions correspondant aux entiers n�gatifs sont alors
  \alert{�limin�s} du vecteur.
\begin{Schunk}
\begin{Sinput}
> letters[c(-(1:3), -5, -22)]
\end{Sinput}
\begin{Soutput}
 [1] "d" "f" "g" "h" "i" "j" "k" "l" "m" "n" "o" "p"
[13] "q" "r" "s" "t" "u" "w" "x" "y" "z"
\end{Soutput}
\end{Schunk}
\item Avec un vecteur bool�en. Le vecteur d'indi�age doit alors �tre
  de la m�me longueur que le vecteur indic�. Les �l�ments
  correspondant � une valeur \code{TRUE} sont extraits du vecteur,
  alors que ceux correspondant � \code{FALSE} sont �limin�s.
\begin{Schunk}
\begin{Sinput}
> letters > "f" & letters < "q"
\end{Sinput}
\begin{Soutput}
 [1] FALSE FALSE FALSE FALSE FALSE FALSE  TRUE  TRUE
 [9]  TRUE  TRUE  TRUE  TRUE  TRUE  TRUE  TRUE  TRUE
[17] FALSE FALSE FALSE FALSE FALSE FALSE FALSE FALSE
[25] FALSE FALSE
\end{Soutput}
\begin{Sinput}
> letters[letters > "f" & letters < "q"]
\end{Sinput}
\begin{Soutput}
 [1] "g" "h" "i" "j" "k" "l" "m" "n" "o" "p"
\end{Soutput}
\end{Schunk}
\item Avec une cha�ne de caract�res. Utile pour extraire les �l�ments
  d'un vecteur � condition que ceux-ci soient nomm�s.
\begin{Schunk}
\begin{Sinput}
> x <- c(Rouge = 2, Bleu = 4, Vert = 9, Jaune = -5)
> x[c("Bleu", "Jaune")]
\end{Sinput}
\begin{Soutput}
 Bleu Jaune 
    4    -5 
\end{Soutput}
\end{Schunk}
\end{enumerate}

%%% Local Variables: 
%%% mode: latex
%%% TeX-master: "introduction_programmation_S_slides"
%%% End: 

%\section{Op�rateurs et fonctions}

\begin{frame}
  \frametitle{Une liste non exhaustive}

  \begin{itemize}
  \item Principaux op�rateurs arithm�tiques, fonctions math�matiques
    et structures de contr�les offertes par le S.
  \item Liste loin d'�tre exhaustive.
  \item Consulter aussi la section \texttt{See Also} des rubriques
    d'aide.
  \end{itemize}
\end{frame}


\subsection{Op�rations arithm�tiques}

\begin{frame}[fragile=singleslide]
  \frametitle{L'unit� de base est le vecteur}
  
  \begin{itemize}
  \item Les op�rations sur les vecteurs sont effectu�es \alert{�l�ment
      par �l�ment}:
\begin{Schunk}
\begin{Sinput}
> c(1, 2, 3) + c(4, 5, 6)
\end{Sinput}
\begin{Soutput}
[1] 5 7 9
\end{Soutput}
\begin{Sinput}
> 1:3 * 4:6
\end{Sinput}
\begin{Soutput}
[1]  4 10 18
\end{Soutput}
\end{Schunk}
  \end{itemize}
\end{frame}

\begin{frame}[fragile=singleslide]
  \frametitle{Recyclage des vecteurs}
  
  \begin{itemize}
  \item Si les vecteurs impliqu�s dans une expression arithm�tique ne
    sont pas de la m�me longueur, les plus courts sont
    \alert{recycl�s}.
  \item Particuli�rement apparent avec les vecteurs de longueur 1:
\begin{Schunk}
\begin{Sinput}
> 1:10 + 2
\end{Sinput}
\begin{Soutput}
 [1]  3  4  5  6  7  8  9 10 11 12
\end{Soutput}
\begin{Sinput}
> 1:10 + rep(2, 10)
\end{Sinput}
\begin{Soutput}
 [1]  3  4  5  6  7  8  9 10 11 12
\end{Soutput}
\end{Schunk}
  \end{itemize}
\end{frame}

\begin{frame}[fragile=singleslide]
  \frametitle{Longueur du plus long vecteur multiple de celle des
    autres vecteurs}

  \begin{itemize}
  \item Les vecteurs les plus courts sont recycl�s un nombre entier de
    fois:
\begin{Schunk}
\begin{Sinput}
> 1:10 + 1:5 + c(2, 4)
\end{Sinput}
\begin{Soutput}
[1]  4  8  8 12 12 11 11 15 15 19
\end{Soutput}
\begin{Sinput}
> 1:10 + rep(1:5, 2) + rep(c(2, 4), 5)
\end{Sinput}
\begin{Soutput}
[1]  4  8  8 12 12 11 11 15 15 19
\end{Soutput}
\end{Schunk}
  \end{itemize}
\end{frame}

\begin{frame}[fragile=singleslide]
  \frametitle{Sinon...}

  \begin{itemize}
  \item Recyclage un nombre fractionnaire de fois et un avertissement
    est affich�:
\begin{Schunk}
\begin{Sinput}
> 1:10 + c(2, 4, 6)
\end{Sinput}
\begin{Soutput}
[1]  3  6  9  6  9 12  9 12 15 12
Message d'avis :
la longueur de l'objet le plus long n'est 
pas un multiple de la longueur de l'objet 
le plus court in: 1:10 + c(2, 4, 6)
\end{Soutput}
\end{Schunk}
  \end{itemize}
\end{frame}


\subsection{Op�rateurs}

\begin{frame}[fragile=singleslide]
  \frametitle{Op�rateurs math�matiques et logiques les plus
    fr�quemment employ�s}

  Ordre d�croissant de priorit� des op�rations.

  \begin{center}
    \renewcommand{\arraystretch}{1.1}
    \begin{tabular}{lp{30ex}}
      \toprule
      \Fonction{\^} ou \Fonction{**} & puissance \\
      \Fonction{-} & changement de signe \\
      \Fonction{*} \Fonction{/} & multiplication, division \\
      \Fonction{+} \Fonction{-} & addition, soustraction \\
      \Fonction{\%*\%} \Fonction{\%\%} \Fonction{\%/\%} & produit
      matriciel, modulo, division enti�re \\
      \Fonction{<} \Fonction{<=} \Fonction{==} \Fonction{>=}
      \Fonction{>} \verb|!=| & plus petit, plus petit ou �gal, �gal,
      plus grand ou �gal, plus grand, diff�rent de \\
      \verb|!| & n�gation logique \\
      \Fonction{\&} \Fonction{|} & �et� logique, �ou� logique \\
      \bottomrule
    \end{tabular}
  \end{center}
\end{frame}

\subsection{Appels de fonctions}

\begin{frame}
  \frametitle{Comment sp�cifier les arguments d'une fonction}

  \begin{itemize}[<+->]
  \item Pas de limite pratique au nombre d'arguments.
  \item Arguments peuvent �tre sp�cifi�s dans l'ordre �tabli dans la
    d�finition de la fonction.
  \item Plus prudent et \alert{fortement recommand�} de sp�cifier les
    arguments par leur nom, surtout apr�s les deux ou trois premiers
    arguments.
  \item N�cessaire de nommer les arguments s'ils ne sont pas appel�s
    dans l'ordre.
  \item Certains arguments ont une valeur par d�faut qui sera utilis�e
    si l'argument n'est pas sp�cifi�.
  \end{itemize}
\end{frame}

\subsection{Exemple}

\begin{frame}[fragile]
  \frametitle{Exemple}

  D�finition de la fonction \texttt{matrix}:

\begin{semiverbatim}
    matrix({\color{red}data=NA}, {\color{blue}nrow=1}, {\color{magenta}ncol=1},
           {\color{orange}byrow=FALSE}, {\color{brown}dimnames=NULL})
\end{semiverbatim}
  \pause

  \begin{itemize}[<+->]
  \item Chaque argument a une valeur par d�faut (ce n'est pas toujours
    le cas).
  \item Ainsi, un appel � \code{matrix} sans argument r�sulte en
\begin{Schunk}
\begin{Sinput}
> matrix()
\end{Sinput}
\begin{Soutput}
     [,1]
[1,]   NA
\end{Soutput}
\end{Schunk}
  \end{itemize}
\end{frame}

\begin{frame}[fragile=singleslide]

  \begin{itemize}
  \item Appel plus �labor� utilisant tous les arguments. Le premier
    argument est rarement nomm�.
\begin{Schunk}
\begin{Sinput}
> matrix(1:6, nrow = 2, ncol = 3, 
+  byrow = TRUE, 
+  dimnames = list(c("Gauche", "Droit"), 
+    c("Rouge", "Vert", "Bleu")))
\end{Sinput}
\begin{Soutput}
       Rouge Vert Bleu
Gauche     1    2    3
Droit      4    5    6
\end{Soutput}
\end{Schunk}
  \end{itemize}
\end{frame}


\subsection{Quelques fonctions utiles}

\begin{frame}
  \frametitle{Syst�me de classement des fonctions}

  \begin{itemize}[<+->]
  \item Diff�re entre S-Plus et \textsf{R}.
  \item Dans S-Plus, les fonctions sont class�es dans des
    \alert{sections} d'une biblioth�que (\emph{library}).
  \item Dans \textsf{R}, un ensemble de fonctions est appel� un
    \alert{package}.
  \item Par d�faut, \textsf{R} charge en m�moire quelques packages de
    la biblioth�que seulement.
  \item Cela �conomise l'espace m�moire et acc�l�re le d�marrage.
  \item On charge de nouveaux packages en m�moire avec la fonction
    \fonction{library}.
  \end{itemize}
\end{frame}

\subsection{Manipulation de vecteurs}

\begin{frame}
  \frametitle{Manipulation de vecteurs}

  \begin{ttscript}{unique}
  \item[\Fonction{seq}] g�n�ration de suites de nombres\index{suite de
      nombres}
  \item[\Fonction{rep}] r�p�tition\index{repetition@r�p�tition de
      valeurs} de valeurs ou de vecteurs
  \item[\Fonction{sort}] tri\index{tri} en ordre croissant ou
    d�croissant
  \item[\Fonction{order}] positions dans un vecteur des valeurs en
    ordre croissant ou d�croissant
  \item[\Fonction{rank}] rang\index{rang} des �l�ments d'un vecteur en
    ordre croissant ou d�croissant
  \item[\Fonction{rev}] renverser\index{renverser un vecteur} un
    vecteur
  \item[\Fonction{head}] extraction\index{extraction!premi�res
      valeurs} des $n$ premi�res valeurs ou suppression des $n$
    derni�res (\textsf{R} seulement)
    \index{extraction|seealso{indi�age}}
  \item[\Fonction{tail}] extraction\index{extraction!derni�res
      valeurs} des $n$ derni�res valeurs ou suppression des $n$
    premi�res (\textsf{R} seulement)
  \item[\Fonction{unique}] extraction des �l�ments
    diff�rents\index{extraction!elements diff�rents@�l�ments
      diff�rents} d'un vecteur
  \end{ttscript}
\end{frame}

\subsection{Recherche d'�l�ments dans un vecteur}

\begin{frame}
  \frametitle{Recherche d'�l�ments dans un vecteur}
  \begin{ttscript}{which.max}
  \item[\Fonction{which}] positions des valeurs \texttt{TRUE} dans un
    vecteur bool�en
  \item[\Fonction{which.min}] position du
    minimum\index{minimum!position dans un vecteur} dans un vecteur
  \item[\Fonction{which.max}] position du
    maximum\index{maximum!position dans un vecteur} dans un vecteur
  \item[\Fonction{match}] position de la premi�re occurrence d'un
    �l�ment dans un vecteur
  \item[\Fonction{\%in\%}] appartenance d'une ou plusieurs valeurs �
    un vecteur
  \end{ttscript}
\end{frame}

\subsection{Arrondi}

\begin{frame}
  \frametitle{Arrondi}
  \begin{ttscript}{ceiling}
  \item[\Fonction{round}] arrondi\index{arrondi} � un nombre d�fini de
    d�cimales
  \item[\Fonction{floor}] plus grand entier inf�rieur ou �gal �
    l'argument
  \item[\Fonction{ceiling}] plus petit entier sup�rieur ou �gal �
    l'argument
  \item[\Fonction{trunc}] troncature vers z�ro de l'argument;
    diff�rent de \texttt{floor} pour les nombres n�gatifs
  \end{ttscript}
\end{frame}

\subsection{Sommaires et statistiques descriptives}

\begin{frame}
  \frametitle{Sommaires et statistiques descriptives}
  \begin{ttscript}{sum, prod}
  \item[\Fonction{sum}, \Fonction{prod}] somme\index{somme} et
    produit\index{produit} des �l�ments d'un vecteur
  \item[\Fonction{diff}] diff�rences\index{diff�rences} entre les
    �l�ments d'un vecteur
  \item[\Fonction{mean}] moyenne
    arithm�tique\index{moyenne!arithm�tique} et moyenne
    tronqu�e\index{moyenne!tronqu�e}
  \item[\Fonction{var}, \Fonction{sd}] variance\index{variance} et
    �cart type\index{ecart type@�cart type} (versions sans biais)
  \item[\Fonction{min}, \Fonction{max}] minimum\index{minimum!d'un
      vecteur} et maximum\index{maximum!d'un vecteur} d'un vecteur
  \item[\Fonction{range}] vecteur contenant le minimum et le maximum
    d'un vecteur
  \item[\Fonction{median}] m�diane\index{mediane@m�diane} empirique
  \item[\Fonction{quantile}] quantiles\index{quantile} empiriques
  \item[\Fonction{summary}] statistiques descriptives d'un �chantillon
  \end{ttscript}
\end{frame}

\subsection{Sommaires cumulatifs et comparaisons �l�ment par �l�ment}

\begin{frame}
  \frametitle{Sommaires cumulatifs et comparaisons �l�ment par
    �l�ment}
  \begin{ttscript}{cumsum, cumprod}
  \item[\Fonction{cumsum}, \Fonction{cumprod}]
    somme\index{somme!cumulative} et produit\index{produit!cumulatif}
    cumulatif d'un vecteur
  \item[\Fonction{cummin}, \Fonction{cummax}]
    minimum\index{minimum!cumulatif} et
    maximum\index{maximum!cumulatif} cumulatif
  \item[\Fonction{pmin}, \Fonction{pmax}]
    minimum\index{minimum!parall�le} et
    maximum\index{maximum!parall�le} en parall�le, c'est-�-dire
    �l�ment par �l�ment entre deux vecteurs ou plus
  \end{ttscript}
\end{frame}

\subsection{Op�rations sur les matrices}

\begin{frame}
  \frametitle{Op�rations sur les matrices}
  \begin{ttscript}{nrow, ncol}
  \item[\Fonction{t}] transpos�e\index{matrice!transpos�e}
  \item[\Fonction{solve}] avec un seul argument (une matrice carr�e):
    inverse\index{matrice!inverse} d'une matrice; avec deux arguments
    (une matrice carr�e et un vecteur): solution du syst�me d'�quation
    $\mat{A} \mat{x} = \mat{b}$
  \item[\Fonction{diag}] avec une matrice en argument: diagonale de la
    matrice; avec un vecteur en argument: matrice
    diagonale\index{matrice!diagonale} form�e avec le vecteur; avec un
    scalaire $p$ en argument: matrice identit�\index{matrice!identit�}
    $p \times p$
  \item[\Fonction{nrow}, \Fonction{ncol}] nombre de lignes et de
    colonnes d'une matrice
  \end{ttscript}
\end{frame}

\begin{frame}
  \frametitle{Op�rations sur les matrices (suite)}
  \begin{ttscript}{rowMeans, colMeans}
  \item[\Fonction{rowSums}, \Fonction{colSums}]
    sommes\index{matrice!sommes par ligne} par ligne et par
    colonne\index{matrice!somme par colonne}, respectivement, des
    �l�ments d'une matrice; voir aussi la fonction \fonction{apply}
  \item[\Fonction{rowMeans}, \Fonction{colMeans}]
    moyennes\index{matrice!moyennes par ligne} par ligne et par
    colonne\index{matrice!moyennes par colonne}, respectivement, des
    �l�ments d'une matrice; voir aussi la fonction \fonction{apply}
  \item[\Fonction{rowVars}, \Fonction{colVars}]
    variance\index{matrice!variance par ligne} par ligne et par
    colonne\index{matrice!variance par colonne} des �l�ments d'une
    matrice (S-Plus seulement)
  \end{ttscript}
\end{frame}

\subsection{Produit ext�rieur}

\begin{frame}
  \frametitle{Produit ext�rieur}
  
  La fonction \Fonction{outer}, dont la syntaxe est
  \begin{center}
    \code{outer(X, Y, FUN)},
  \end{center}
  applique la fonction \code{FUN} (\fonction{prod} par d�faut) entre
  chacun des �l�ments de \code{X} et chacun des �l�ments de \code{Y}.

  \begin{itemize}
  \item La dimension du r�sultat est par cons�quent \code{c(dim(X),
      dim(Y))}.
  \end{itemize}
\end{frame}

\begin{frame}[fragile]
  \begin{itemize}[<+->]
  \item Par exemple: le produit ext�rieur entre deux vecteurs est une
    matrice contenant tous les produits entre les �l�ments des deux
    vecteurs:
\begin{Schunk}
\begin{Sinput}
> outer(c(1, 2, 5), c(2, 3, 6))
\end{Sinput}
\begin{Soutput}
     [,1] [,2] [,3]
[1,]    2    3    6
[2,]    4    6   12
[3,]   10   15   30
\end{Soutput}
\end{Schunk}
  \item L'op�rateur \Fonction{\%o\%} est un raccourci de \code{outer(X,
      Y, prod)}.
  \end{itemize}
\end{frame}


\subsection{Structures de contr�le}

\subsubsection{Ex�cution conditionnelle}

\begin{frame}
  \frametitle{Ex�cution conditionnelle}
  
  \begin{block}{\code{if (\emph{condition}) \emph{branche.vrai} else
      \emph{branche.faux}}}
    Si \code{\emph{condition}} est vraie, \code{\emph{branche.vrai}}
    est ex�cut�e, et \code{\emph{branche.faux}} sinon. 
    \medskip

    Si l'une ou l'autre de \code{\emph{branche.vrai}} ou
    \code{\emph{branche.faux}} comporte plus d'une expression, les
    grouper dans des accolades $\{~\}$.
  \end{block}

  \begin{block}<2>{\code{ifelse(\emph{condition}, \emph{expression.vrai},
      \emph{expression.faux})}}
    Fonction vectoris�e qui remplace chaque �l�ment \code{TRUE} du
    vecteur \code{\emph{condition}} par l'�l�ment correspondant de
    \code{\emph{expression.vrai}} et chaque �l�ment \code{FALSE} par
    l'�l�ment correspondant de \code{\emph{expression.faux}}.
  \end{block}
\end{frame}

\subsubsection{Boucles}

\begin{frame}
  \frametitle{Boucles}

  \begin{itemize}[<+->]
  \item Les boucles sont et \alert{doivent} �tre utilis�es avec
    parcimonie en S car elles sont g�n�ralement inefficaces
    (particuli�rement avec S-Plus).
  \item Dans la majeure partie des cas, il est possible de vectoriser
    les calcul pour �viter les boucles explicites.
  \item Sinon, s'en remettre aux fonctions \fonction{apply},
    \fonction{lapply} et \fonction{sapply} pour faire les boucles de
    mani�re plus efficace.
  \end{itemize}
\end{frame}

\begin{frame}
  \frametitle{Boucles de longueur d�termin�e}

  \begin{block}{\code{for (\emph{variable} in \emph{suite}) \emph{expression}}}
    Ex�cuter \code{\emph{expression}} successivement pour chaque
    valeur de \code{\emph{variable}} contenue dans
    \code{\emph{suite}}.
    \medskip

    Encore ici, on groupera les expressions dans des accolades
    $\{~\}$.
    \medskip

    � noter que \code{\emph{suite}} n'a pas � �tre compos�e de nombres
    cons�cutifs, ni m�me par ailleurs de nombres.
  \end{block}
\end{frame}

\begin{frame}
  \frametitle{Boucles de longueur ind�termin�e}

  \begin{block}{\code{while (\emph{condition}) \emph{expression}}}
    Ex�cuter \code{\emph{expression}} tant que \code{\emph{condition}}
    est vraie.
    \medskip

    Si \code{\emph{condition}} est fausse lors de l'entr�e dans la
    boucle, celle-ci n'est pas ex�cut�e.
    \medskip

    Une boucle \code{while} n'est par cons�quent pas n�cessairement
    toujours ex�cut�e.
  \end{block}

  \begin{block}{\code{repeat \emph{expression}}}
    R�p�ter \code{\emph{expression}}. Cette derni�re devra comporter un
    test d'arr�t qui utilisera la commande \code{break}.
    \medskip
    
    Une boucle \code{repeat} est toujours ex�cut�e au moins une fois.
  \end{block}
\end{frame}

\begin{frame}
  \frametitle{Modification du d�roulement d'une boucle}

  \begin{block}{\code{break}}
    Sortie imm�diate d'une boucle \code{for}, \code{while} ou
    \code{repeat}.
  \end{block}
  \begin{block}{\code{next}}
    Passage imm�diat � la prochaine it�ration d'une boucle \code{for},
    \code{while} ou \code{repeat}.
  \end{block}
\end{frame}

%%% Local Variables: 
%%% mode: latex
%%% TeX-master: "introduction_programmation_S_slides"
%%% End: 

%\section{Exemples r�solus}

\subsection{Calcul de valeurs pr�sentes}

\begin{frame}
  \frametitle{Calcul de valeurs pr�sentes}

  \begin{block}{�nonc�}
    Un pr�t est rembours� par une s�rie de cinq paiements, le premier
    dans un an. Trouver le montant du pr�t pour chacune des hypoth�ses
    ci-dessous.
    \begin{enumerate}[(a)]
    \item Paiement annuel de \nombre{1000}, taux d'int�r�t de 6~\%
      effectif annuellement.
    \item Paiements annuels de 500, 800, 900, 750 et \nombre{1000},
      taux d'int�r�t de 6~\% effectif annuellement.
    \item Paiements annuels de 500, 800, 900, 750 et \nombre{1000},
      taux d'int�r�t de 5~\%, 6~\%, 5,5~\%, 6,5~\% et 7~\% effectifs
      annuellement.
    \end{enumerate}
  \end{block}
\end{frame}

\begin{frame}
  \frametitle{Solution}
  De mani�re g�n�rale, la valeur pr�sente d'une s�rie de paiements
  $P_1, P_2, \dots, P_n$ � la fin des ann�es $1, 2, \dots, n$ est
  \begin{displaymath}
    \sum_{j=1}^n \prod_{k=1}^j (1 + i_k)^{-1} P_j,
  \end{displaymath}
\end{frame}

\begin{frame}[fragile]
  \NoAutoSpaceBeforeFDP
  \begin{proof}[(a) Un seul paiement annuel, un seul taux d'int�r�t]
    Cas sp�cial
    \begin{displaymath}
      \color{orange}P \color{blue} \sum_{j=1}^n \color{red}(1 + i)^{-j}
    \end{displaymath}
    En S:
    \pause
\begin{semiverbatim} \slshape
> \uncover<4->{\color{orange}1000 * }\uncover<3->{\color{blue}sum(}{\color{red}(1 + 0.06)^(-(1:5))}\uncover<3->{)}
\end{semiverbatim}
    \onslide<5->
\begin{Soutput}
[1] 4212.364
\end{Soutput}
  \end{proof}
\end{frame}

\begin{frame}[fragile]
  \NoAutoSpaceBeforeFDP
  \begin{proof}[(b) Diff�rents paiements annuels, un seul taux d'int�r�t]
    On a, cette fois,
    \begin{displaymath}
      \color{blue} \sum_{j=1}^n \color{red}(1 + i)^{-j} \color{orange}P_j 
    \end{displaymath}
    En S:
    \pause
\begin{semiverbatim}
> \uncover<3->{\color{blue}sum(}\color{orange}c(500, 800, 900, 750, 1000) *
\normalcolor+     \color{red}(1 + 0.06)^(-(1:5))\uncover<3->{\color{blue})}
\end{semiverbatim}
    \onslide<4->
\begin{Soutput}
[1] 3280.681
\end{Soutput}
  \end{proof}
\end{frame}

\begin{frame}[fragile]
  \begin{proof}[(c) Diff�rents paiements annuels, diff�rents taux d'int�r�t]
    On doit utiliser la formule g�n�rale
    \begin{displaymath}
      \color{blue}\sum_{j=1}^n 
      \color{red} \prod_{k=1}^j (1 + i_k)^{-1} 
      \color{orange} P_j
    \end{displaymath}
    En S:
    \pause
\begin{semiverbatim}
> \uncover<4->{\color{blue}sum(}\uncover<3->{\color{orange}c(500, 800, 900, 750, 1000)/}
\normalcolor+ \color{red}cumprod(c(1.05, 1.06, 1.055, 1.065, 1.07))\uncover<4->{\color{blue})}
\end{semiverbatim}
    \onslide<5->
\begin{Soutput}
[1] 3308.521
\end{Soutput}
  \end{proof}
\end{frame}

\subsection{Fonctions de probabilit�}
\label{exemples:fp}

\begin{frame}
  \frametitle{Fonctions de probabilit�}

  \begin{block}{�nonc�}
    Calculer toutes ou la majeure partie des probabilit�s des deux
    lois de probabilit� ci-dessous. V�rifier que la somme des
    probabilit�s est bien �gale � 1.

    \begin{enumerate}[(a)]
    \item Binomiale
      \begin{displaymath}
        f(x) = \binom{n}{x} p^x (1 - p)^{n - x}, \quad x = 0, \dots, n.
      \end{displaymath}
    \item Poisson
      \begin{displaymath}
        f(x) = \frac{\lambda^x e^{-\lambda}}{x!}, \quad x = 0, 1, \dots,
      \end{displaymath}
      o� $x! = x(x - 1) \cdots 2 \cdot 1$.
    \end{enumerate}
  \end{block}
\end{frame}

\begin{frame}[fragile]
  \frametitle{Solution}

  \begin{proof}[(a) Binomiale$(10,\, 0,8)$]
\begin{Schunk}
\begin{Sinput}
> n <- 10
> p <- 0.8
> x <- 0:n
> choose(n, x) * p^x * (1 - p)^rev(x)
\end{Sinput}
\begin{Soutput}
 [1] 0.0000001024 0.0000040960 0.0000737280
 [4] 0.0007864320 0.0055050240 0.0264241152
 [7] 0.0880803840 0.2013265920 0.3019898880
[10] 0.2684354560 0.1073741824
\end{Soutput}
\end{Schunk}
    \pause
\begin{Schunk}
\begin{Sinput}
> sum(choose(n, x) * p^x * (1 - p)^rev(x))
\end{Sinput}
\begin{Soutput}
[1] 1
\end{Soutput}
\end{Schunk}
  \end{proof}
\end{frame}

\begin{frame}[fragile]
  \begin{proof}[(b) Poisson$(5)$]
    On calcule les probabilit�s en $x = 0, 1, \dots, 10$ seulement.
\begin{Schunk}
\begin{Sinput}
> lambda <- 5
> x <- 0:10
> exp(-lambda) * (lambda^x/factorial(x))
\end{Sinput}
\begin{Soutput}
 [1] 0.006737947 0.033689735 0.084224337
 [4] 0.140373896 0.175467370 0.175467370
 [7] 0.146222808 0.104444863 0.065278039
[10] 0.036265577 0.018132789
\end{Soutput}
\end{Schunk}
  \pause
\begin{Schunk}
\begin{Sinput}
> x <- 0:200
> exp(-lambda) * sum((lambda^x/factorial(x)))
\end{Sinput}
\begin{Soutput}
[1] 1
\end{Soutput}
\end{Schunk}
  \end{proof}
\end{frame}


\subsection{Fonction de r�partition de la loi gamma}

\begin{frame}
  \frametitle{Fonction de r�partition de la loi gamma}

  La loi gamma est fr�quemment utilis�e pour la mod�lisation
  d'�v�nements ne pouvant prendre que des valeurs positives et pour
  lesquels les petites valeurs sont plus fr�quentes que les grandes.

  Nous utiliserons la param�trisation o� la fonction de densit� de
  probabilit� est
  \begin{displaymath}
    f(x) = \frac{\lambda^\alpha}{\Gamma(\alpha)}\, x^{\alpha-1}
    e^{-\lambda x}, \quad x > 0,
  \end{displaymath}
  o�
  \begin{displaymath}
    \Gamma(n) = \int_0^\infty x^{n - 1} e^{-x}\, dx
    = (n - 1) \Gamma(n - 1).
  \end{displaymath}
\end{frame}

\begin{frame}
  \begin{block}{�nonc�}
    Il n'existe pas de formule explicite de la fonction de r�partition
    de la loi gamma. 

    N�anmoins, pour $\alpha$ entier et $\lambda = 1$ on a
    \begin{displaymath}
      F(x; \alpha, 1) = 1 - e^{-x} \sum_{j=0}^{\alpha-1} \frac{x^j}{j!}.
    \end{displaymath}

    \begin{enumerate}[(a)]
    \item �valuer $F(4; 5, 1)$.
    \item �valuer $F(x; 5, 1)$ pour $x = 2, 3, \dots, 10$ en une seule
      expression.
    \end{enumerate}
  \end{block}
\end{frame}

\begin{frame}[fragile]
  \frametitle{Solution}

  \begin{proof}[(a) Une seule valeur de $x$, param�tre $\alpha$ fixe]
\begin{Schunk}
\begin{Sinput}
> alpha <- 5
> x <- 4
> 1 - exp(-x) * sum(x^(0:(alpha - 1))/
+ gamma(1:alpha))
\end{Sinput}
\begin{Soutput}
[1] 0.3711631
\end{Soutput}
\end{Schunk}
  \pause
  V�rification avec la fonction interne \fonction{pgamma}:
\begin{Schunk}
\begin{Sinput}
> pgamma(x, alpha)
\end{Sinput}
\begin{Soutput}
[1] 0.3711631
\end{Soutput}
\end{Schunk}
  \end{proof}
\end{frame}

\begin{frame}[fragile]
  \begin{alertblock}{Astuce}
    On peut aussi �viter de g�n�rer essentiellement la m�me suite de
    nombres � deux reprises en ayant recours � une variable
    interm�diaire. 

    L'affectation et le calcul final peuvent se faire dans une
    seule expression.
\begin{Schunk}
\begin{Sinput}
> 1 - exp(-x) * sum(x^(-1 + (j <- 1:alpha))/
+ gamma(j))
\end{Sinput}
\begin{Soutput}
[1] 0.3711631
\end{Soutput}
\end{Schunk}
  \end{alertblock}
\end{frame}

\begin{frame}[fragile]
  \begin{proof}[(b) Plusieurs valeurs de $x$, param�tre $\alpha$ fixe]
    C'est un travail pour la fonction \fonction{outer}.
\begin{Schunk}
\begin{Sinput}
> x <- 2:10
> 1 - exp(-x) * 
+ colSums(
+   t( outer(x, 0:(alpha - 1), "^") )
+   /gamma(1:alpha)
+ )
\end{Sinput}
\begin{Soutput}
[1] 0.05265302 0.18473676 0.37116306
[4] 0.55950671 0.71494350 0.82700839
[7] 0.90036760 0.94503636 0.97074731
\end{Soutput}
\end{Schunk}
  \end{proof}
\end{frame}


\subsection{Algorithme du point fixe}

\begin{frame}
  \frametitle{Algorithme du point fixe}

  \begin{itemize}[<+->]
  \item Probl�me classique: trouver la racine d'une fonction $g$,
    c'est-�-dire le point $x$ o� $g(x) = 0$.
  \item Souvent possible de reformuler le probl�me de fa�on � plut�t
    chercher le point $x$ o� $f(x) = x$.
  \item Solution appel�e \alert{point fixe}.
  \item L'algorithme du calcul num�rique du point fixe d'une fonction
    $f(x)$ est tr�s simple:
    \begin{enumerate}
    \item choisir une valeur de d�part $x_0$;
    \item calculer $x_n = f(x_{n-1})$;
    \item r�p�ter l'�tape 2 jusqu'� ce que $|x_n - x_{n-1}| < \varepsilon$
      ou $|x_n - x_{n-1}|/|{x_{n-1}}| < \varepsilon$.
    \end{enumerate}
  \end{itemize}
\end{frame}

\begin{frame}
  \begin{block}{�nonc�}
    Trouver, � l'aide de la m�thode du point fixe, la valeur de $i$
    telle que
    \begin{displaymath}
      a_{\angl{10}} = \frac{1 - (1 + i)^{-10}}{i} = 8,21.
    \end{displaymath}
  \end{block}
\end{frame}

\begin{frame}
  \frametitle{Solution}

  \begin{block}{Quelques consid�rations.}
    \begin{itemize}[<+->]
    \item On doit r�soudre
      \begin{displaymath}
        \frac{1 - (1 + i)^{-10}}{8,21} = i.
      \end{displaymath}
    \item Nous ignorons combien de fois la proc�dure it�rative devra
      �tre r�p�t�e.
    \item Il faut ex�cuter la proc�dure au moins une fois.
    \item La structure de contr�le � utiliser dans cette proc�dure
      it�rative est donc \fonction{\alert{repeat}}.
    \end{itemize}
  \end{block}
\end{frame}

\begin{frame}[fragile]
  \begin{proof}[Le code]
\begin{Schunk}
\begin{Sinput}
> i <- 0.05
> repeat {
+     it <- i
+     i <- (1 - (1 + it)^(-10))/8.21
+     if (abs(i - it)/it < 1e-10) 
+         break
+ }
> i
\end{Sinput}
\begin{Soutput}
[1] 0.03756777
\end{Soutput}
\end{Schunk}
  \pause
\begin{Schunk}
\begin{Sinput}
> (1 - (1 + i)^(-10))/i
\end{Sinput}
\begin{Soutput}
[1] 8.21
\end{Soutput}
\end{Schunk}
  \end{proof}
\end{frame}
%%% Local Variables: 
%%% mode: latex
%%% TeX-master: "introduction_programmation_S_slides"
%%% End: 

%\section{Fonctions d�finies par l'usager}

\subsection{D�finition d'une fonction}

\begin{frame}
  \frametitle{D�finition d'une fonction}

  On d�finit une fonction de la mani�re suivante:
  \begin{center}
    \code{fun <- function(\emph{arguments}) \emph{expression}}
  \end{center}
  o�
  \begin{itemize}
  \item \code{fun} est le nom de la fonction;
  \item \code{\itshape arguments} est la liste des arguments, s�par�s
    par des virgules;
  \item \code{\itshape expression} constitue le corps de la fonction,
    soit une liste d'expressions group�es entre accolades (n�cessaires
    s'il y a plus d'une expression seulement).
  \end{itemize}
\end{frame}

\subsection{Retourner des r�sultats}

\begin{frame}
  \frametitle{Retourner des r�sultats}

  \begin{itemize}[<+->]
  \item Une fonction retourne tout simplement le r�sultat de la
    \alert{derni�re expression} du corps de la fonction.
  \item �viter que la derni�re expression soit une affectation: la
    fonction ne retournera rien!
  \item Autre possibilit�: utiliser explicitement la fonction
    \Fonction{\alert{return}}. Rarement n�cessaire.
  \item Utiliser une liste nomm�e pour retourner plusieurs r�sultats.
  \end{itemize}
\end{frame}

\subsection{Variables locales et globales}

\begin{frame}[fragile]
  \frametitle{Variables locales et globales}

  Les concepts de variable locale et de variable globale existent
  aussi en S.

  \begin{itemize}
  \item<+-> Toute variable d�finie dans une fonction est locale � cette
    fonction, c'est-�-dire
    \begin{itemize}
    \item<+-> qu'elle n'appara�t pas dans l'espace de travail;
    \item<+-> qu'elle n'�crase pas une variable du m�me nom dans l'espace
      de travail.
    \end{itemize}
  \item<+| only@-+> On peut d�finir une variable dans l'espace de
    travail depuis une fonction avec l'op�rateur $<<-$.
  \item<only@+-> \sout{On peut d�finir une variable dans l'espace de
      travail depuis une fonction avec l'op�rateur $<<-$.}
  \item<+-> Une fonction d�finie � l'int�rieur d'une autre fonction sera
    locale � celle-ci. Pratique!
  \end{itemize}
\end{frame}


\subsection{Exemple de fonction}

\begin{frame}[fragile]
  \frametitle{Fonction � partir du code de point fixe}
  
\begin{semiverbatim}
\alert<2>{fp} <- function(\alert<3>{k}, \alert<3>{n}, \alert<3>{start=\alert<4>{0.05}}, \alert<3>{TOL=\alert<4>{1E-10}})
\{
    \alert<5>{i <- start
    repeat
    \{
        it <- i
        i <- (1 - (1 + it)^(-n))/k
        if (abs(i - it)/it < TOL)
            break
    \}
    \alert<6>{i}  # ou return(i)}
\}
\end{semiverbatim}
\end{frame}


\subsection{Fonctions anonymes}


\begin{frame}
  \frametitle{Fonctions anonymes}

  \begin{itemize}
  \item Parfois utile de d�finir une fonction sans lui attribuer un
    nom
  \item C'est une \alert{fonction anonyme}.
  \item En g�n�ral pour des fonctions courtes utilis�es dans une autre
    fonction.
  \end{itemize}
\end{frame}

\begin{frame}[fragile]
  \frametitle{Un exemple}
  
  \begin{itemize}[<+->]
  \item Calculer la valeur de $x y^2$ pour toutes les combinaisons de
    $x$ et $y$ stock�es dans des vecteurs du m�me nom
  \item Avec \fonction{outer}:
  \begin{Schunk}
\begin{Sinput}
> x <- 1:3
> y <- 4:6
> f <- function(x, y) x * y^2
> outer(x, y, f)
\end{Sinput}
\begin{Soutput}
     [,1] [,2] [,3]
[1,]   16   25   36
[2,]   32   50   72
[3,]   48   75  108
\end{Soutput}
\end{Schunk}
  \end{itemize}
\end{frame}

\begin{frame}[fragile]
  \begin{itemize}[<+->]
  \item La fonction \code{f} ne sert � rien ult�rieurement.
  \item Utiliser simplement une fonction anonyme � l'int�rieur de
    \fonction{outer}:
\begin{Schunk}
\begin{Sinput}
> outer(x, y, function(x, y) x * y^2)
\end{Sinput}
\begin{Soutput}
     [,1] [,2] [,3]
[1,]   16   25   36
[2,]   32   50   72
[3,]   48   75  108
\end{Soutput}
\end{Schunk}
  \end{itemize}
\end{frame}


\subsection{D�bogage de fonctions}

\begin{frame}
  \frametitle{Techniques les plus simples et na�ves}

  \begin{itemize}[<+->]
  \item Simples erreurs de syntaxe sont les plus fr�quentes (en
    particulier l'oubli de virgules).
  \item V�rification de la syntaxe lors de la d�finition d'une
    fonction.
  \item Lorsqu'une fonction ne retourne pas le r�sultat attendu,
    placer des commandes \fonction{\alert{print}} � l'int�rieur de la
    fonction.
  \item Permet de d�terminer les valeurs des variables dans le
    d�roulement de la fonction.
  \end{itemize}
\end{frame}

\begin{frame}[fragile=singleslide]
  \begin{exampleblock}{Exemple}
    Modification de la boucle du point fixe pour d�tecter une
    proc�dure divergente.
\begin{verbatim}
repeat
{
    it <- i
    i <- (1 - (1 + it)^(-n))/k
    print(i)
    if (abs((i - it)/it < TOL))
        break
}
\end{verbatim}
  \end{exampleblock}
\end{frame}

\begin{frame}
  \frametitle{Avec Emacs et le mode ESS}

  \begin{itemize}[<+->]
  \item S'assurer que toutes les variables pass�es en arguments � une
    fonction existent dans l'espace de travail.
  \item Ex�cuter successivement les lignes de la fonction avec
    \ess{C-c C-n}.
  \item Impossible avec les interfaces graphiques car la fen�tre
    d'�dition de fonctions bloque l'acc�s � l'interface de commande.
  \end{itemize}
\end{frame}


\subsection{Styles de codage}

\begin{frame}[fragile=singleslide]
  \frametitle{Styles reconnus par Emacs}
  
  \small
  \begin{center}
    \begin{minipage}[t]{9cm}
      C++/Stroustrup
      \hfill
      \begin{minipage}[t]{5cm}
\begin{verbatim}
for (i in 1:10)
{
    expression
}
\end{verbatim}
      \end{minipage}
    \end{minipage}
    \vspace{\baselineskip}
    
    \begin{minipage}[t]{9cm}
      K\&R (1TBS)
      \hfill
      \begin{minipage}[t]{5cm}
\begin{verbatim}
for (i in 1:10){
     expression
}
\end{verbatim}
      \end{minipage}
    \end{minipage}
    \vspace{\baselineskip}
    
    \begin{minipage}[t]{9cm}
      Whitesmith
      \hfill
      \begin{minipage}[t]{5cm}
\begin{verbatim}
for (i in 1:10)
     {
     expression
     }
\end{verbatim}
      \end{minipage}
    \end{minipage}
    \vspace{\baselineskip}
    
    \begin{minipage}[t]{9cm}
      GNU
      \hfill
      \begin{minipage}[t]{5cm}
\begin{verbatim}
for (i in 1:10)
  {
    expression
  }
\end{verbatim}
      \end{minipage}
    \end{minipage}
  \end{center}
\end{frame}

\begin{frame}[fragile]
  \frametitle{Standard pour la programmation en S}

  \begin{itemize}
  \item Style C$++$, avec les accolades sur leurs propres lignes.
  \item Une indentation de quatre (4) espaces.
  \item<2-> Pour utiliser ce style dans Emacs\index{Emacs}, faire
\begin{verbatim}
    M-x ess-set-style RET C++ RET
\end{verbatim}
    une fois qu'un fichier de script est ouvert.
  \end{itemize}
\end{frame}

%%% Local Variables: 
%%% mode: latex
%%% TeX-master: "introduction_programmation_S_slides"
%%% End: 

%\part{CONCEPTS AVANC�S}

\begin{frame}
  \partpage
\end{frame}

\begin{frame}
  \frametitle{Sommaire}
  \tableofcontents
\end{frame}

\section{L'argument `\code{...}'}

\begin{frame}
  \frametitle{Pas un signe de paresse des r�dacteurs}

  \begin{itemize}[<+->]
  \item `\code{...}' est un argument formel dont `\code{...}' est le
    nom.
  \item Signifie qu'une fonction peut accepter un ou plusieurs autres
    arguments autres que ceux faisant partie de sa d�finition.
  \item Contenu de `\code{...}' n'est ni pris en compte, ni modifi�
    par la fonction.
  \item G�n�ralement simplement pass� tel quel � une autre fonction.
  \item Voir les d�finitions des fonctions \fonction{apply},
    \fonction{lapply} et \fonction{sapply} pour des exemples.
  \end{itemize}
\end{frame}


\section{Fonction \code{apply}}

\begin{frame}
  \frametitle{Sommaires g�n�raux pour matrices et tableaux}

  La fonction \Fonction{apply} sert � appliquer une fonction
  quelconque sur une partie d'une matrice\index{matrice} ou, plus
  g�n�ralement, d'un tableau\index{tableau}.
  \pause
  \begin{center}
    \code{apply(X, MARGIN, FUN, ...)},
  \end{center}
  o�
  \begin{itemize}
  \item \code{X} est une matrice ou un tableau;
  \item \code{MARGIN} est un vecteur d'entiers contenant la ou les
    dimensions de la matrice ou du tableau sur lesquelles la fonction
    doit s'appliquer;
  \item \code{FUN} est la fonction � appliquer;
  \item `\code{...}' est un ensemble d'arguments suppl�mentaires,
    s�par�s par des virgules, � passer � la fonction \code{FUN}.
  \end{itemize}
\end{frame}

\begin{frame}
  \begin{itemize}[<+->]
  \item Principalement pour calculer des sommaires par ligne
    (dimension 1) ou par colonne (dimension 2) autres que la somme et
    la moyenne.
  \item Utiliser la fonction \fonction{apply} plut�t que des boucles
    puisque celle-ci est plus efficace.
  \end{itemize}
\end{frame}

\begin{frame}[fragile]
  \frametitle{Exemples avec une matrice}
\begin{Schunk}
\begin{Sinput}
> m
\end{Sinput}
\begin{Soutput}
     [,1] [,2] [,3] [,4]
[1,]   54   33   30   17
[2,]    3   46   95   83
[3,]   47    6   56   58
[4,]   18   22   50   36
[5,]   41   41   77   31
\end{Soutput}
  \pause
\begin{Sinput}
> apply(m, 1, var)
\end{Sinput}
\begin{Soutput}
[1]  235.0000 1718.9167  590.9167  211.6667
[5]  409.0000
\end{Soutput}
  \pause
\begin{Sinput}
> apply(m, 2, min)
\end{Sinput}
\begin{Soutput}
[1]  3  6 30 17
\end{Soutput}
  \pause
\begin{Sinput}
> apply(m, 1, mean, trim = 0.2)
\end{Sinput}
\begin{Soutput}
[1] 33.50 56.75 41.75 31.50 47.50
\end{Soutput}
\end{Schunk}
\end{frame}

\begin{frame}[fragile]
  \frametitle{Exemple avec un tableau}

  Si \code{X} est un tableau de plus de deux dimensions, alors
  l'argument pass� � \code{FUN} peut �tre une matrice ou un tableau.

\begin{Schunk}
\begin{Sinput}
> dim(arr)
\end{Sinput}
\begin{Soutput}
[1] 4 4 5
\end{Soutput}
\begin{Sinput}
> apply(arr, 3, det)
\end{Sinput}
\begin{Soutput}
[1]  1178800 16153716 14298240 20093933
[5]  6934743
\end{Soutput}
\end{Schunk}
\end{frame}


\section{Fonctions \code{lapply} et \code{sapply}}

\begin{frame}
  \frametitle{Les \code{apply} des vecteurs et des listes}

  \begin{itemize}
  \item Les fonctions \Fonction{lapply} et \Fonction{sapply}
    permettent d'appliquer une fonction aux �l�ments d'un vecteur ou
    d'une liste.
  \item  Syntaxe similaire:
    \begin{center}
      \code{lapply(X, FUN, ...)} \\
      \code{sapply(X, FUN, ...)}
    \end{center}
  \end{itemize}
\end{frame}

\begin{frame}
  \frametitle{Des fonctions tr�s utiles}

  \begin{itemize}[<+->]
  \item \fonction{lapply} applique une fonction \code{FUN} � tous les
    �l�ments d'un vecteur ou d'une liste \code{X} et retourne le
    r�sultat sous forme de liste.
  \item \fonction{sapply} est similaire, sauf que le r�sultat est
    retourn� sous forme de vecteur, si possible.
  \item Si le r�sultat de chaque application de la fonction est un
    vecteur, \fonction{sapply} retourne une matrice, remplie comme
    toujours par colonne.
  \item Dans un grand nombre de cas, il est possible de remplacer les
    boucles \fonction{for} par l'utilisation de \fonction{lapply} ou
    \fonction{sapply}.
  \end{itemize}
\end{frame}


\section{Fonction \code{mapply}}

\begin{frame}
  \frametitle{Version multidimensionnelle de \code{sapply}}

  \begin{itemize}[<+->]
  \item Syntaxe:
    \begin{center}
      \code{mapply(FUN, ...)}
    \end{center}
  \item Le r�sultat est l'application de \code{FUN} aux premiers
    �l�ments de tous les arguments contenus dans `\code{...}', puis �
    tous les seconds �l�ments, et ainsi de suite.
  \item Ainsi, si \code{v} et \code{w} sont des vecteurs,
    \begin{center}
      \code{mapply(FUN, v, w)}
    \end{center}
    retourne \code{FUN(v[1], w[1])}, \code{FUN(v[2], w[2])}, etc.
  \end{itemize}
\end{frame}

\begin{frame}[fragile]
  \frametitle{Exemple}

\begin{Schunk}
\begin{Sinput}
> mapply(rep, 1:4, 4:1)
\end{Sinput}
  \pause
\begin{Soutput}
[[1]]
[1] 1 1 1 1

[[2]]
[1] 2 2 2

[[3]]
[1] 3 3

[[4]]
[1] 4
\end{Soutput}
\end{Schunk}
\end{frame}

\begin{frame}[fragile=singleslide]
  \frametitle{Les �l�ments de `\code{...}' sont recycl�s au besoin}
\begin{Schunk}
\begin{Sinput}
> mapply(seq, 1:6, 6:8)
\end{Sinput}
\begin{Soutput}
[[1]]
[1] 1 2 3 4 5 6

[[2]]
[1] 2 3 4 5 6 7

[[3]]
[1] 3 4 5 6 7 8

[[4]]
[1] 4 5 6

[[5]]
[1] 5 6 7

[[6]]
[1] 6 7 8
\end{Soutput}
\end{Schunk}
\end{frame}


\section{Fonction \code{replicate}}

\begin{frame}[fragile]
  \frametitle{Une fonction pour la simulation}

  \begin{itemize}[<+->]
  \item Fonction enveloppante de \fonction{sapply} propre �
    \textsf{R}.
  \item Simplifie la syntaxe pour l'ex�cution r�p�t�e d'une
    expression.
  \item Usage particuli�rement indiqu� pour les simulations.
  \item Si la fonction \code{fun} fait tous les calculs d'une
    simulation, on obtient les r�sultats pour \nombre{10000}
    simulations avec
\begin{Schunk}
\begin{Sinput}
> replicate(10000, fun(...))
\end{Sinput}
\end{Schunk}
  \item Voir l'annexe D du document d'accompagnement.
  \end{itemize}
\end{frame}


\section{Classes et fonctions g�n�riques}

\begin{frame}
  \frametitle{Quelques notions de programmation OO}
  
  \begin{itemize}[<+->]
  \item Tous les objets dans le langage S ont une classe.
  \item La classe est parfois implicite ou d�riv�e du mode de l'objet
    (consulter la rubrique d'aide de \fonction{class} pour de plus
    amples d�tails).
  \item Certaines fonctions \alert{g�n�riques} se comportent
    diff�remment selon la classe de l'objet donn� en argument.
  \item Les fonctions g�n�riques les plus fr�quemment employ�es
    sont \fonction{print}, \fonction{plot} et \fonction{summary}.
  \item Une fonction g�n�rique poss�de une \alert{m�thode}
    correspondant � chaque classe qu'elle reconna�t.
  \item Sinon, une m�thode \code{default} pour les autres objets. 
  \end{itemize}
\end{frame}

\begin{frame}[fragile=singleslide]
  \begin{itemize}
  \item La liste des m�thodes existant pour une fonction g�n�rique
    s'obtient avec \Fonction{methods}:
\begin{Schunk}
\begin{Sinput}
> methods(plot)
\end{Sinput}
\begin{Soutput}
 [1] plot.acf*       plot.data.frame*   
 [3] plot.Date*      plot.decomposed.ts*
 [5] plot.default    plot.dendrogram*   
 [7] plot.density    plot.ecdf          
 [9] plot.factor*    plot.formula*      
[11] plot.hclust*    plot.histogram*    

[...]

   Non-visible functions are asterisked
\end{Soutput}
\end{Schunk}
  \end{itemize}
\end{frame}

\begin{frame}
  \begin{itemize}
  \item � chaque m�thode \code{methode} d'une fonction g�n�rique
    \code{fun} correspond une fonction \code{fun.methode}.
  \item Consulter cette rubrique d'aide et non celle de la fonction
    g�n�rique, qui contient en g�n�ral peu d'informations.
  \end{itemize}
\end{frame}

\begin{frame}
  \begin{alertblock}{Astuce}
    Lorsque l'on tape le nom d'un objet � la ligne de commande
    pour voir son contenu, c'est la fonction g�n�rique
    \fonction{print} qui est appel�e. 

    On peut donc compl�tement modifier la repr�sentation � l'�cran du
    contenu d'un objet est cr�ant une nouvelle classe et une nouvelle
    m�thode pour la fonction \code{print}.
  \end{alertblock}
\end{frame}

%%% Local Variables: 
%%% mode: latex
%%% TeX-master: "introduction_programmation_S_slides"
%%% End: 


\lecture{Emacs et ESS}{lab 1}
%\section{GNU Emacs et ESS: la base}

\subsection{GNU Emacs}

\begin{frame}
  \frametitle{Qu'est-ce que Emacs?}
  \begin{itemize}
  \item Emacs est l'�diteur de texte des �diteurs de texte.
  \item D'abord et avant tout un �diteur pour programmeurs (avec des
    modes sp�ciaux pour une multitude de langages diff�rents).
  \item �galement un environnement id�al pour travailler sur des
    documents \LaTeX, interagir avec \textsf{R}, S-Plus, SAS ou SQL,
    ou m�me pour lire son courrier �lectronique.
  \end{itemize}
\end{frame}

\begin{frame}
  \frametitle{Mise en contexte}

  Emacs est le logiciel �tendard du projet GNU (�\emph{GNU is not
    Unix}�), dont le principal commanditaire est la \emph{Free
    Software Foundation}.

  \begin{itemize}
  \item Distribu� sous la GNU \emph{General Public License} (GPL),
    donc gratuit, ou �libre�.
  \item Le nom provient de �\emph{Editing MACroS}�.
  \item La premi�re version de Emacs a �t� �crite par Richard M.\
    Stallman, pr�sident de la FSF.
  \end{itemize}
\end{frame}

\begin{frame}
  \frametitle{Configuration de l'�diteur}

  Une des grandes forces de Emacs est d'�tre configurable � l'envi.

  \begin{itemize}
  \item Depuis la version 21, le menu \texttt{Customize} rend la
    configuration ais�e.
  \item Une grande part de la configuration provient du fichier
    \texttt{.emacs}:
    \begin{itemize}
    \item nomm� \texttt{.emacs} sous Linux et Unix, Windows 2000 et
      Windows XP;
    \item sous Windows 95/98/Me, utiliser plut�t \texttt{\_emacs}.
    \end{itemize}
  \end{itemize}
\end{frame}

\begin{frame}
  \frametitle{\emph{Emacs-ismes} et \emph{Unix-ismes}}

  \begin{itemize}
  \item Un \emph{buffer} contient un fichier ouvert
    (�\emph{visited}�).  �quivalent � une fen�tre dans Windows.
  \item Le \emph{minibuffer} est la r�gion au bas de l'�cran Emacs o�
    l'on entre des commandes et re�oit de l'information de Emacs.
  \item La ligne de mode (�\emph{mode line}�) est le s�parateur
    horizontal contenant diverses informations sur le fichier ouvert
    et l'�tat de Emacs.
  \end{itemize}
\end{frame}

\begin{frame}
  \begin{itemize}
  \item Toutes les fonctionnalit�s de Emacs correspondent � une
    commande pouvant �tre tap�e dans le \emph{minibuffer}.
    \texttt{M-x} d�marre l'interpr�teur (ou invite) de commandes.
  \item Dans les d�finitions de raccourcis claviers:
    \begin{itemize}[<+(1)->]
    \item \texttt{C} est la touche \texttt{Ctrl} (\texttt{Control});
    \item \texttt{M} est la touche \texttt{Meta}, qui correspond � la
      touche \texttt{Alt} de gauche sur un PC;
    \item \texttt{ESC} est la touche \texttt{�chap} (\texttt{Esc}) et
      est �quivalente � \texttt{Meta};
    \item \texttt{SPC} est la barre d'espacement;
    \item \texttt{RET} est la touche Entr�e.
    \end{itemize}
  \end{itemize}
\end{frame}

\begin{frame}[fragile]
  \begin{itemize}[<+->]
  \item Le caract�re \verb=~= repr�sente le dossier vers lequel pointe
    la variable d'environnement \texttt{\$HOME} (Unix) ou
    \texttt{\%HOME\%} (Windows).
  \item La barre oblique (\texttt{/}) est utilis�e pour s�parer les
    dossiers dans les chemins d'acc�s aux fichiers, m�me sous Windows.
  \item En g�n�ral, il est possible d'appuyer sur \texttt{TAB} dans le
    \emph{minibuffer} pour compl�ter les noms de fichiers ou de
    commandes.
  \end{itemize}
\end{frame}

\begin{frame}
  \frametitle{Commandes d'�dition de base}

  \begin{itemize}
  \item Il n'est pas vain de lire le tutoriel de Emacs, que l'on
    d�marre avec
    \begin{quote}
      \texttt{C-h t}
    \end{quote}
  \item Pour une liste plus exhaustive des commandes Emacs les plus
    importantes, consulter la \emph{GNU Emacs Reference Card} �
    l'adresse
    \begin{quote}
      \url{http://refcards.com/refcards/gnu-emacs/}
    \end{quote}
  \end{itemize}
\end{frame}

\begin{frame}
  \begin{itemize}
  \item<1-> Pour cr�er un nouveau fichier, ouvrir un fichier n'existant
    pas.\index{Emacs!nouveau fichier}
  \item<2-> Principales commandes d'�dition avec, entre parenth�ses, le
    nom de la commande correspondant au raccourci clavier:
    \begin{ttscript}{C-x C-w}
      \raggedright
    \item[\emacs{C-x C-f}] ouvrir un fichier
    \item[\emacs{C-x C-s}] sauvegarder
    \item[\emacs{C-x C-w}] sauvegarder sous
    \item[\emacs{C-x k}] fermer un fichier
    \item[\emacs{C-x C-c}] quitter Emacs
    \end{ttscript}
  \end{itemize}
\end{frame}

\begin{frame}
  \begin{itemize}
  \item[]
    \begin{ttscript}{C-x C-w}
    \item[\emacs{C-g}] bouton de panique: quitter!
    \item[\emacs{C-\_}] annuler (pratiquement illimit�); aussi
      \emacs{C-x u}
      \\[\baselineskip]
    \item[\emacs{C-s}] recherche incr�mentale avant
    \item[\emacs{C-r}] Recherche incr�mentale arri�re
    \item[\emacs{M-\%}] rechercher et remplacer
    \end{ttscript}
  \end{itemize}
\end{frame}

\begin{frame}
  \raggedright
  \begin{itemize}
  \item[]
    \begin{ttscript}{C-x C-w}
    \item[\emacs{C-x b}] changer de \emph{buffer}
    \item[\emacs{C-x 2}] s�parer l'�cran en deux fen�tres
    \item[\emacs{C-x 1}] conserver uniquement la fen�tre courante
    \item[\emacs{C-x 0}] fermer la fen�tre courante
    \item[\emacs{C-x o}] aller vers une autre fen�tre lorsqu'il y en a
      plus d'une
    \end{ttscript}
  \end{itemize}
\end{frame}

\begin{frame}
  \frametitle{S�lection de texte}

  La s�lection de texte fonctionne diff�remment du standard Windows.

  \begin{itemize}
  \item<1-> Les raccourcis clavier standards sous Emacs sont:
    \begin{ttscript}{C-SPC}
      \raggedright
    \item[\emacs{C-SPC}] d�bute la s�lection
    \item[\emacs{C-w}] couper la s�lection
    \item[\emacs{M-w}] copier la s�lection
    \item[\emacs{C-y}] coller
    \item[\emacs{M-y}] remplacer le dernier texte coll� par la
      s�lection pr�c�dente
    \end{ttscript}
  \item<2-> Il existe quelques extensions de Emacs permettant d'utiliser
    les raccourcis clavier usuels de Windows (\texttt{C-c},
    \texttt{C-x}, \texttt{C-v}).
  \end{itemize}
\end{frame}


\subsection{Mode ESS}

\begin{frame}
  \frametitle{\emph{Emacs Speaks Statistics}}

  Mode pour interagir avec des logiciels statistiques (S-Plus,
  \textsf{R}, SAS, etc.) depuis Emacs.

  \begin{itemize}
  \item Voir
    \begin{quote}
      \url{http://ess.r-project.org/}
    \end{quote}
    pour la documentation compl�te.
  \item Deux modes mineurs: \texttt{ESS} pour les fichiers de script
    (code source) et \texttt{iESS} pour l'invite de commande.
  \item Une fois install�, le mode mineur \texttt{ESS} s'active
    automatiquement en �ditant des fichiers avec l'extension
    \texttt{.S} ou \texttt{.R}.
  \end{itemize}
\end{frame}

\begin{frame}
  \frametitle{D�marrer un processus S}

  \begin{itemize}
  \item Pour d�marrer un processus S et activer le mode mineur
    \texttt{iESS}, entrer l'une des commandes
    \begin{itemize}
    \item \texttt{S}
    \item \texttt{Sqpe} ou
    \item \texttt{R}
    \end{itemize}
    dans l'invite de commande de Emacs
  \item Par exemple, pour d�marrer un processus \textsf{R} �
    l'int�rieur de Emacs:
    \begin{itemize}
    \item[] \ttfamily M-x R RET
    \end{itemize}
  \end{itemize}
\end{frame}

\begin{frame}
  \frametitle{Raccourcis clavier les plus utiles � la ligne de
    commande (mode \texttt{iESS})}
  
  \begin{ttscript}{C-c C-c}
    \raggedright
  \item[\ess{C-c C-e}] replacer la derni�re ligne au bas de la
    fen�tre
  \item[\ess{M-h}] s�lectionner le r�sultat de la derni�re commande
  \item[\ess{C-c C-o}] effacer le r�sultat de la derni�re commande
  \item[\ess{C-c C-v}] aide sur une commande S
  \item[\ess{C-c C-q}] terminer le processus S
  \end{ttscript}
\end{frame}

\begin{frame}
  \frametitle{Raccourcis clavier les plus utiles lors de l'�dition
    d'un fichier de script (mode \texttt{ESS})}
  
  \begin{ttscript}{C-c C-c}
    \raggedright
  \item[\ess{C-c C-n}] �valuer la ligne sous le curseur dans le
    processus S, puis d�placer le curseur � la prochaine ligne de
    commande
  \item[\ess{C-c C-r}] �valuer la r�gion s�lectionn�e dans le
    processus S
  \item[\ess{C-c C-f}] �valuer le code de la fonction courante dans
    le processus S
  \item[\ess{C-c C-l}] �valuer le code du fichier courant dans le
    processus S
  \item[\ess{C-c C-v}] aide sur une commande S
  \item[\ess{C-c C-s}] changer de processus (utile si l'on a plus
    d'un processus S actif)
  \end{ttscript}
\end{frame}

\begin{frame}[fragile=singleslide]
  \frametitle{Consultation des rubriques d'aide}
  
  \begin{itemize}
  \item Quelques avantages � lire les rubriques d'aide dans Emacs.
  \item Modifier l'option \textsf{R} \code{chmhelp} ainsi:
\begin{Sinput}
> options(chmhelp = FALSE)
\end{Sinput}
  \item Pour que la commande s'ex�cute automatiquement � chaque
  lancement de \textsf{R}, entrer la commande dans un fichier nomm�
  \code{.Rprofile} sauvegard� dans le dossier mentionn� dans le
  r�sultat de
\begin{Sinput}
> Sys.getenv("R_USER")  
\end{Sinput}
  \item Consulter aussi la rubrique d'aide de \code{Startup}.
  \end{itemize}
\end{frame}

\begin{frame}
  \frametitle{Raccourcis clavier utiles lors de la consultation des
    rubriques d'aide}

  \begin{ttscript}{m, m}
    \raggedright
  \item[\ess{h}] ouvrir une nouvelle rubrique d'aide, par d�faut pour le mot
    se trouvant sous le curseur
  \item[\ess{n}, \ess{p}] aller � la section suivante (\texttt{n}) ou
    pr�c�dente (\texttt{p}) de la rubrique
  \item[\ess{l}] �valuer la ligne sous le curseur; pratique pour
    ex�cuter les exemples
  \item[\ess{r}] �valuer la r�gion s�lectionn�e
  \item[\ess{q}] retourner au processus ESS en laissant la rubrique
    d'aide visible
  \item[\ess{x}] fermer la rubrique d'aide et retourner au processus
    ESS 
  \end{ttscript}
\end{frame}

%%% Local Variables: 
%%% mode: latex
%%% TeX-master: "introduction_programmation_S_slides"
%%% End: 


\end{document}

%%% Local Variables: 
%%% mode: latex
%%% TeX-master: t
%%% End: 
