\chapter{Utilisation de ESS et S-Plus sous Windows}
\index{Emacs!et S-Plus|(}
\label{s-plus_windows}

Avec ESS dans Emacs, l'utilisation de \textsf{R} et de S-Plus est
virtuellement identique sous Unix. Sous Windows, la procédure est
exactement la même que sous Unix pour \textsf{R}, mais l'interface
avec S-Plus est légèrement plus complexe.

Avant toute chose, il faut s'assurer d'avoir une installation de
Emacs, ESS et S-Plus fonctionnelle. L'installation de la version
modifiée de Emacs distribuée dans le site Internet
\begin{quote}
  \url{http://vgoulet.act.ulaval.ca/emacs/}
\end{quote}
devrait permettre de satisfaire cette exigence rapidement.

Il y a deux façons de travailler avec S-Plus depuis Emacs sous
Windows: tout dans Emacs ou une combinaison de Emacs et de l'interface
graphique de S-Plus.


\section{Tout dans Emacs}

L'approche tout dans Emacs est similaire à celle favorisée sous Unix
ainsi qu'avec \textsf{R}. Elle consiste à démarrer un processus S-Plus
à l'intérieur même de Emacs, à ouvrir un fichier de script
(habituellement avec une extension \texttt{.S}) dans Emacs et à
exécuter les lignes de ce fichier dans le processus S-Plus.  La
fenêtre Emacs est alors scindée en deux. C'est l'approche prônée à la
section \ref{presentation:strategies}.

Le truc consiste ici à utiliser non pas l'exécutable
\texttt{splus.exe} (qui est l'interface graphique), mais plutôt
l'interface en ligne de commande, plus simple et rapide. L'exécutable
est \texttt{sqpe.exe}. Pour démarrer une session S-Plus dans Emacs, on
fera donc
\begin{quote}
  \ttfamily M-x Sqpe RET
\end{quote}
Après avoir spécifié le dossier de travail, l'invite de commande
S-Plus apparaît. On exécute par la suite les lignes d'un fichier de
script dans le processus S-Plus avec \texttt{C-c C-n}, \texttt{C-c
  C-f}, etc.

Il y a toutefois un os avec cette approche: aucun périphérique
graphique n'est disponible. Cependant, depuis la version 6.1 de
S-Plus, on peut utiliser un périphérique graphique Java mais il faut
exécuter les deux lignes suivantes \emph{avant} de créer un graphique:
\begin{Schunk}
\begin{Sinput}
> library(winjava)
> java.graph()
\end{Sinput}
\end{Schunk}

Il est possible d'automatiser ce processus en sauvegardant ces deux
lignes dans un fichier nommé \texttt{S.init} dans le dossier de
travail. Le contenu de ce fichier sera exécuté chaque fois que l'on
démarrera S-Plus dans ce dossier.


\section[Combinaison Emacs et S-Plus GUI]{Combinaison Emacs et interface graphique de S-Plus}

Moins élégante que la précédente, cette seconde option présente, pour
certains, l'avantage d'utiliser l'interface graphique (GUI) de S-Plus.
En fin de compte, la procédure ci-dessous revient à remplacer par
Emacs la fenêtre d'édition de script incluse dans S-Plus.

En faisant
\begin{quote}
  \ttfamily M-x S RET
\end{quote}
à l'intérieur de Emacs on démarrera une nouvelle session graphique de
S-Plus (il faut être patient, les négociations entre les deux
logiciels peuvent prendre du temps). On se retrouve donc avec deux
fenêtres: une pour Emacs et une pour S-Plus.

Ouvrir un fichier de script dans Emacs et exécuter les lignes de code
comme ci-dessus. Les lignes de code seront exécutées dans l'interface
graphique. En d'autres mots, le code source se trouve dans une fenêtre
(Emacs) et les résultats de ce code source dans une autre (S-Plus). Il
faut bien disposer les fenêtres côte à côte pour que cette stratégie
se révèle minimalement efficace.

L'information ci-dessus se trouve dans la documentation de ESS.

\index{Emacs!et S-Plus|)}

%%% Local Variables:
%%% mode: latex
%%% TeX-master: "introduction_programmation_S"
%%% coding: utf-8
%%% End:
