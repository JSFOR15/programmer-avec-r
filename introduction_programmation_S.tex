\documentclass[letterpaper,10pt]{memoir}
  \usepackage[latin1]{inputenc}
  \usepackage[T1]{fontenc}
  \usepackage[english,francais]{babel}
  \usepackage{vgmath,actu,amsmath,amsthm,icomma,url,natbib}
  \usepackage{lucidabr,palatino,mathpazo}
  \usepackage{textcomp,fourier-orns,mathabx}
  \usepackage[noae]{Sweave}
  \usepackage{graphicx,color}
  \usepackage[absolute]{textpos}
  \usepackage{threeparttable,paralist,expdlist}
  \usepackage{listings,answers}

  %%% Page titre
  \title{\HUGE 
    \fontseries{ub}\selectfont Introduction \\ 
    \fontseries{m}\selectfont  � la \\
    \fontseries{ub}\selectfont programmation \\ 
    \fontseries{m}\selectfont  en \\ 
    \fontseries{ub}\selectfont S}
  \author{\huge Vincent Goulet \\[3mm]
    \large \textit{�cole d'actuariat, Universit� Laval}}
  \date{\Large Seconde �dition}
  \newcommand{\ISBN}{978-2-9809136-7-9}

  %%% Commandes pour notes marginales
  \newcommand{\R}{\marginpar[\hfill \sffamily R]{\sffamily R}}
  \newcommand{\Splus}{\marginpar[\hfill S$+$]{S$+$}}
  \newcommand{\warning}{\marginpar[\hfill\LARGE\danger]%
    {\LARGE\danger}}
  \setlength{\marginparsep}{7mm}
  \setlength{\marginparwidth}{13mm}

  %%% Style des ent�tes de chapitres
  \chapterstyle{hangnum}

  %%% Styles des ent�tes et pieds de page
  \setlength{\headwidth}{\textwidth}
  \addtolength{\headwidth}{\marginparsep}
  \addtolength{\headwidth}{\marginparwidth}

  %%% Num�roter les sous-sections
  \maxsecnumdepth{subsection}
  \setsecnumdepth{subsection}

  %%% Styles pour les environnements d'exemples et al.
  \theoremstyle{plain}
  \newtheorem{ex}{Exemple}[chapter]
  \theoremstyle{remark}
  \newtheorem*{rem}{Remarque}
  \newtheorem*{rems}{Remarques}
  \theoremstyle{definition}
  \newtheorem*{astuce}{Astuce}
  \newenvironment{sol}{\begin{proof}[Solution]}{\end{proof}}

  %%% Environnements pour les exercices et r�ponses. On utilise
  %%% \thechapter plut�t que \arabic{chapter} pour la num�rotation des
  %%% exercices dans les annexes.
  \Newassociation{rep}{reponse}{reponses}
  \newcounter{exercice}[chapter]
  \newenvironment{exercice}{%
    \begin{list}{\bfseries \thechapter.\arabic{exercice}}{%
        \refstepcounter{exercice}
        \settowidth{\labelwidth}{\bfseries \thechapter.\arabic{exercice}}
        \setlength{\leftmargin}{\labelwidth}
        \addtolength{\leftmargin}{\labelsep}
        \setdefaultenum{a)}{i)}{}{}}\item}
    {\end{list}}
  \renewenvironment{reponse}[1]{%
    \begin{list}{\bfseries #1}{%
        \settowidth{\labelwidth}{#1}
        \setlength{\leftmargin}{\labelwidth}
        \addtolength{\leftmargin}{\labelsep}
        \setdefaultenum{a)}{i)}{}{}}\item}
    {\end{list}}
  \renewcommand{\reponseparams}{{\thechapter.\theexercice}}

  %%% Environnement pour les listes de commandes
  \newenvironment{ttscript}[1]{%
    \begin{list}{}{%
        \setlength{\labelsep}{1.5ex}
        \settowidth{\labelwidth}{\code{#1}}
        \setlength{\leftmargin}{\labelwidth}
        \addtolength{\leftmargin}{\labelsep}
        \setlength{\parsep}{0.5ex plus0.2ex minus0.2ex}
        \setlength{\itemsep}{0.3ex}
        \renewcommand{\makelabel}[1]{##1\hfill}}}
    {\end{list}}

  %%% Environnement pour les listes de structures de contr�le
  \newenvironment{struclist}{%
    \begin{description}[\breaklabel\setlabelstyle{\mdseries\ttfamily}%
      \setleftmargin{\parindent}]}
    {\end{description}}

  %%% Param�tres pour les sections de code source
  \lstloadlanguages{R}
  \lstdefinelanguage{Renhanced}[]{R}{%
    morekeywords={colMeans,colSums,head,is.na,is.null,mapply,ms,na.rm,%
      nlmin,replicate,row.names,rowMeans,rowSums,sys.time,system.time,%
      tail,which.max,which.min},
    deletekeywords={c},
    alsoletter={.\%},%
    alsoother={:_\$}}
  \lstset{language=Renhanced,extendedchars=true,
    basicstyle=\small\ttfamily,
    commentstyle=\textsl,
    keywordstyle=\mdseries,
    showstringspaces=false,
    index=[1][keywords], 
    indexstyle=\indexfonction}

  %%% Styles pour les noms de fonctions, code, etc.
  \newcommand{\code}[1]{\texttt{#1}}

  %%% Index
  \renewcommand{\preindexhook}{%
    Les num�ros de page en caract�res gras indiquent les pages o� les
    concepts sont introduits, d�finis ou expliqu�s.\vskip\onelineskip}
  \newcommand{\Index}[1]{\index{#1|textbf}}
  \newcommand{\indexargument}[1]{\index{#1@\code{#1}}}
  \newcommand{\Indexargument}[1]{\Index{#1@\code{#1}}}
  \newcommand{\indexattribut}[1]{\index{#1@\code{#1} (attribut)}}
  \newcommand{\Indexattribut}[1]{\Index{#1@\code{#1} (attribut)}}
  \newcommand{\indexclasse}[1]{\index{#1@\code{#1} (classe)}}
  \newcommand{\Indexclasse}[1]{\Index{#1@\code{#1} (classe)}}
  \newcommand{\indexfonction}[1]{\index{#1@\code{#1}}}
  \newcommand{\Indexfonction}[1]{\Index{#1@\code{#1}}}
  \newcommand{\indexmode}[1]{\index{#1@\code{#1} (mode)}}
  \newcommand{\Indexmode}[1]{\Index{#1@\code{#1} (mode)}}
  \newcommand{\indexobjet}[1]{\index{#1@\code{#1}}}
  \newcommand{\Indexobjet}[1]{\Index{#1@\code{#1}}} 
  \newcommand{\indexemacs}[1]{\index{Emacs!#1@\texttt{#1}}}
  \newcommand{\indexess}[1]{\index{ESS!#1@\texttt{#1}}}

  \newcommand{\attribut}[1]{\code{#1}\indexattribut{#1}}
  \newcommand{\Attribut}[1]{\code{#1}\Indexattribut{#1}}
  \newcommand{\argument}[1]{\code{#1}\indexargument{#1}}
  \newcommand{\Argument}[1]{\code{#1}\Indexargument{#1}}
  \newcommand{\classe}[1]{\code{#1}\indexclasse{#1}}
  \newcommand{\Classe}[1]{\code{#1}\Indexclasse{#1}}
  \newcommand{\fonction}[1]{\code{#1}\indexfonction{#1}}
  \newcommand{\Fonction}[1]{\code{#1}\Indexfonction{#1}}
  \newcommand{\mode}[1]{\code{#1}\indexmode{#1}}
  \newcommand{\Mode}[1]{\code{#1}\Indexmode{#1}}
  \newcommand{\objet}[1]{\code{#1}\indexobjet{#1}}
  \newcommand{\Objet}[1]{\code{#1}\Indexobjet{#1}}
  \newcommand{\emacs}[1]{\code{#1}\indexemacs{#1}}
  \newcommand{\ess}[1]{\code{#1}\indexess{#1}}
  \makeindex

%  \includeonly{pagetitre}

\begin{document}

\renewcommand{\FrenchLabelItem}{\small$\sqbullet$}
\renewcommand{\labelitemii}{\textendash}

\shortcites{Rintro}

\frontmatter

\pagestyle{empty}
\newlength{\gauche}
\newlength{\droite}

\calccentering{\unitlength}
\addtolength{\gauche}{\unitlength}
\addtolength{\gauche}{-1.5cm}
\addtolength{\droite}{\unitlength}
\addtolength{\droite}{1.5cm}
\begin{adjustwidth*}{\gauche}{-\droite}
  \centering
  \vspace*{2cm}
  \thetitle \\
\end{adjustwidth*}
\cleardoublepage

\begin{adjustwidth*}{\gauche}{-\droite}
  \centering
  \vspace*{2cm}
  \thetitle \\
  \vspace*{3cm}
  \theauthor \\
  \vspace*{\fill}
  \thedate
\end{adjustwidth*}
\clearpage

\begingroup
\begin{adjustwidth*}{\unitlength}{-\unitlength}
  \footnotesize 
  \setlength{\parindent}{0pt}
  \setlength{\parskip}{\baselineskip}
  
  \textcopyright{} 2005 Vincent Goulet \\
  Tous droits r�serv�s

  Il est permis de copier, distribuer et/ou modifier ce document selon
  les termes de la GNU Free Documentation License, Version 1.2 ou
  toute version subs�quente publi�e par la Free Software Foundation;
  avec aucune section inalt�rable (\emph{Invariant Sections}), aucun
  texte de couverture avant (\emph{Front-Cover Texts}), et aucun texte
  de couverture arri�re (\emph{Back-Cover Texts}). Une copie de la
  licence est incluse dans l'annexe \ref{fdl}.

  \begin{center}
    \begin{tabular}{ll}
      Version 0.99: & 29 novembre 2005 \\
      Version 0.9:  & 9 novembre 2005 \\
      Version 0.8:  & 16 septembre 2005 \\
    \end{tabular}
  \end{center}

  \textbf{Avis de marque de commerce} \\
  S-Plus\textregistered{} est une marque d�pos�e de Insightful
  Corporation.
\end{adjustwidth*}
\endgroup
\clearpage

%%% Local Variables: 
%%% mode: latex
%%% TeX-master: "introduction_programmation_S"
%%% End: 


\pagestyle{companion}
\chapter*{Introduction}
\addcontentsline{toc}{chapter}{Introduction}
\markboth{Introduction}{Introduction}


Depuis maintenant plus d'une décennie, le système R connaît une
progression remarquable dans ses fonctionnalités, dans la variété de
ses domaines d'application ou, plus simplement, dans le nombre de ses
utilisateurs. La documentation disponible a suivi la même tangente,
plusieurs maisons d'édition ayant démarré des collections dédiées
spécifiquement aux utilisations que l'on fait de R en sciences
naturelles, en sciences sociales, en finance, etc. Néanmoins, peu
d'ouvrages se concentrent sur l'apprentissage de R en tant que langage
de programmation sous-jacent aux fonctions statistiques. C'est la
niche que nous tâchons d'occuper.

Cette troisième édition de l'ouvrage se distingue de la précédente
\citep{Goulet:introS:2007} à plusieurs égards. Tout d'abord, le
nouveau titre indique déjà que nous ne traitons plus du système
S-PLUS. S'il y a eu lutte d'importance et d'influence entre le
logiciel libre R et le logiciel commercial S-PLUS, la cause est
aujourd'hui entendue: R a gagné. Ensuite, l'ensemble du texte a fait
l'objet d'une révision et d'une mise à jour en profondeur, en
particulier les trois premiers chapitres. Enfin, même si nous
conservons un biais en faveur du tandem GNU~Emacs et ESS pour
interagir avec R, la présentation est moins liée à ces outils.

L'ouvrage est basé sur des notes de cours et des exercices utilisés à
l'École d'actuariat de l'Université Laval. L'enseignement du langage R
est axé sur l'exposition à un maximum de code --- que nous avons la
prétention de croire bien écrit --- et sur la pratique de la
programmation. C'est pourquoi les chapitres sont rédigés de manière
synthétique et qu'ils comportent peu d'exemples au fil du texte. En
revanche, le lecteur est appelé à lire et à exécuter le code
informatique se trouvant dans les sections d'exemples à la fin de
chacun des chapitres. Ce code et les commentaires qui l'accompagnent
reviennent sur l'essentiel des concepts du chapitre et les
complémentent souvent. Nous considérons l'exercice d'«étude active»
consistant à exécuter du code et à voir ses effet comme essentielle à
l'apprentissage du langage R.

Le texte des sections d'exemples est disponible en format électronique
sous la rubrique de la documentation par des tiers
\emph{(Contributed)} du site \emph{Comprehensive R Archive Network}:
\begin{quote}
  \url{http://cran.r-project.org/other-docs.html}
\end{quote}

Certains exemples et exercices trahissent le premier public de ce
document: on y fait à l'occasion référence à des concepts de base de
la théorie des probabilités et des mathématiques financières. Les
contextes actuariels demeurent néanmoins peu nombreux et ne devraient
généralement pas dérouter le lecteur pour qui ces notions sont moins
familières.

Nous tenons à remercier M.~Mathieu Boudreault pour sa collaboration
dans la rédaction des exercices et Mme~Mireille Côté pour la révision
linguistique de la seconde édition.

\begin{flushright}
  Vincent Goulet \\
  Québec, avril 2012
\end{flushright}


%%% Local Variables:
%%% mode: latex
%%% TeX-master: "introduction_programmation_r"
%%% coding: utf-8
%%% End:

\tableofcontents*

\mainmatter
\chapter{Pr�sentation du langage S}
\label{presentation}


\section{Le langage S}
\label{presentation:langage}

Le S est un langage pour �programmer avec des donn�es� d�velopp� chez
Bell Laboratories (anciennement propri�t� de AT\&T, maintenant de
Lucent Technologies).

\begin{itemize}
\item Ce n'est pas seulement un �autre� environnement statistique
  (comme SPSS ou SAS, par exemple), mais bien un langage de
  programmation complet et autonome.
\item Inspir� de plusieurs langages, dont l'APL et le Lisp, le S est:
  \begin{itemize}
  \item interpr�t� (et non compil�);
  \item sans d�claration obligatoire des variables;
  \item bas� sur la notion de vecteur;
  \item particuli�rement puissant pour les applications math�matiques
    et statistiques (et donc actuarielles).
  \end{itemize}
\end{itemize}


\section{Les moteurs S}
\label{presentation:moteurs}

Il existe quelques �moteurs� ou dialectes du langage S.

\begin{itemize}
\item Le plus connu est S-Plus, un logiciel commercial de
  Insightful Corporation (Bell Labs octroie � Insightful la licence
  exclusive de son syst�me S).
\item \textsf{R}, ou GNU S, est une version libre (\emph{Open Source})
  �\emph{not unlike S}�.
\end{itemize}

S-Plus et \textsf{R} constituent tous deux des environnements int�gr�s
de manipulation de donn�es, de calcul et de pr�paration de graphiques.


\section{O� trouver de la documentation}
\label{presentation:doc}

S-Plus est livr� avec quatre livres (disponibles en format PDF depuis
le menu \texttt{Help} de l'interface graphique), mais aucun ne s'av�re
vraiment utile pour apprendre le langage S.

Plusieurs livres --- en versions papier ou �lectronique, gratuits ou
non --- ont �t� publi�s sur S-Plus et \textsf{R}. On trouvera des
listes exhaustives dans les sites de Insightful et du projet
\textsf{R}:
\begin{itemize}
\item \url{http://www.insightful.com/support/splusbooks.asp}
\item \url{http://www.r-project.org} (dans la section
  \texttt{Documentation}).
\end{itemize}

De plus, les ouvrages de \citet{Sprogramming,MASS} constituent des
r�f�rences sur le langage S devenues au cours des derni�res ann�es des
standards \emph{de facto}.


\section{Interfaces pour S-Plus et \textsf{R}}
\label{presentation:interfaces}

Provenant du monde Unix, tant S-Plus que \textsf{R} sont d'abord et
avant tout des applications en ligne de commande (\texttt{sqpe.exe} et
\texttt{rterm.exe} sous Windows).

\begin{itemize}
\item S-Plus poss�de toutefois une interface graphique �labor�e
  permettant d'utiliser le logiciel sans trop conna�tre le langage de
  programmation.
\item \textsf{R} dispose �galement d'une interface graphique
  rudimentaire sous Windows et Mac OS.
\item L'�dition s�rieuse de code S b�n�ficie cependant grandement d'un
  bon �diteur de texte.
\item � la question 6.2 de la foire aux questions (FAQ) de \textsf{R},
  �Devrais-je utiliser \textsf{R} � l'int�rieur de
  Emacs?�\index{Emacs}, la r�ponse est: �Oui, absolument.� Nous
  partageons cet avis, aussi ce document supposera-t-il que S-Plus ou
  \textsf{R} sont utilis�s � l'int�rieur de GNU Emacs avec le mode
  ESS\index{ESS}.
\item Autre option: WinEdt (partagiciel) avec l'ajout R-WinEdt.
\end{itemize}


\section{Installation de Emacs avec ESS}
\label{presentation:emacs}

Il n'existe pas de proc�dure d'installation similaire aux autres
applications Windows pour Emacs.  L'installation n'en demeure pas
moins tr�s simple: il suffit de d�compresser un ensemble de fichiers
au bon endroit.

\begin{itemize}
\item Pour une installation simplifi�e de Emacs et ESS, consulter le
  site Internet
  \begin{quote}
    \url{http://vgoulet.act.ulaval.ca/pub/emacs/}
  \end{quote}
  On y trouve une version modifi�e de GNU Emacs et des instructions
  d'installation d�taill�es.
\item L'annexe \ref{ess} pr�sente les plus importantes commandes �
  conna�tre pour utiliser Emacs et le mode ESS.
\end{itemize}


\section{D�marrer et quitter S-Plus ou \textsf{R}}
\label{presentation:demarrer}

On suppose ici que S-Plus ou R sont utilis�s � l'int�rieur de Emacs.

\begin{itemize}
\item Pour d�marrer \textsf{R} \R � l'int�rieur de Emacs:
\begin{verbatim}
M-x R RET
\end{verbatim}
  puis sp�cifier un dossier de travail (voir la section
  \ref{presentation:workspace}). Une console \textsf{R} est ouverte
  dans une fen�tre (\emph{buffer} dans la terminologie de Emacs)
  nomm�e \texttt{*R*}.
\item Pour d�marrer S-Plus sous Windows, \Splus la proc�dure est
  similaire, sauf que la commande � utiliser est
\begin{verbatim}
M-x Sqpe RET
\end{verbatim}
  Consulter l'annexe \ref{s-plus_windows} pour de plus amples
  d�tails.
\item Pour quitter, deux options sont disponibles:
  \begin{enumerate}
  \item Taper \fonction{q()} � la ligne de commande.
  \item Dans Emacs, faire \ess{C-c C-q}. ESS va alors s'occuper de
    fermer le processus S ainsi que tous les \emph{buffers} associ�s �
    ce processus.
  \end{enumerate}
\end{itemize}


\section{Strat�gies de travail}
\label{presentation:strategies}

Il existe principalement deux fa�ons de travailler avec S-Plus et
\textsf{R}.
\begin{enumerate}
\item Le code est virtuel et les objets sont r�els. C'est l'approche
  qu'encouragent les interfaces graphiques, mais c'est aussi la moins
  pratique � long terme. On entre des expressions directement � la
  ligne de commande pour les �valuer imm�diatement.
\begin{Schunk}
\begin{Sinput}
> 2 + 3
\end{Sinput}
\begin{Soutput}
[1] 5
\end{Soutput}
\begin{Sinput}
> -2 * 7
\end{Sinput}
\begin{Soutput}
[1] -14
\end{Soutput}
\begin{Sinput}
> exp(1)
\end{Sinput}
\begin{Soutput}
[1] 2.718282
\end{Soutput}
\begin{Sinput}
> log(exp(1))
\end{Sinput}
\begin{Soutput}
[1] 1
\end{Soutput}
\end{Schunk}
  Les objets cr��s au cours d'une session de travail sont sauvegard�s.
  Par contre, � moins d'avoir �t� sauvegard� dans un fichier, le code
  utilis� pour cr�er ces objets est perdu lorsque l'on quitte S-Plus
  ou \textsf{R}.
\item Le code est r�el et les objets sont virtuels. C'est l'approche
  que nous favorisons. Le travail se fait essentiellement dans des
  fichiers de script (de simples fichiers de texte) dans lesquels sont
  sauvegard�es les expressions (parfois complexes!) et le code des
  fonctions personnelles. Les objets sont cr��s au besoin en ex�cutant
  le code. Emacs permet ici de passer efficacement des fichiers de
  script � l'ex�cution du code:
  \begin{enumerate}[i)]
  \item d�marrer un processus S-Plus (\texttt{M-x Sqpe}) ou
    \textsf{R} (\texttt{M-x R}) et sp�cifier le dossier de travail;
  \item ouvrir un fichier de script avec \ess{C-x C-f}. Pour cr�er un
    nouveau fichier, ouvrir un fichier inexistant;
  \item positionner le curseur sur une expression et faire \ess{C-c
      C-n} pour l'�valuer;
  \item le r�sultat appara�t dans le \emph{buffer} \texttt{*S+6*} ou
    \texttt{*R*}.
  \end{enumerate}
\end{enumerate}


\section{Gestion des projets ou environnements de travail}
\label{presentation:workspace}

S-Plus et \textsf{R} ont une mani�re diff�rente mais tout aussi
particuli�re de sauvegarder les objets cr��s au cours d'une session de
travail.
\begin{itemize}
\item Tous deux doivent travailler dans un dossier et non avec des
  fichiers individuels.
\item Dans S-Plus, \Splus tout objet cr�� au cours d'une session de
  travail est sauvegard� de fa�on permanente sur le disque dur dans le
  sous-dossier \texttt{\_\_Data} du dossier de travail.
\item Dans \textsf{R}, \R les objets cr��s sont conserv�s en m�moire
  jusqu'� ce que l'on quitte l'application ou que l'on enregistre le
  travail avec la commande \fonction{save.image()}. L'environnement de
  travail (\emph{workspace}) est alors sauvegard� dans le fichier
  \texttt{.RData} du dossier de travail.
\end{itemize}

Le dossier de travail est d�termin� au lancement de l'application.
\begin{itemize}
\item Avec Emacs et ESS, on doit sp�cifier le dossier de travail
  chaque fois que l'on d�marre un processus S-Plus ou R.
\item Les interfaces graphiques permettent �galement de sp�cifier le
  dossier de travail.
  \begin{itemize}
    \sloppy
  \item Dans \Splus l'interface graphique de S-Plus, choisir
    \texttt{General Settings} dans le menu \texttt{Options}, puis
    l'onglet \texttt{Startup}. Cocher la case \texttt{Prompt for
      project folder}.  Consulter �galement le chapitre 13 du guide de
    l'utilisateur de S-Plus.
  \item Dans \R l'interface graphique de \textsf{R}, le plus simple
    consiste � changer le dossier de travail � partir du menu
    \texttt{Fichier|Changer le r�pertoire courant...} Consulter aussi
    la \emph{R for Windows FAQ}.
  \end{itemize}
\end{itemize}


\section{Consulter l'aide en ligne}
\label{presentation:aide}

Les rubriques d'aide des diverses fonctions disponibles dans S-Plus et
\textsf{R} contiennent une foule d'informations ainsi que des exemples
d'utilisation. Leur consultation est tout � fait essentielle.

\begin{itemize}
\item Pour consulter la rubrique d'aide de la fonction \code{foo},
  on peut entrer � la ligne de commande
\begin{Schunk}
\begin{Sinput}
> ?foo
\end{Sinput}
\end{Schunk}
\item Dans Emacs, \code{C-c C-v foo RET}\indexess{C-c C-v} ouvrira la
  rubrique d'aide de la fonction \code{foo} dans un nouveau
  \emph{buffer}.
\item Plusieurs touches de raccourcis facilitent la consultation des
  rubriques d'aide (voir la carte de r�f�rence ESS).
\item Entre autres, la touche \texttt{l} permet d'ex�cuter ligne par
  ligne les exemples se trouvant � la fin de chaque rubrique d'aide.
\end{itemize}


\section{Exemples}
\label{presentation:exemples}

\lstinputlisting{presentation.R}


\section{Exercices}
\label{presentation:exercices}

\begin{exercice}
  D�marrer un processus S-Plus ou \textsf{R} � l'int�rieur de Emacs.
\end{exercice}

\begin{exercice}
  Ex�cuter un � un les exemples de la section pr�c�dente. Une version
  �lectronique du code de cette section est disponible dans le site
  mentionn� dans la pr�face.
\end{exercice}

\begin{exercice}
  Consulter les rubriques d'aide d'une ou plusieurs des fonctions
  rencontr�es lors de l'exercice pr�c�dent. Observer d'abord comment
  les ru\-bri\-ques d'aide sont structur�es --- elles sont toutes
  identiques --- puis ex�cuter quelques lignes d'exemples.
\end{exercice}

\begin{exercice}
  Lire le chapitre 1 de \cite{MASS} et ex�cuter les commandes de
  l'exemple de session de travail de la section 1.3. Bien que
  davantage orient� vers les application statistiques que vers la
  programmation, cet exemple d�montre quelques-unes des possibilit�s
  du langage S.
\end{exercice}

%%% Local Variables:
%%% mode: latex
%%% TeX-master: "introduction_programmation_S"
%%% End:

\chapter{Bases du langage S}
\label{bases}


Ce chapitre pr�sente les bases du langage S, soit les notions
d'expression et d'affectation, la description d'un objet S et les
mani�res de cr�er les objets les plus usuels lorsque le S est utilis�
comme langage de programmation.

\section{Commandes S}
\label{bases:commandes}

Toute commande S est soit une \emph{expression}\index{expression},
soit une \emph{affectation}\index{affectation}.
\begin{itemize}
\item Normalement, une expression est imm�diatement �valu�e et le
  r�sultat est affich� � l'�cran:
\begin{Schunk}
\begin{Sinput}
> 2 + 3
\end{Sinput}
\begin{Soutput}
[1] 5
\end{Soutput}
\begin{Sinput}
> pi
\end{Sinput}
\begin{Soutput}
[1] 3.141593
\end{Soutput}
\begin{Sinput}
> cos(pi/4)
\end{Sinput}
\begin{Soutput}
[1] 0.7071068
\end{Soutput}
\end{Schunk}
\item Lors d'une affectation, une expression est �valu�e, mais le
  r�sultat est stock� dans un objet (variable) et rien n'est affich� �
  l'�cran. Le symbole d'affectation est \Fonction{<-} (ou
  \Fonction{->}).
\begin{Schunk}
\begin{Sinput}
> a <- 5
> a
\end{Sinput}
\begin{Soutput}
[1] 5
\end{Soutput}
\begin{Sinput}
> b <- a
> b
\end{Sinput}
\begin{Soutput}
[1] 5
\end{Soutput}
\end{Schunk}
\item �viter d'utiliser l'op�rateur \fonction{=} pour affecter une
  valeur � une variable, puisqu'il ne fonctionne que dans certaines
  situations seulement.
\item Dans S-Plus (mais plus dans \textsf{R} depuis la version 1.8.0),
  on peut �galement affecter avec le caract�re �\fonction{\_}�, mais cet
  emploi est fortement d�courag� puisqu'il rend le code difficile �
  lire. Dans le mode ESS de Emacs, taper ce caract�re g�n�re carr�ment
  \verb*| <- |.
\end{itemize}

\begin{astuce}
  Il arrive fr�quemment que l'on souhaite affecter le r�sultat d'un
  calcul dans un objet et en m�me temps voir ce r�sultat. Pour ce
  faire, placer l'affectation entre parenth�ses (l'op�ration
  d'affectation devient alors une nouvelle expression):
\begin{Schunk}
\begin{Sinput}
> (a <- 2 + 3)
\end{Sinput}
\begin{Soutput}
[1] 5
\end{Soutput}
\end{Schunk}
\end{astuce}


\section{Conventions pour les noms d'objets}
\index{noms d'objets!conventions}
\label{bases:noms}

Les caract�res permis pour les noms d'objets sont les lettres a--z,
A--Z, les chiffres 0--9 et le point �.�. Le caract�re \R �\code{\_}�
est maintenant permis dans \textsf{R}, mais son utilisation est
d�courag�e.

\begin{itemize}
\item Les noms d'objets ne peuvent commencer par un chiffre.
\item Le S est sensible � la casse, ce qui signifie que \code{foo},
  \code{Foo} et \code{FOO} sont trois objets distincts. Un moyen
  simple d'�viter des erreurs li�es � la casse consiste � n'employer
  que des lettres minuscules.
\item Certains noms sont utilis�s par le syst�me, aussi vaut-il mieux
  �viter de les utiliser. En particulier, �viter d'utiliser
  \begin{center}
    \code{c}, \code{q}, \code{t}, \code{C}, \code{D},
    \code{I}, \code{diff}, \code{length}, \code{mean},
    \code{pi}, \code{range}, \code{var}.
  \end{center}
\item \index{noms d'objets!r�serv�s} Certains mots sont r�serv�s pour
  le syst�me et il est interdit de les utiliser comme nom d'objet. Les
  mots r�serv�s sont:
  \begin{center}
    \code{Inf}, \code{NA}, \code{NaN},
    \code{NULL} \\
    \code{break}, \code{else},
    \code{for}, \code{function}, \code{if}, \code{in},
    \code{next}, \code{repeat}, \code{return}, \code{while}.
  \end{center}
\item Dans S-Plus 6.1 et plus, \Splus
  \code{T}\index{T@\code{T}|see{\code{TRUE}}} et \objet{TRUE} (vrai),
  ainsi que \code{F}\index{F@\code{F}|see{\code{FALSE}}} et
  \objet{FALSE} (faux) sont �galement des noms r�serv�s.
\item Dans \textsf{R}, \R les noms \code{TRUE} et \code{FALSE}
  sont �galement r�serv�s. Les variables \code{T} et \code{F}
  prennent par d�faut les valeurs \code{TRUE} et \code{FALSE},
  respectivement, mais peuvent �tre r�affect�es.
\begin{Schunk}
\begin{Sinput}
> T
\end{Sinput}
\begin{Soutput}
[1] TRUE
\end{Soutput}
\end{Schunk}
\begin{Schunk}
\begin{Sinput}
> TRUE <- 3
\end{Sinput}
\end{Schunk}
\begin{Schunk}
\begin{Soutput}
Erreur dans TRUE <- 3 : membre gauche de
l'assignation (do_set) incorrect
\end{Soutput}
\end{Schunk}
\begin{Schunk}
\begin{Sinput}
> (T <- 3)
\end{Sinput}
\begin{Soutput}
[1] 3
\end{Soutput}
\end{Schunk}
\end{itemize}


\section{Les objets S}
\label{bases:objets}

Tout dans le langage S est un objet, m�me les fonctions et les
op�rateurs. Les objets poss�dent au minimum un \emph{mode} et une
\emph{longueur}.

\begin{itemize}
\item Le mode d'un objet est obtenu avec la fonction \Fonction{mode}.
\begin{Schunk}
\begin{Sinput}
> v <- c(1, 2, 5, 9)
> mode(v)
\end{Sinput}
\begin{Soutput}
[1] "numeric"
\end{Soutput}
\end{Schunk}
\item La longueur d'un objet est obtenue avec la fonction
  \Fonction{length}.
\begin{Schunk}
\begin{Sinput}
> length(v)
\end{Sinput}
\begin{Soutput}
[1] 4
\end{Soutput}
\end{Schunk}
\item Certains objets sont �galement dot�s d'un ou plusieurs
  \emph{attributs}.
\end{itemize}


\subsection{Modes et types de donn�es}

Le mode\Index{mode} prescrit ce qu'un objet peut contenir. � ce titre,
un objet ne peut avoir qu'un seul mode. Le tableau \ref{tab:modes}
contient la liste des modes disponibles en S. � chacun de ces modes
correspond une fonction du m�me nom servant � cr�er un objet de ce
mode.

\begin{table}
  \centering
  \begin{tabular}{ll}
    \toprule
    \Mode{numeric}   & nombres r�els \\
    \Mode{complex}   & nombres complexes \\
    \Mode{logical}   & valeurs bool�ennes (vrai/faux) \\
    \Mode{character} & cha�nes de caract�res \\
    \Mode{function}  & fonction \\
    \Mode{list}      & donn�es quelconques \\
    \bottomrule
  \end{tabular}
  \caption{Modes disponibles et contenus correspondants}
  \label{tab:modes}
\end{table}

\subsection{Longueur}

La longueur\Index{longueur} d'un objet est �gale au nombre d'�l�ments
qu'il contient.

\begin{itemize}
\item La longueur d'une cha�ne de caract�res est toujours 1. Un
  objet de mode \code{character} doit contenir plusieurs cha�nes
  de caract�res pour que sa longueur soit sup�rieure � 1.
\begin{Schunk}
\begin{Sinput}
> v <- "actuariat"
> length(v)
\end{Sinput}
\begin{Soutput}
[1] 1
\end{Soutput}
\begin{Sinput}
> v <- c("a", "c", "t", "u", "a", "r", "i", 
+     "a", "t")
> length(v)
\end{Sinput}
\begin{Soutput}
[1] 9
\end{Soutput}
\end{Schunk}
\item Un objet peut �tre de longueur 0 et doit alors �tre interpr�t�
  comme un contenant vide\index{vide|see{\code{NULL}}}.
\begin{Schunk}
\begin{Sinput}
> v <- numeric(0)
> length(v)
\end{Sinput}
\begin{Soutput}
[1] 0
\end{Soutput}
\end{Schunk}
\end{itemize}


\subsection{Attributs}

Les attributs\Index{attribut} d'un objet sont des �l�ments
d'information additionnels li�s � cet objet. La liste des attributs
les plus fr�quemment rencontr�s se trouve au tableau
\ref{tab:attributs}. Pour chaque attribut, il existe une fonction du
m�me nom servant � extraire l'attribut correspondant d'un objet.

\begin{table}
  \centering
  \begin{tabular}{ll}
    \toprule
    \Attribut{class}    &
    affecte le comportement d'un objet \\
    \Attribut{dim}      &
    dimensions\index{dimension} des matrices et tableaux \\
    \Attribut{dimnames} &
    �tiquettes\index{etiquette@�tiquette} des dimensions des matrices
    et tableaux \\
    \Attribut{names}    &
    �tiquettes des �l�ments d'un objet \\
    \bottomrule
  \end{tabular}
  \caption{Attributs les plus usuels d'un objet et leur effet}
  \label{tab:attributs}
\end{table}


\subsection{L'objet sp�cial \code{NA}}

\Objet{NA} est fr�quemment utilis� pour repr�senter les donn�es
manquantes.

\begin{itemize}
\item Son mode est \mode{logical}.
\item Toute op�ration impliquant une donn�e \code{NA} a comme
  r�sultat \code{NA}.
\item Certaines fonctions (\fonction{sum}, \fonction{mean}, par
  exemple) ont par cons�quent un argument \Argument{na.rm} qui,
  lorsque \code{TRUE}, �limine les donn�es manquantes avant de
  faire un calcul.
\item La fonction \Fonction{is.na} permet de tester si les �l�ments
  d'un objet sont \code{NA} ou non.
\end{itemize}

\subsection{L'objet sp�cial \code{NULL}}

\Objet{NULL} repr�sente �rien�, ou le vide.

\begin{itemize}
\item Son mode est \Mode{NULL}.
\item Sa longueur est 0.
\item Diff�rent d'un objet vide:
  \begin{itemize}
  \item un objet de longueur 0 est un contenant vide;
  \item \code{NULL} est �pas de contenant�.
  \end{itemize}
\item La fonction \Fonction{is.null} teste si un objet est \code{NULL}
  ou non.
\end{itemize}


\section{Vecteurs}
\label{bases:vecteurs}

En S, � peu de choses pr�s, \emph{tout} est un vecteur\index{vecteur}.
(Il n'y a pas de notion de scalaire.)

\begin{itemize}
\item Dans un vecteur simple, tous les �l�ments doivent �tre du
  m�me mode.
\item Il est possible (et souvent souhaitable) de donner une �tiquette
  � chacun des �l�ments d'un vecteur.
\begin{Schunk}
\begin{Sinput}
> (v <- c(a = 1, b = 2, c = 5))
\end{Sinput}
\begin{Soutput}
a b c 
1 2 5 
\end{Soutput}
\begin{Sinput}
> v <- c(1, 2, 5)
> names(v) <- c("a", "b", "c")
> v
\end{Sinput}
\begin{Soutput}
a b c 
1 2 5 
\end{Soutput}
\end{Schunk}
\item Les fonctions de base pour cr�er des vecteurs sont:
  \begin{itemize}
  \item \Fonction{c} (concat�nation);
  \item \Fonction{numeric} (vecteur de mode \mode{numeric});
  \item \Fonction{logical} (vecteur de mode \mode{logical});
  \item \Fonction{character} (vecteur de mode \mode{character}).
  \end{itemize}
\item L'indi�age dans un vecteur se fait avec \fonction{[\ ]}. On
  peut extraire un �l�ment d'un vecteur par sa position ou par son
  �tiquette, si elle existe (auquel cas cette approche est beaucoup
  plus s�re).
\begin{Schunk}
\begin{Sinput}
> v[3]
\end{Sinput}
\begin{Soutput}
c 
5 
\end{Soutput}
\begin{Sinput}
> v["c"]
\end{Sinput}
\begin{Soutput}
c 
5 
\end{Soutput}
\end{Schunk}
  La section \ref{bases:indicage} traite plus en d�tail de l'indi�age de
  vecteurs et matrices.
\end{itemize}


\section{Matrices et tableaux}
\label{bases:matrices}

Une matrice\index{matrice} ou, de fa�on plus g�n�rale, un
tableau\index{tableau} (\emph{array}) n'est rien d'autre qu'un vecteur
dot� d'un attribut \attribut{dim}. � l'interne, une matrice est donc
stock�e sous forme de vecteur.

\begin{itemize}
\item La fonction de base pour cr�er des matrices est
  \Fonction{matrix}.
\item La fonction de base pour cr�er des tableaux est
  \Fonction{array}.
\item \emph{Important}: \warning les matrices et tableaux sont remplis
  en faisant d'abord varier la premi�re dimension, puis la seconde,
  etc. Pour les matrices, cela revient � remplir par colonne.
\begin{Schunk}
\begin{Sinput}
> matrix(1:6, nrow = 2, ncol = 3)
\end{Sinput}
\begin{Soutput}
     [,1] [,2] [,3]
[1,]    1    3    5
[2,]    2    4    6
\end{Soutput}
\begin{Sinput}
> matrix(1:6, nrow = 2, ncol = 3, byrow = TRUE)
\end{Sinput}
\begin{Soutput}
     [,1] [,2] [,3]
[1,]    1    2    3
[2,]    4    5    6
\end{Soutput}
\end{Schunk}
\item L'indi�age \index{indi�age!matrice} d'une matrice se fait
  �galement avec \fonction{[\ ]}. On extrait les �l�ments en pr�cisant
  leurs positions sous la forme (ligne, colonne) dans la matrice, ou
  encore leurs positions dans le vecteur sous-jacent.
\begin{Schunk}
\begin{Sinput}
> (m <- matrix(c(40, 80, 45, 21, 55, 32), 
+     nrow = 2, ncol = 3))
\end{Sinput}
\begin{Soutput}
     [,1] [,2] [,3]
[1,]   40   45   55
[2,]   80   21   32
\end{Soutput}
\begin{Sinput}
> m[1, 2]
\end{Sinput}
\begin{Soutput}
[1] 45
\end{Soutput}
\begin{Sinput}
> m[3]
\end{Sinput}
\begin{Soutput}
[1] 45
\end{Soutput}
\end{Schunk}
\item La fonction \Fonction{rbind} permet de fusionner verticalement
  deux matrices (ou plus) ayant le m�me nombre de colonnes.
\begin{Schunk}
\begin{Sinput}
> n <- matrix(1:9, nrow = 3)
> rbind(m, n)
\end{Sinput}
\begin{Soutput}
     [,1] [,2] [,3]
[1,]   40   45   55
[2,]   80   21   32
[3,]    1    4    7
[4,]    2    5    8
[5,]    3    6    9
\end{Soutput}
\end{Schunk}
\item La fonction \Fonction{cbind} permet de fusionner horizontalement
  deux matrices (ou plus) ayant le m�me nombre de lignes.
\begin{Schunk}
\begin{Sinput}
> n <- matrix(1:4, nrow = 2)
> cbind(m, n)
\end{Sinput}
\begin{Soutput}
     [,1] [,2] [,3] [,4] [,5]
[1,]   40   45   55    1    3
[2,]   80   21   32    2    4
\end{Soutput}
\end{Schunk}
\end{itemize}


\section{Listes}
\label{bases:listes}

Une liste\index{liste} est un type de vecteur sp�cial dont les
�l�ments peuvent �tre de n'importe quel mode, y compris le mode
\mode{list} (ce qui permet d'embo�ter des listes).

\begin{itemize}
\item La fonction de base pour cr�er des listes est \Fonction{list}.
\item Il est g�n�ralement pr�f�rable de nommer les �l�ments d'une
  liste. Il est en effet plus simple et s�r d'extraire les �l�ments
  par leur �tiquette.
\item L'extraction\Index{indi�age!liste} des �l�ments d'une liste peut
  se faire de deux fa�ons:
  \begin{enumerate}
  \item avec des doubles crochets \fonction{[[\ ]]};
  \item par leur �tiquette avec \code{nom.liste\$etiquette.element}.
  \end{enumerate}
\item La fonction \Fonction{unlist} convertit une liste en un vecteur
  simple. Attention, cette fonction peut �tre destructrice si la
  structure de la liste est importante.
\end{itemize}


\section{\emph{Data frames}}
\label{bases:dataframes}

Les vecteurs, matrices, tableaux (\emph{arrays}) et listes sont les
types d'objets les plus fr�quemment utilis�s en S pour la
programmation de fonctions personnelles ou la simulation. L'analyse de
donn�es --- la r�gression lin�aire, par exemple --- repose toutefois
davantage sur les \emph{data frames}\Index{data frame}.

\begin{itemize}
\item Un \emph{data frame} est une liste de classe \classe{data.frame}
  dont tous les �l�ments sont de la m�me longueur.
\item G�n�ralement repr�sent� sous la forme d'un tableau � deux
  dimensions (visuellement similaire � une matrice). Chaque �l�ment de
  la liste sous-jacente correspond � une colonne.
\item On peut donc obtenir les �tiquettes des colonnes avec la fonction
  \fonction{names} (ou \fonction{colnames} \R dans \textsf{R}).  Les
  �tiquettes des lignes sont quant � elles obtenues avec
  \fonction{row.names} (ou \fonction{rownames} dans \textsf{R}).
\item Plus g�n�ral qu'une matrice puisque les colonnes peuvent �tre de
  modes diff�rents (\mode{numeric}, \mode{complex}, \mode{character}
  ou \mode{logical}).
\item Peut �tre indic� � la fois comme une liste et comme une matrice.
\item Cr�� avec la fonction \Fonction{data.frame} ou
  \Fonction{as.data.frame} (pour convertir une matrice en \emph{data
    frame}, par exemple).
\item Les fonctions \fonction{rbind} et \fonction{cbind} peuvent �tre
  utilis�es pour ajouter des lignes ou des colonnes, respectivement.
\item On peut rendre les colonnes d'un \emph{data frame} (ou d'une
  liste) visibles dans l'espace de travail avec la fonction
  \Fonction{attach}, puis les masquer avec \Fonction{detach}.
\end{itemize}

Ce type d'objet est moins important lors de l'apprentissage du langage
de programmation.




\section{Indi�age}
\label{bases:indicage}

L'indi�age des vecteurs\Index{indi�age!vecteur} et
matrices\Index{indi�age!matrice} a d�j� �t� bri�vement pr�sent� aux
sections \ref{bases:vecteurs} et \ref{bases:matrices}. La pr�sente
section contient plus de d�tails sur cette proc�dure des plus communes
lors de l'utilisation du langage S. On se concentre toutefois sur le
traitement des vecteurs. Se r�f�rer �galement � \citet[section
2.3]{MASS} pour de plus amples renseignements.

Il existe quatre fa�ons d'indicer un vecteur dans le langage S. Dans
tous les cas, l'indi�age se fait � l'int�rieur de crochets \Fonction{[\ ]}.
\begin{enumerate}
\item Avec un vecteur d'entiers positifs. Les �l�ments se trouvant aux
  positions correspondant aux entiers sont extraits du vecteur, dans
  l'ordre. C'est la technique la plus courante.
\begin{Schunk}
\begin{Sinput}
> letters[c(1:3, 22, 5)]
\end{Sinput}
\begin{Soutput}
[1] "a" "b" "c" "v" "e"
\end{Soutput}
\end{Schunk}
\item Avec un vecteur d'entiers n�gatifs. Les �l�ments se trouvant aux
  positions correspondant aux entiers n�gatifs sont alors
  \emph{�limin�s} du vecteur.
\begin{Schunk}
\begin{Sinput}
> letters[c(-(1:3), -5, -22)]
\end{Sinput}
\begin{Soutput}
 [1] "d" "f" "g" "h" "i" "j" "k" "l" "m" "n" "o" "p"
[13] "q" "r" "s" "t" "u" "w" "x" "y" "z"
\end{Soutput}
\end{Schunk}
\item Avec un vecteur bool�en. Le vecteur d'indi�age doit alors �tre
  de la m�me longueur que le vecteur indic�. Les �l�ments
  correspondant � une valeur \code{TRUE} sont extraits du vecteur,
  alors que ceux correspondant � \code{FALSE} sont �limin�s.
\begin{Schunk}
\begin{Sinput}
> letters > "f" & letters < "q"
\end{Sinput}
\begin{Soutput}
 [1] FALSE FALSE FALSE FALSE FALSE FALSE  TRUE  TRUE
 [9]  TRUE  TRUE  TRUE  TRUE  TRUE  TRUE  TRUE  TRUE
[17] FALSE FALSE FALSE FALSE FALSE FALSE FALSE FALSE
[25] FALSE FALSE
\end{Soutput}
\begin{Sinput}
> letters[letters > "f" & letters < "q"]
\end{Sinput}
\begin{Soutput}
 [1] "g" "h" "i" "j" "k" "l" "m" "n" "o" "p"
\end{Soutput}
\end{Schunk}
\item Avec une cha�ne de caract�res. Utile pour extraire les �l�ments
  d'un vecteur � condition que ceux-ci soient nomm�s.
\begin{Schunk}
\begin{Sinput}
> x <- c(Rouge = 2, Bleu = 4, Vert = 9, Jaune = -5)
> x[c("Bleu", "Jaune")]
\end{Sinput}
\begin{Soutput}
 Bleu Jaune 
    4    -5 
\end{Soutput}
\end{Schunk}
\end{enumerate}


\section{Exemples}
\label{bases:exemples}

\lstinputlisting{bases.R}


\section{Exercices}
\label{bases:exercices}

\Opensolutionfile{reponses}[reponses-bases]
\Writetofile{reponses}{\protect\section*{Chapitre \protect\ref{bases}}}

\begin{exercice}
  \begin{enumerate}
  \item �crire une expression S pour cr�er la liste suivante:
\begin{Schunk}
\begin{Soutput}
[[1]]
[1] 1 2 3 4 5

$data
     [,1] [,2] [,3]
[1,]    1    3    5
[2,]    2    4    6

[[3]]
[1] 0 0 0

$test
[1] FALSE FALSE FALSE FALSE
\end{Soutput}
\end{Schunk}
  \item \index{etiquette@�tiquette} Extraire les �tiquettes de la
    liste.
  \item \index{mode} \index{longueur} Trouver le mode et la longueur
    du quatri�me �l�ment de la liste.
  \item \index{dimension} Extraire les dimensions du second �l�ment de
    la liste.
  \item \index{indi�age!liste} Extraire les deuxi�me et troisi�me
    �l�ments du second �l�ment de la liste.
  \item Remplacer le troisi�me �l�ment de la liste par le vecteur
    \verb|3:8|.
  \end{enumerate}
  \begin{rep}
    Soit \code{x} le nom de la liste.
    \begin{enumerate}
\item
\begin{Schunk}
\begin{Sinput}
> list(1:5, data = matrix(1:6, 2, 3), numeric(3), 
+     test = logical(4))
\end{Sinput}
\end{Schunk}
\item
\begin{Schunk}
\begin{Sinput}
> names(x)
\end{Sinput}
\end{Schunk}
\item
\begin{Schunk}
\begin{Sinput}
> mode(x$test)
> length(x$test)
\end{Sinput}
\end{Schunk}
\item
\begin{Schunk}
\begin{Sinput}
> dim(x$data)
\end{Sinput}
\end{Schunk}
\item
\begin{Schunk}
\begin{Sinput}
> x[[2]][c(2, 3)]
\end{Sinput}
\end{Schunk}
\item
\begin{Schunk}
\begin{Sinput}
> x[[3]] <- 3:8
\end{Sinput}
\end{Schunk}
    \end{enumerate}
  \end{rep}
\end{exercice}

\begin{exercice}
  \index{indi�age!vecteur}
  Soit \code{obs} un vecteur contenant les valeurs suivantes:
  \begin{center}
\begin{Schunk}
\begin{Sinput}
> obs
\end{Sinput}
\begin{Soutput}
 [1] 12 14 18  7 15 13 11 19  2  2 10 16 20 14 19 13
[17] 19  1  9  6
\end{Soutput}
\end{Schunk}
  \end{center}
  �crire une expression S permettant d'extraire les �l�ments suivants.
  \begin{enumerate}
  \item Le deuxi�me �l�ment de l'�chantillon.
  \item Les cinq premiers �l�ments de l'�chantillon.
  \item Les �l�ments strictement sup�rieurs � 14.
  \item Tous les �l�ments sauf les �l�ments en positions 6, 10 et 12.
  \end{enumerate}
  \begin{rep}
    \begin{enumerate}
\item
\begin{Schunk}
\begin{Sinput}
> obs[2]
\end{Sinput}
\end{Schunk}
\item
\begin{Schunk}
\begin{Sinput}
> obs[1:5]
\end{Sinput}
\end{Schunk}
\item
\begin{Schunk}
\begin{Sinput}
> obs[obs > 14]
\end{Sinput}
\end{Schunk}
\item
\begin{Schunk}
\begin{Sinput}
> obs[-c(6, 10, 12)]
\end{Sinput}
\end{Schunk}
    \end{enumerate}
  \end{rep}
\end{exercice}

\begin{exercice}
  \index{indi�age!matrice}
  Soit \code{mat} une matrice $10 \times 7$ obtenue al�atoirement avec
\begin{Schunk}
\begin{Sinput}
> (mat <- matrix(sample(1:100, 70), 7, 10))
\end{Sinput}
\end{Schunk}
  �crire une expression S permettant d'obtenir les �l�ments demand�s
  ci-dessous.
  \begin{enumerate}
  \item L'�l�ment $(4, 3)$ de la matrice.
  \item Le contenu de la sixi�me ligne de la matrice.
  \item Les premi�re et quatri�me colonnes de la matrice
    (simultan�ment).
  \item Les lignes de la matrice dont le premier �l�ment est
    sup�rieur � 50.
  \end{enumerate}
  \begin{rep}
    \begin{enumerate}
\item
\begin{Schunk}
\begin{Sinput}
> mat[4, 3]
\end{Sinput}
\end{Schunk}
\item
\begin{Schunk}
\begin{Sinput}
> mat[6, ]
\end{Sinput}
\end{Schunk}
\item
\begin{Schunk}
\begin{Sinput}
> mat[, c(1, 4)]
\end{Sinput}
\end{Schunk}
\item
\begin{Schunk}
\begin{Sinput}
> mat[mat[, 1] > 50, ]
\end{Sinput}
\end{Schunk}
    \end{enumerate}
  \end{rep}
\end{exercice}

\Closesolutionfile{reponses}

%%% Local Variables:
%%% mode: latex
%%% TeX-master: "introduction_programmation_S"
%%% End:

\chapter{Op�rateurs et fonctions}
\label{operateurs}


Ce chapitre pr�sente les principaux op�rateurs arithm�tiques,
fonctions math�matiques et structures de contr�le offerts par le S.
La liste est �videmment loin d'�tre exhaustive, surtout �tant donn�
l'�volution rapide du langage. Un des meilleurs endroits pour
d�couvrir de nouvelles fonctions demeure la section \texttt{See Also}
des rubriques d'aide, qui offre des hyperliens vers des fonctions
apparent�es au sujet de la rubrique.


\section{Op�rations arithm�tiques}
\label{operateurs:operations}

L'unit� de base en S est le vecteur\index{vecteur}.

\begin{itemize}
\item Les op�rations sur les vecteurs sont effectu�es \emph{�l�ment
    par �l�ment}:
\begin{Schunk}
\begin{Sinput}
> c(1, 2, 3) + c(4, 5, 6)
\end{Sinput}
\begin{Soutput}
[1] 5 7 9
\end{Soutput}
\begin{Sinput}
> 1:3 * 4:6
\end{Sinput}
\begin{Soutput}
[1]  4 10 18
\end{Soutput}
\end{Schunk}
\item Si les vecteurs impliqu�s dans une expression arithm�tique ne
  sont pas de la m�me longueur, les plus courts sont \emph{recycl�s}
  de fa�on � correspondre au plus long vecteur.  Cette r�gle est
  particuli�rement apparente avec les vecteurs de longueur 1:
\begin{Schunk}
\begin{Sinput}
> 1:10 + 2
\end{Sinput}
\begin{Soutput}
 [1]  3  4  5  6  7  8  9 10 11 12
\end{Soutput}
\begin{Sinput}
> 1:10 + rep(2, 10)
\end{Sinput}
\begin{Soutput}
 [1]  3  4  5  6  7  8  9 10 11 12
\end{Soutput}
\end{Schunk}
\item Si la longueur du plus long vecteur est un multiple de celle du
  ou des autres vecteurs, ces derniers sont recycl�s un nombre entier
  de fois:
\begin{Schunk}
\begin{Sinput}
> 1:10 + 1:5 + c(2, 4)
\end{Sinput}
\begin{Soutput}
 [1]  4  8  8 12 12 11 11 15 15 19
\end{Soutput}
\begin{Sinput}
> 1:10 + rep(1:5, 2) + rep(c(2, 4), 5)
\end{Sinput}
\begin{Soutput}
 [1]  4  8  8 12 12 11 11 15 15 19
\end{Soutput}
\end{Schunk}
\item Sinon, le plus court vecteur est recycl� un nombre fractionnaire
  de fois, mais comme ce r�sultat est rarement souhait� et provient
  g�n�ralement d'une erreur de programmation, un avertissement est
  affich�:
\begin{Schunk}
\begin{Sinput}
> 1:10 + c(2, 4, 6)
\end{Sinput}
\begin{Soutput}
 [1]  3  6  9  6  9 12  9 12 15 12
Message d'avis :
la longueur de l'objet le plus long n'est pas un
multiple de la longueur de l'objet le plus court in:
1:10 + c(2, 4, 6)
\end{Soutput}
\end{Schunk}
\end{itemize}


\section{Op�rateurs}
\label{operateurs:operateurs}

On trouvera dans le tableau \ref{tab:operateurs} les op�rateurs
math�matiques et logiques les plus fr�quemment employ�s, en ordre
d�croissant de priorit� des op�rations. Le tableau 3.1 de \citet{MASS}
contient une liste plus compl�te.

\begin{table}
  \centering
  \renewcommand{\arraystretch}{1.1}
  \begin{tabular}{lp{7cm}}
    \toprule
    \Fonction{\^} ou \Fonction{**} & puissance \\
    \Fonction{-} & changement de signe \\
    \Fonction{*} \Fonction{/} & multiplication, division \\
    \Fonction{+} \Fonction{-} & addition, soustraction \\
    \Fonction{\%*\%} \Fonction{\%\%} \Fonction{\%/\%} & produit
    matriciel, modulo, division enti�re \\
    \Fonction{<} \Fonction{<=} \Fonction{==} \Fonction{>=}
    \Fonction{>} \verb|!=|\Index{"!=@\code{"!=}} & plus petit, plus petit ou �gal, �gal,
    plus grand ou �gal, plus grand, diff�rent de \\
    \verb|!|\Index{"!@\code{"!}} & n�gation logique \\
    \Fonction{\&} \Fonction{|} & �et� logique, �ou� logique \\
    \bottomrule
  \end{tabular}
  \caption{Principaux op�rateurs math�matiques, en ordre d�croissant
    de priorit� des op�rations}
  \label{tab:operateurs}
\end{table}


\section{Appels de fonctions}
\index{fonction!appel}
\label{operateurs:appelfonctions}

Il existe certaines r�gles quant � la fa�on de sp�cifier les arguments
d'une fonction interne ou personnelle.

\begin{itemize}
\item Il n'y a pas de limite pratique quant au nombre d'arguments que
  peut avoir une fonction.
\item Les arguments d'une fonction peuvent �tre sp�cifi�s selon
  l'ordre �tabli dans la d�finition de la fonction.
\item Cependant, il est beaucoup plus prudent et \emph{fortement
    recommand�} de sp�cifier les arguments par leur nom, surtout apr�s
  les deux ou trois premiers arguments.
\item L'ordre des arguments est important; il est donc n�cessaire de
  les nommer s'ils ne sont pas appel�s dans l'ordre.
\item Certains arguments ont une valeur par d�faut qui sera utilis�e
  si l'argument n'est pas sp�cifi� dans l'appel de la fonction.
\end{itemize}

Par exemple, la d�finition de la fonction \texttt{matrix} est la
suivante:
\begin{verbatim}
   matrix(data=NA, nrow=1, ncol=1, byrow=FALSE,
          dimnames=NULL)
\end{verbatim}
\begin{itemize}
  \sloppy
\item La fonction compte cinq arguments: \argument{data},
  \argument{nrow}, \argument{ncol}, \argument{byrow} et
  \argument{dimnames}.
\item Ici, chaque argument a une valeur par d�faut (ce n'est pas
  toujours le cas). Ainsi, un appel � \code{matrix} sans
  argument r�sulte en une matrice $1 \times 1$ remplie par colonne
  (sans importance, ici) de la �valeur� \code{NA} et dont les
  dimensions sont d�pourvues d'�tiquettes.
\begin{Schunk}
\begin{Sinput}
> matrix()
\end{Sinput}
\begin{Soutput}
     [,1]
[1,]   NA
\end{Soutput}
\end{Schunk}
\item Appel plus �labor� utilisant tous les arguments. Le premier
  argument est rarement nomm�.
\begin{Schunk}
\begin{Sinput}
> matrix(1:6, nrow = 2, ncol = 3, byrow = TRUE, 
+     dimnames = list(c("Gauche", "Droit"), 
+         c("Rouge", "Vert", "Bleu")))
\end{Sinput}
\begin{Soutput}
       Rouge Vert Bleu
Gauche     1    2    3
Droit      4    5    6
\end{Soutput}
\end{Schunk}
\end{itemize}

La section 3.6 de \citet{MASS} contient de plus amples d�tails.


\section{Quelques fonctions utiles}
\label{operateurs:fonctionsutiles}

Le langage S compte un tr�s grand nombre de fonctions internes. La
terminologie du syst�me de classement de ces fonctions et la fa�on de
les charger en m�moire diff�rent quelque peu selon que l'on utilise
S-Plus ou \textsf{R}.

Dans S-Plus, \Splus les fonctions sont class�es dans des
\emph{sections} d'une biblioth�que\index{biblioth�que}
(\emph{library}). La biblioth�que principale se trouve dans le dossier
\texttt{library} du dossier d'installation de S-Plus.  Au d�marrage,
plusieurs sections de la biblioth�que de base (dont, entre autres,
\texttt{main}, \texttt{splus} et \texttt{stat}) sont imm�diatement
charg�es en m�moire, avec comme cons�quence qu'un tr�s grand nombre de
fonctions sont imm�diatement disponibles.

Dans \textsf{R}, \R un ensemble de fonctions est appel� un
package\index{package} (terme non traduit). Par d�faut, \textsf{R}
charge en m�moire quelques packages de la biblioth�que seulement, ce
qui �conomise l'espace m�moire et acc�l�re le d�marrage.  En revanche,
on a plus souvent recours � la fonction \texttt{library} pour charger
de nouveaux packages.

Nous utiliserons dor�navant la terminologie de \textsf{R} pour
d�signer un �l�ment de la biblioth�que.

Cette section pr�sente quelques-unes seulement des nombreuses
fonctions disponibles dans S-Plus et \textsf{R}. On s'y concentre sur
les fonctions de base les plus souvent utilis�es pour programmer en S
et pour manipuler des donn�es.

\subsection{Manipulation de vecteurs}

\begin{ttscript}{unique}
\item[\Fonction{seq}] g�n�ration de suites de nombres\index{suite de nombres}
\item[\Fonction{rep}] r�p�tition\index{repetition@r�p�tition de valeurs} de
  valeurs ou de vecteurs
\item[\Fonction{sort}] tri\index{tri} en ordre croissant ou
  d�croissant
\item[\Fonction{order}] positions dans un vecteur des valeurs en ordre
  croissant ou d�croissant
\item[\Fonction{rank}] rang\index{rang} des �l�ments d'un vecteur en
  ordre croissant ou d�croissant
\item[\Fonction{rev}] renverser\index{renverser un vecteur} un vecteur
\item[\Fonction{head}] extraction\index{extraction!premi�res valeurs}
  des $n$ premi�res valeurs (\textsf{R} seulement)
  \index{extraction|seealso{indi�age}}
\item[\Fonction{tail}] extraction\index{extraction!derni�res valeurs}
  des $n$ derni�res valeurs (\textsf{R} seulement)
\item[\Fonction{unique}] extraction des �l�ments
  diff�rents\index{extraction!elements diff�rents@�l�ments diff�rents}
  d'un vecteur
\end{ttscript}

\subsection{Recherche d'�l�ments dans un vecteur}

\begin{ttscript}{which.max}
\item[\Fonction{which}] positions des valeurs \texttt{TRUE} dans un vecteur
  bool�en
\item[\Fonction{which.min}] position du minimum\index{minimum!position
    dans un vecteur} dans un vecteur
\item[\Fonction{which.max}] position du maximum\index{maximum!position
    dans un vecteur} dans un vecteur
\item[\Fonction{match}] position de la premi�re occurrence d'un �l�ment dans un
  vecteur
\item[\Fonction{\%in\%}] appartenance d'une ou plusieurs valeurs � un vecteur
\end{ttscript}

\subsection{Arrondi}

\begin{ttscript}{ceiling}
\item[\Fonction{round}] arrondi\index{arrondi} � un nombre d�fini de
  d�cimales
\item[\Fonction{floor}] plus grand entier inf�rieur ou �gal � l'argument
\item[\Fonction{ceiling}] plus petit entier sup�rieur ou �gal � l'argument
\item[\Fonction{trunc}] troncature vers z�ro de l'argument; diff�rent de
  \texttt{floor} pour les nombres n�gatifs
\end{ttscript}

\subsection{Sommaires et statistiques descriptives}

\begin{ttscript}{sum, prod}
\item[\Fonction{sum}, \Fonction{prod}] somme\index{somme} et
  produit\index{produit} des �l�ments d'un vecteur
\item[\Fonction{diff}] diff�rences\index{diff�rences} entre les
  �l�ments d'un vecteur
\item[\Fonction{mean}] moyenne
  arithm�tique\index{moyenne!arithm�tique} et moyenne
  tronqu�e\index{moyenne!tronqu�e}
\item[\Fonction{var}, \Fonction{sd}] variance\index{variance} et �cart
  type\index{ecart type@�cart type} (versions sans biais)
\item[\Fonction{min}, \Fonction{max}] minimum\index{minimum!d'un
    vecteur} et maximum\index{maximum!d'un vecteur} d'un vecteur
\item[\Fonction{range}] vecteur contenant le minimum et le maximum
  d'un vecteur
\item[\Fonction{median}] m�diane\index{mediane@m�diane} empirique
\item[\Fonction{quantile}] quantiles\index{quantile} empiriques
\item[\Fonction{summary}] statistiques descriptives d'un �chantillon
\end{ttscript}

\subsection{Sommaires cumulatifs et comparaisons �l�ment par �l�ment}

\begin{ttscript}{cumsum, cumprod}
\item[\Fonction{cumsum}, \Fonction{cumprod}]
  somme\index{somme!cumulative} et produit\index{produit!cumulatif}
  cumulatif d'un vecteur
\item[\Fonction{cummin}, \Fonction{cummax}]
  minimum\index{minimum!cumulatif} et maximum\index{maximum!cumulatif}
  cumulatif
\item[\Fonction{pmin}, \Fonction{pmax}]
  minimum\index{minimum!parall�le} et maximum\index{maximum!parall�le}
  en parall�le, c'est-�-dire �l�ment par �l�ment entre deux vecteurs
  ou plus
\end{ttscript}

\subsection{Op�rations sur les matrices}

\begin{ttscript}{rowMeans, colMeans}
\item[\Fonction{t}] transpos�e\index{matrice!transpos�e}
\item[\Fonction{solve}] avec un seul argument (une matrice carr�e):
  inverse\index{matrice!inverse} d'une matrice; avec deux arguments
  (une matrice carr�e et un vecteur): solution du syst�me d'�quation
  $\mat{A} \mat{x} = \mat{b}$
\item[\Fonction{diag}] avec une matrice en argument: diagonale de la
  matrice; avec un vecteur en argument: matrice
  diagonale\index{matrice!diagonale} form�e avec le vecteur; avec un
  scalaire $p$ en argument: matrice identit�\index{matrice!identit�}
  $p \times p$
\item[\Fonction{nrow}, \Fonction{ncol}] nombre de lignes et de
  colonnes d'une matrice
\item[\Fonction{rowSums}, \Fonction{colSums}]
  sommes\index{matrice!sommes par ligne} par ligne et par
  colonne\index{matrice!somme par colonne}, respectivement, des
  �l�ments d'une matrice; voir aussi la fonction \texttt{apply} � la
  section \ref{avance:apply}
\item[\Fonction{rowMeans}, \Fonction{colMeans}]
  moyennes\index{matrice!moyennes par ligne} par ligne et par
  colonne\index{matrice!moyennes par colonne}, respectivement, des
  �l�ments d'une matrice; voir aussi la fonction \texttt{apply} � la
  section \ref{avance:apply}
\item[\Fonction{rowVars}, \Fonction{colVars}]
  variance\index{matrice!variance par ligne} par ligne et par
  colonne\index{matrice!variance par colonne} des �l�ments d'une
  matrice (S-Plus seulement)
\end{ttscript}

\subsection{Produit ext�rieur}
\index{produit!ext�rieur}

La fonction \Fonction{outer}, dont la syntaxe est
\begin{center}
  \code{outer(X, Y, FUN)},
\end{center}
applique la fonction \code{FUN} (\fonction{prod} par d�faut) entre
chacun des �l�ments de \code{X} et chacun des �l�ments de \code{Y}.
\begin{itemize}
\item La dimension du r�sultat est par cons�quent \code{c(dim(X),
    dim(Y))}.
\item Par exemple, le r�sultat du produit ext�rieur entre
  deux vecteurs est une matrice contenant tous les produits entre les
  �l�ments des deux vecteurs:
\begin{Schunk}
\begin{Sinput}
> outer(c(1, 2, 5), c(2, 3, 6))
\end{Sinput}
\begin{Soutput}
     [,1] [,2] [,3]
[1,]    2    3    6
[2,]    4    6   12
[3,]   10   15   30
\end{Soutput}
\end{Schunk}
\item L'op�rateur \Fonction{\%o\%} est un raccourci de \code{outer(X,
    Y, prod)}.
\end{itemize}


\section{Structures de contr�le}
\label{operateurs:structures}

On se contente, ici, de mentionner les structures de contr�le
disponibles en S. Se reporter � \citet[section 3.8]{MASS} pour plus de
d�tails sur leur utilisation.

\subsection{Ex�cution conditionnelle}

\begin{struclist}
\item[\fbox{if (\emph{condition}) \emph{branche.vrai} else
    \emph{branche.faux}}] \rule{0em}{2.5ex}%
  \Indexfonction{if}%
  \Indexfonction{else}%
  \sloppy Si \code{\emph{condition}} est vraie,
  \code{\emph{branche.vrai}} est ex�cut�e, sinon ce sera
  \code{\emph{branche.faux}}. Dans le cas o� l'une ou l'autre de
  \code{\emph{branche.vrai}} ou \code{\emph{branche.faux}} comporte
  plus d'une expression, grouper celles-ci dans des accolades
  \verb={ }=.
\item[\fbox{ifelse(\emph{condition}, \emph{expression.vrai},
    \emph{expression.faux})}]
  \rule{0em}{2.5ex}%
  \Indexfonction{ifelse}%
  Fonction vectoris�e qui remplace chaque �l�ment \code{TRUE} du
  vecteur \code{\emph{condition}} par l'�l�ment correspondant de
  \code{\emph{expression.vrai}} et chaque �l�ment \code{FALSE} par
  l'�l�ment correspondant de \code{\emph{expression.faux}}.
  L'utilisation n'est pas tr�s intuitive, alors examiner attentivement
  les exemples de la rubrique d'aide.
\item[\fbox{switch(\emph{test}, \emph{cas.1 = action.1}, \emph{cas.2 =
      action.2}, ...)}]
  \rule{0em}{2.5ex}%
  \Indexfonction{switch}%
  Structure utilis�e plut�t rarement.
\end{struclist}

\subsection{Boucles}

Les boucles\index{boucle} sont et doivent �tre utilis�es avec
parcimonie en S, car elles sont g�n�ralement inefficaces
(particuli�rement avec S-Plus).  Dans la majeure partie des cas, il
est possible de vectoriser les calculs pour �viter les boucles
explicites, ou encore de s'en remettre aux fonctions \fonction{apply},
\fonction{lapply} et \fonction{sapply} (section \ref{avance:apply})
pour faire les boucles de mani�re plus efficace.

\begin{struclist}
\item[\fbox{for (\emph{variable} in \emph{suite}) \emph{expression}}]
  \rule{0em}{2.5ex}%
  \Indexfonction{for}%
  Ex�cuter \code{\emph{expression}} successivement pour chaque valeur
  de \code{\emph{variable}} contenue dans \code{\emph{suite}}.  Encore
  ici, on groupera les expressions dans des accolades \verb={ }=. �
  noter que \code{\emph{suite}} n'a pas � �tre compos�e de nombres
  cons�cutifs, ni m�me de nombres, en fait.
\item[\fbox{while (\emph{condition}) \emph{expression}}]
  \rule{0em}{2.5ex}%
  \Indexfonction{while}%
  Ex�cuter \code{\emph{expression}} tant que \code{\emph{condition}}
  est vraie. Si \code{\emph{condition}} est fausse lors de l'entr�e
  dans la boucle, celle-ci n'est pas ex�cut�e. Une boucle \code{while}
  n'est par cons�quent pas n�cessairement toujours ex�cut�e.
\item[\fbox{repeat \emph{expression}}]
  \rule{0em}{2.5ex}%
  \Indexfonction{repeat}%
  R�p�ter \code{\emph{expression}}. Cette derni�re devra comporter un
  test d'arr�t qui utilisera la commande \code{break}. Une boucle
  \code{repeat} est toujours ex�cut�e au moins une fois.
\item[\fbox{break}]
  \rule{0em}{2.5ex}%
  \Indexfonction{break}%
  Sortie imm�diate d'une boucle \code{for}, \code{while} ou
  \code{repeat}.
\item[\fbox{next}]
  \rule{0em}{2.5ex}%
  \Indexfonction{next}%
  Passage imm�diat � la prochaine it�ration d'une boucle \code{for},
  \code{while} ou \code{repeat}.
\end{struclist}


\section{Exemples}
\label{operateurs:exemples}

\lstinputlisting{operateurs.R}


\section{Exercices}
\label{operateurs:exercices}

\Opensolutionfile{reponses}[reponses-operateurs]
\Writetofile{reponses}{\protect\section*{Chapitre \protect\ref{operateurs}}}


\begin{exercice}
  � l'aide des fonctions \fonction{rep}, \fonction{seq} et
  \code{c} seulement, g�n�rer les s�quences suivantes.
  \begin{enumerate}
  \item
\begin{Schunk}
\begin{Soutput}
0 6 0 6 0 6
\end{Soutput}
\end{Schunk}
  \item
\begin{Schunk}
\begin{Soutput}
1 4 7 10
\end{Soutput}
\end{Schunk}
  \item
\begin{Schunk}
\begin{Soutput}
1 2 3 1 2 3 1 2 3 1 2 3
\end{Soutput}
\end{Schunk}
  \item
\begin{Schunk}
\begin{Soutput}
1 2 2 3 3 3
\end{Soutput}
\end{Schunk}
  \item
\begin{Schunk}
\begin{Soutput}
1 1 1 2 2 3
\end{Soutput}
\end{Schunk}
  \item
\begin{Schunk}
\begin{Soutput}
1 5.5 10
\end{Soutput}
\end{Schunk}
  \item
\begin{Schunk}
\begin{Soutput}
1 1 1 1 2 2 2 2 3 3 3 3
\end{Soutput}
\end{Schunk}
  \end{enumerate}

  \begin{rep}
    \begin{enumerate}
\item
\begin{Schunk}
\begin{Sinput}
> rep(c(0, 6), 3)
\end{Sinput}
\end{Schunk}
\item
\begin{Schunk}
\begin{Sinput}
> seq(1, 10, by = 3)
\end{Sinput}
\end{Schunk}
\item
\begin{Schunk}
\begin{Sinput}
> rep(1:3, 4)
\end{Sinput}
\end{Schunk}
\item
\begin{Schunk}
\begin{Sinput}
> rep(1:3, 1:3)
\end{Sinput}
\end{Schunk}
\item
\begin{Schunk}
\begin{Sinput}
> rep(1:3, 3:1)
\end{Sinput}
\end{Schunk}
\item
\begin{Schunk}
\begin{Sinput}
> seq(1, 10, length = 3)
\end{Sinput}
\end{Schunk}
\item
\begin{Schunk}
\begin{Sinput}
> rep(1:3, rep(4, 3))
\end{Sinput}
\end{Schunk}
    \end{enumerate}
  \end{rep}
\end{exercice}


\begin{exercice}
  G�n�rer les suites de nombres suivantes � l'aide des fonctions
  \verb|:|\index{:@\verb|:|} et \texttt{rep} seulement, donc sans
  utiliser la fonction \fonction{seq}.
  \begin{enumerate}
  \item
\begin{Schunk}
\begin{Soutput}
1.1 1.2 1.3 1.4 1.5 1.6 1.7 1.8 1.9 2
\end{Soutput}
\end{Schunk}
  \item
\begin{Schunk}
\begin{Soutput}
1 3 5 7 9 11 13 15 17 19
\end{Soutput}
\end{Schunk}
  \item
\begin{Schunk}
\begin{Soutput}
-2 -1 0 1 2 -2 -1 0 1 2
\end{Soutput}
\end{Schunk}
  \item
\begin{Schunk}
\begin{Soutput}
-2 -2 -1 -1 0 0 1 1 2 2
\end{Soutput}
\end{Schunk}
  \item
\begin{Schunk}
\begin{Soutput}
10 20 30 40 50 60 70 80 90 100
\end{Soutput}
\end{Schunk}
  \end{enumerate}

  \begin{rep}
    \begin{enumerate}
\item
\begin{Schunk}
\begin{Sinput}
> 11:20/10
\end{Sinput}
\end{Schunk}
\item
\begin{Schunk}
\begin{Sinput}
> 2 * 0:9 + 1
\end{Sinput}
\end{Schunk}
\item
\begin{Schunk}
\begin{Sinput}
> rep(-2:2, 2)
\end{Sinput}
\end{Schunk}
\item
\begin{Schunk}
\begin{Sinput}
> rep(-2:2, each = 2)
\end{Sinput}
\end{Schunk}
\item
\begin{Schunk}
\begin{Sinput}
> 10 * 1:10
\end{Sinput}
\end{Schunk}
    \end{enumerate}
  \end{rep}
\end{exercice}

\begin{exercice}
  � l'aide de la commande \fonction{apply}, �crire des expressions S
  qui remplaceraient les fonctions suivantes.
  \begin{enumerate}
  \item \fonction{rowSums}
  \item \fonction{colSums}
  \item \fonction{rowMeans}
  \item \fonction{colMeans}
  \end{enumerate}
  \begin{rep}
    Soit \code{mat} une matrice.
    \begin{enumerate}
\item
\begin{Schunk}
\begin{Sinput}
> apply(mat, 1, sum)
\end{Sinput}
\end{Schunk}
\item
\begin{Schunk}
\begin{Sinput}
> apply(mat, 2, sum)
\end{Sinput}
\end{Schunk}
\item
\begin{Schunk}
\begin{Sinput}
> apply(mat, 1, mean)
\end{Sinput}
\end{Schunk}
\item
\begin{Schunk}
\begin{Sinput}
> apply(mat, 2, mean)
\end{Sinput}
\end{Schunk}
    \end{enumerate}
  \end{rep}
\end{exercice}

\begin{exercice}
  Sans utiliser les fonctions \fonction{factorial},
  \fonction{lfactorial}, \fonction{gamma} ou \fonction{lgamma},
  g�n�rer la s�quence 1!, 2!, ..., 10!
  \begin{rep}
\begin{Schunk}
\begin{Sinput}
> cumprod(1:10)
\end{Sinput}
\end{Schunk}
  \end{rep}
\end{exercice}

\begin{exercice}
  Trouver une relation entre \code{x}, \code{y}, \code{x \%\% y}
  et \code{x \%/\% y}, o� \code{y != 0}.
  \begin{rep}
    \verb|x == (x %% y) + y * ( x %/% y )|
  \end{rep}
\end{exercice}

\enlargethispage{10mm}
\begin{exercice}
  Simuler un �chantillon $\mat{x} = (x_1, x_2, x_3, ..., x_{20})$ avec
  la fonction \fonction{sample}.  �crire une expression S permettant
  d'obtenir ou de calculer chacun des r�sultats demand�s ci-dessous.
  \begin{enumerate}
  \item Les cinq premiers �l�ments de l'�chantillon.
  \item La valeur maximale de l'�chantillon.
  \item La moyenne des cinq premiers �l�ments de l'�chantillon.
  \item La moyenne des cinq derniers �l�ments de l'�chantillon.
  \end{enumerate}
  \begin{rep}
    \begin{enumerate}
\item
\begin{Schunk}
\begin{Sinput}
> x[1:5]
> head(x, 5)
\end{Sinput}
\end{Schunk}
\item
\begin{Schunk}
\begin{Sinput}
> max(x)
\end{Sinput}
\end{Schunk}
\item
\begin{Schunk}
\begin{Sinput}
> mean(x[1:5])
> mean(head(x, 5))
\end{Sinput}
\end{Schunk}
\item
\begin{Schunk}
\begin{Sinput}
> mean(x[16:20])
> mean(x[(length(x) - 4):length(x)])
> mean(tail(x, 5))
\end{Sinput}
\end{Schunk}
    \end{enumerate}
  \end{rep}
\end{exercice}

\begin{exercice}
  \label{exercice:operateurs:ijk}
  \begin{enumerate}
  \item Trouver une formule pour calculer la position, dans le vecteur
    sous-jacent, de l'�l�ment $(i, j)$ d'une matrice\index{matrice} $I
    \times J$ remplie par colonne.
  \item R�p�ter la partie (a) pour l'�l�ment $(i, j, k)$ d'un
    tableau\index{tableau} $I \times J \times K$.
  \end{enumerate}
  \begin{rep}
    \begin{enumerate}
    \item \verb|(j - 1)*I + i|
    \item \verb|((k - 1)*J + j - 1)*I + i|
    \end{enumerate}
  \end{rep}
\end{exercice}

\begin{exercice}
  Simuler une matrice\index{matrice} \code{mat} $10 \times 7$, puis
  �crire des expressions S permettant d'effectuer les t�ches demand�es
  ci-dessous.
  \begin{enumerate}
  \item Calculer la somme des �l�ments de chacunes des lignes de la
    matrice.
  \item Calculer la moyenne des �l�ments de chacunes des colonnes de
    la matrice.
  \item Calculer la valeur maximale de la sous-matrice form�e par les
    trois premi�res lignes et les trois premi�res colonnes.
  \item Extraire toutes les lignes de la matrice dont la moyenne des
    �l�ments est sup�rieure � 7.
  \end{enumerate}
  \begin{rep}
    \begin{enumerate}
\item
\begin{Schunk}
\begin{Sinput}
> rowSums(mat)
\end{Sinput}
\end{Schunk}
\item
\begin{Schunk}
\begin{Sinput}
> colMeans(mat)
\end{Sinput}
\end{Schunk}
\item
\begin{Schunk}
\begin{Sinput}
> max(mat[1:3, 1:3])
\end{Sinput}
\end{Schunk}
\item
\begin{Schunk}
\begin{Sinput}
> mat[rowMeans(mat) > 7, ]
\end{Sinput}
\end{Schunk}
    \end{enumerate}
  \end{rep}
\end{exercice}

\begin{exercice}
  On vous donne la liste et la date des 31 meilleurs temps enregistr�s
  au 100~m�tres homme entre 1964 et 2005:
\begin{Schunk}
\begin{Sinput}
> temps <- c(10.06, 10.03, 10.02, 9.95, 10.04, 
+     10.07, 10.08, 10.05, 9.98, 10.09, 10.01, 
+     10, 9.97, 9.93, 9.96, 9.99, 9.92, 9.94, 
+     9.9, 9.86, 9.88, 9.87, 9.85, 9.91, 9.84, 
+     9.89, 9.79, 9.8, 9.82, 9.78, 9.77)
> names(temps) <- c("1964-10-15", "1968-06-20", 
+     "1968-10-13", "1968-10-14", "1968-10-14", 
+     "1968-10-14", "1968-10-14", "1975-08-20", 
+     "1977-08-11", "1978-07-30", "1979-09-04", 
+     "1981-05-16", "1983-05-14", "1983-07-03", 
+     "1984-05-05", "1984-05-06", "1988-09-24", 
+     "1989-06-16", "1991-06-14", "1991-08-25", 
+     "1991-08-25", "1993-08-15", "1994-07-06", 
+     "1994-08-23", "1996-07-27", "1996-07-27", 
+     "1999-06-16", "1999-08-22", "2001-08-05", 
+     "2002-09-14", "2005-06-14")
\end{Sinput}
\end{Schunk}
  Extraire de ce vecteur les records du monde seulement, c'est-�-dire
  la premi�re fois que chaque temps a �t� r�alis�.
  \begin{rep}
\begin{Schunk}
\begin{Sinput}
> temps[match(unique(cummin(temps)), temps)]
\end{Sinput}
\end{Schunk}
  \end{rep}
\end{exercice}

\Closesolutionfile{reponses}

%%% Local Variables:
%%% mode: latex
%%% TeX-master: "introduction_programmation_S"
%%% End:

\section{Exemples r�solus}

\subsection{Calcul de valeurs pr�sentes}

\begin{frame}
  \frametitle{Calcul de valeurs pr�sentes}

  \begin{block}{�nonc�}
    Un pr�t est rembours� par une s�rie de cinq paiements, le premier
    dans un an. Trouver le montant du pr�t pour\dots
    \begin{enumerate}[a)]
    \item Paiement annuel de \nombre{1000}, taux d'int�r�t de 6~\%
      effectif annuellement
    \item Paiements annuels de 500, 800, 900, 750 et \nombre{1000},
      taux d'int�r�t de 6~\% effectif annuellement
    \item Paiements annuels de 500, 800, 900, 750 et \nombre{1000},
      taux d'int�r�t de 5~\%, 6~\%, 5,5~\%, 6,5~\% et 7~\% effectifs
      annuellement
    \end{enumerate}
  \end{block}
\end{frame}

\begin{frame}
  \frametitle{Solution}
  Formule g�n�rale pour la valeur pr�sente d'une s�rie de paiements
  $P_1, P_2, \dots, P_n$ � la fin des ann�es $1, 2, \dots, n$:
  \begin{displaymath}
    \sum_{j=1}^n \prod_{k=1}^j (1 + i_k)^{-1} P_j
  \end{displaymath}
\end{frame}

\begin{frame}[fragile]
  \NoAutoSpaceBeforeFDP
  \begin{proof}[a) Un seul paiement annuel, un seul taux d'int�r�t]
    Cas sp�cial
    \begin{displaymath}
      \color{orange}P \color{blue} \sum_{j=1}^n \color{red}(1 + i)^{-j}
    \end{displaymath}
    En S:
    \pause
\begin{semiverbatim} \slshape
> \uncover<4->{\color{orange}1000 * }\uncover<3->{\color{blue}sum(}{\color{red}(1 + 0.06)^(-(1:5))}\uncover<3->{)}
\end{semiverbatim}
    \onslide<5->
\begin{Soutput}
[1] 4212.364
\end{Soutput}
  \end{proof}
\end{frame}

\begin{frame}[fragile]
  \NoAutoSpaceBeforeFDP
  \begin{proof}[b) Diff�rents paiements annuels, un seul taux d'int�r�t]
    On a, cette fois,
    \begin{displaymath}
      \color{blue} \sum_{j=1}^n \color{red}(1 + i)^{-j} \color{orange}P_j
    \end{displaymath}
    En S:
    \pause
\begin{semiverbatim}
> \uncover<3->{\color{blue}sum(}\color{orange}c(500, 800, 900, 750, 1000) *
\normalcolor+     \color{red}(1 + 0.06)^(-(1:5))\uncover<3->{\color{blue})}
\end{semiverbatim}
    \onslide<4->
\begin{Soutput}
[1] 3280.681
\end{Soutput}
  \end{proof}
\end{frame}

\begin{frame}[fragile]
  \begin{proof}[c) Diff�rents paiements annuels, diff�rents taux d'int�r�t]
    On doit utiliser la formule g�n�rale
    \begin{displaymath}
      \color{blue}\sum_{j=1}^n
      \color{red} \prod_{k=1}^j (1 + i_k)^{-1}
      \color{orange} P_j
    \end{displaymath}
    En S:
    \pause
\begin{semiverbatim}
> \uncover<4->{\color{blue}sum(}\uncover<3->{\color{orange}c(500, 800, 900, 750, 1000)/}
\normalcolor+ \color{red}cumprod(c(1.05, 1.06, 1.055, 1.065, 1.07))\uncover<4->{\color{blue})}
\end{semiverbatim}
    \onslide<5->
\begin{Soutput}
[1] 3308.521
\end{Soutput}
  \end{proof}
\end{frame}

\subsection{Fonctions de probabilit�}
\label{exemples:fp}

\begin{frame}
  \frametitle{Fonctions de probabilit�}

  \begin{block}{�nonc�}
    Calculer toutes ou la majeure partie des probabilit�s des deux
    lois de probabilit� ci-dessous. V�rifier que la somme des
    probabilit�s est bien �gale � 1.

    \begin{enumerate}[a)]
    \item Binomiale
      \begin{displaymath}
        f(x) = \binom{n}{x} p^x (1 - p)^{n - x}, \quad x = 0, \dots, n
      \end{displaymath}
    \item Poisson
      \begin{displaymath}
        f(x) = \frac{\lambda^x e^{-\lambda}}{x!}, \quad x = 0, 1, \dots,
      \end{displaymath}
      o� $x! = x(x - 1) \cdots 2 \cdot 1$
    \end{enumerate}
  \end{block}
\end{frame}

\begin{frame}[fragile]
  \frametitle{Solution}

  \begin{proof}[a) Binomiale$(10,\, 0,8)$]
\begin{Schunk}
\begin{Sinput}
> n <- 10
> p <- 0.8
> x <- 0:n
> choose(n, x) * p^x * (1 - p)^rev(x)
\end{Sinput}
\begin{Soutput}
 [1] 0.0000001024 0.0000040960 0.0000737280
 [4] 0.0007864320 0.0055050240 0.0264241152
 [7] 0.0880803840 0.2013265920 0.3019898880
[10] 0.2684354560 0.1073741824
\end{Soutput}
\end{Schunk}
    \pause
\begin{Schunk}
\begin{Sinput}
> sum(choose(n, x) * p^x * (1 - p)^rev(x))
\end{Sinput}
\begin{Soutput}
[1] 1
\end{Soutput}
\end{Schunk}
  \end{proof}
\end{frame}

\begin{frame}[fragile]
  \begin{proof}[b) Poisson(5)]
    On calcule les probabilit�s en $x = 0, 1, \dots, 10$ seulement.
\begin{Schunk}
\begin{Sinput}
> lambda <- 5
> x <- 0:10
> exp(-lambda) * (lambda^x/factorial(x))
\end{Sinput}
\begin{Soutput}
 [1] 0.006737947 0.033689735 0.084224337
 [4] 0.140373896 0.175467370 0.175467370
 [7] 0.146222808 0.104444863 0.065278039
[10] 0.036265577 0.018132789
\end{Soutput}
\end{Schunk}
  \pause
\begin{Schunk}
\begin{Sinput}
> x <- 0:200
> exp(-lambda) * sum((lambda^x/factorial(x)))
\end{Sinput}
\begin{Soutput}
[1] 1
\end{Soutput}
\end{Schunk}
  \end{proof}
\end{frame}


\subsection{Fonction de r�partition de la loi gamma}

\begin{frame}
  \frametitle{Fonction de r�partition de la loi gamma}

  Nous utilisons la param�trisation o� la fonction de densit� de
  probabilit� est
  \begin{displaymath}
    f(x) = \frac{\lambda^\alpha}{\Gamma(\alpha)}\, x^{\alpha-1}
    e^{-\lambda x}, \quad x > 0,
  \end{displaymath}
  o�
  \begin{displaymath}
    \Gamma(n) = \int_0^\infty x^{n - 1} e^{-x}\, dx
    = (n - 1) \Gamma(n - 1)
  \end{displaymath}
\end{frame}

\begin{frame}
  \begin{block}{�nonc�}
    Pour $\alpha$ entier et $\lambda = 1$ on a
    \begin{displaymath}
      F(x; \alpha, 1) = 1 - e^{-x} \sum_{j=0}^{\alpha-1} \frac{x^j}{j!}.
    \end{displaymath}

    \begin{enumerate}[a)]
    \item �valuer $F(4; 5, 1)$
    \item �valuer $F(x; 5, 1)$ pour $x = 2, 3, \dots, 10$ en une seule
      expression
    \end{enumerate}
  \end{block}
\end{frame}

\begin{frame}[fragile]
  \frametitle{Solution}

  \begin{proof}[a) Une seule valeur de $x$, param�tre $\alpha$ fixe]
\begin{Schunk}
\begin{Sinput}
> alpha <- 5
> x <- 4
> 1 - exp(-x) * sum(x^(0:(alpha - 1))/
+ gamma(1:alpha))
\end{Sinput}
\begin{Soutput}
[1] 0.3711631
\end{Soutput}
\end{Schunk}
  \pause
  V�rification avec la fonction interne \fonction{pgamma}:
\begin{Schunk}
\begin{Sinput}
> pgamma(x, alpha)
\end{Sinput}
\begin{Soutput}
[1] 0.3711631
\end{Soutput}
\end{Schunk}
  \end{proof}
\end{frame}

\begin{frame}[fragile]
  \begin{alertblock}{Astuce}
    On peut �viter de g�n�rer la m�me suite de nombres � deux reprises:
\begin{Schunk}
\begin{Sinput}
> 1 - exp(-x) * sum(x^(-1 + (j <- 1:alpha))/
+ gamma(j))
\end{Sinput}
\begin{Soutput}
[1] 0.3711631
\end{Soutput}
\end{Schunk}
  \end{alertblock}
\end{frame}

\begin{frame}[fragile]
  \begin{proof}[b) Plusieurs valeurs de $x$, param�tre $\alpha$ fixe]
    C'est un travail pour la fonction \fonction{outer}.
\begin{Schunk}
\begin{Sinput}
> x <- 2:10
> 1 - exp(-x) *
+ colSums(
+   t( outer(x, 0:(alpha - 1), "^") )
+   /gamma(1:alpha)
+ )
\end{Sinput}
\begin{Soutput}
[1] 0.05265302 0.18473676 0.37116306
[4] 0.55950671 0.71494350 0.82700839
[7] 0.90036760 0.94503636 0.97074731
\end{Soutput}
\end{Schunk}
  \end{proof}
\end{frame}


\subsection{Algorithme du point fixe}

\begin{frame}
  \frametitle{Algorithme du point fixe}

  \begin{itemize}
  \item Probl�me classique: trouver la racine d'une fonction $g$,
    c'est-�-dire le point $x$ o� $g(x) = 0$
  \item Souvent possible de reformuler le probl�me de fa�on � plut�t
    chercher le point $x$ o� $f(x) = x$
  \item Solution appel�e \alert{point fixe}
  \item L'algorithme du calcul num�rique du point fixe d'une fonction
    $f(x)$ est tr�s simple:
    \pause
    \begin{enumerate}
    \item choisir une valeur de d�part $x_0$
    \item calculer $x_n = f(x_{n - 1})$
    \item r�p�ter l'�tape 2 jusqu'� ce que $|x_n - x_{n-1}| < \varepsilon$
      ou $|x_n - x_{n-1}|/|{x_{n-1}}| < \varepsilon$
    \end{enumerate}
  \end{itemize}
\end{frame}

\begin{frame}
  \begin{block}{�nonc�}
    Trouver, � l'aide de la m�thode du point fixe, la valeur de $i$
    telle que
    \begin{displaymath}
      a_{\angl{10}} = \frac{1 - (1 + i)^{-10}}{i} = 8,21
    \end{displaymath}
  \end{block}
\end{frame}

\begin{frame}
  \frametitle{Solution}

  \begin{block}{Quelques consid�rations}
    \begin{itemize}
    \item On doit r�soudre
      \begin{displaymath}
        \frac{1 - (1 + i)^{-10}}{8,21} = i
      \end{displaymath}
    \item Nous ignorons combien de fois la proc�dure it�rative devra
      �tre r�p�t�e
    \item Il faut ex�cuter la proc�dure au moins une fois
    \item La structure de contr�le � utiliser dans cette proc�dure
      it�rative est donc \fonction{\alert{repeat}}
    \end{itemize}
  \end{block}
\end{frame}

\begin{frame}[fragile]
  \begin{proof}[Le code]
\begin{Schunk}
\begin{Sinput}
> i <- 0.05
> repeat {
+     it <- i
+     i <- (1 - (1 + it)^(-10))/8.21
+     if (abs(i - it)/it < 1e-10)
+         break
+ }
> i
\end{Sinput}
\begin{Soutput}
[1] 0.03756777
\end{Soutput}
\end{Schunk}
  \pause
\begin{Schunk}
\begin{Sinput}
> (1 - (1 + i)^(-10))/i
\end{Sinput}
\begin{Soutput}
[1] 8.21
\end{Soutput}
\end{Schunk}
  \end{proof}
\end{frame}
%%% Local Variables:
%%% mode: latex
%%% TeX-master: "introduction_programmation_S_diapos"
%%% End:

\chapter{Fonctions d�finies par l'usager}
\label{fonctions}


La possibilit� pour l'usager de d�finir facilement et rapidement de
nouvelles fonctions --- et donc des extensions au langage --- est une
des grandes forces du S.


\section{D�finition d'une fonction}
\index{fonction!d�finie par l'usager}
\label{fonctions:definition}

On d�finit une nouvelle fonction de la mani�re suivante:
\begin{center}
  \Indexfonction{function}
  \code{fun <- function(\emph{arguments}) \emph{expression}}
\end{center}
o�
\begin{itemize}
\item \code{fun} est le nom de la fonction (les r�gles pour les noms
  de fonctions �tant les m�mes que pour tout autre objet);
\item \code{\itshape arguments} est la liste des arguments, s�par�s
  par des virgules;
\item \code{\itshape expression} constitue le corps de la fonction,
  soit une liste d'expressions group�es entre accolades (n�cessaires
  s'il y a plus d'une expression seulement).
\end{itemize}


\section{Retourner des r�sultats}
\index{fonction!resultats@r�sultat}
\label{fonctions:resultats}

La plupart des fonctions sont �crites dans le but de retourner un
r�sultat.
\begin{itemize}
\item Une fonction retourne tout simplement le r�sultat de la
  \emph{derni�re expression} du corps de la fonction.
\item On �vitera donc que la derni�re expression soit une affectation,
  car la fonction ne retournera alors rien.
\item On peut �galement utiliser explicitement la fonction
  \Fonction{return} pour retourner un r�sultat, mais cela est rarement
  n�cessaire.
\item Lorsqu'une fonction doit retourner plusieurs r�sultats, il est
  en g�n�ral pr�f�rable de le faire dans une liste nomm�e.
\end{itemize}


\section{Variables locales et globales}
\label{fonctions:variables}

Comme dans la majorit� des langages de programmation, les concepts de
variable locale et de variable globale existent en S.

\begin{itemize}
\item Toute variable d�finie dans une fonction est
  locale\index{variable!locale} � cette fonction, c'est-�-dire
  \begin{itemize}
  \item qu'elle n'appara�t pas dans l'espace de travail;
  \item qu'elle n'�crase pas une variable du m�me nom dans l'espace de
    travail.
  \end{itemize}
\item Il est possible de d�finir une variable\index{variable!globale}
  dans l'espace de travail depuis une fonction avec l'op�rateur
  d'affectation \verb|<<-|\index{<<-@\verb=<<-=|textbf}. Il est tr�s
  rare --- et g�n�ralement non recommand� --- de devoir recourir � de
  telles variables globales.
\item On peut d�finir une fonction � l'int�rieur d'une autre fonction.
  Cette fonction sera locale � la fonction dans laquelle elle est
  d�finie.
\end{itemize}


\section{Exemple de fonction}
\label{fonctions:exemple}

Le code d�velopp� pour l'exemple de point fixe\index{point fixe} de la
section \ref{exemples:pointfixe} peut �tre int�gr� dans une fonction
tel que montr� � la figure \ref{fig:fp}.

\begin{figure}
  \centering
  \begin{framed}
\begin{lstlisting}
fp <- function(k, n, start=0.05, TOL=1E-10)
{
    ## Fonction pour trouver par la m�thode du point
    ## fixe le taux d'int�r�t auquel une s�rie de 'n'
    ## paiements vaut 'k'.
    ##
    ## ARGUMENTS
    ##
    ##     k: la valeur pr�sente des paiements
    ##     n: le nombre de paiements
    ## start: point de d�part des it�rations
    ##   TOL: niveau de pr�cision souhait�
    ##
    ## RETOURNE
    ##
    ## Le taux d'int�r�t

    i <- start
    repeat
    {
        it <- i
        i <- (1 - (1 + it)^(-n))/k
        if (abs(i - it)/it < TOL)
            break
    }
    i  # ou return(i)
}
\end{lstlisting}
  \end{framed}
  \caption{Exemple de fonction de point fixe}
  \label{fig:fp}
\end{figure}

\begin{itemize}
\item Le nom de la fonction est \code{fp}.
\item La fonction compte cinq arguments: \code{k}, \code{n},
  \code{start} et \code{TOL}.
\item Les deux derniers arguments ont des valeurs par d�faut de $0,05$
  et $10^{-10}$, respectivement.
\item La fonction retourne la valeur de la variable \code{i}.
\item Avec Emacs\index{Emacs} et le mode ESS\index{ESS}, positionner
  le curseur � l'int�rieur de la fonction et soumettre le code d'une
  fonction � un processus S-Plus ou \textsf{R} avec \ess{C-c C-f}.
\end{itemize}



\section{Fonctions anonymes}
\index{fonction!anonyme}
\label{fonctions:anonymes}

Il est parfois utile de d�finir une fonction sans lui attribuer un nom
--- d'o� la notion de \emph{fonction anonyme}. Il s'agira en g�n�ral
de fonctions courtes utilis�es dans une autre fonction. Par exemple,
pour calculer la valeur de $x y^2$ pour toutes les combinaisons de $x$
et $y$ stock�es dans des vecteurs du m�me nom, on pourrait utiliser la
fonction \fonction{outer} ainsi:
\begin{Schunk}
\begin{Sinput}
> x <- 1:3
> y <- 4:6
> f <- function(x, y) x * y^2
> outer(x, y, f)
\end{Sinput}
\begin{Soutput}
     [,1] [,2] [,3]
[1,]   16   25   36
[2,]   32   50   72
[3,]   48   75  108
\end{Soutput}
\end{Schunk}
Cependant, si la fonction \code{f} ne sert � rien ult�rieurement, on
peut simplement utiliser une fonction anonyme � l'int�rieur de
\fonction{outer}:
\begin{Schunk}
\begin{Sinput}
> outer(x, y, function(x, y) x * y^2)
\end{Sinput}
\begin{Soutput}
     [,1] [,2] [,3]
[1,]   16   25   36
[2,]   32   50   72
[3,]   48   75  108
\end{Soutput}
\end{Schunk}


\section{D�bogage de fonctions}
\index{fonction!d�bogage}
\label{fonctions:debogage}

Nous n'abordons ici que les techniques les plus simples et na�ves.
\begin{itemize}
\item Les simples erreurs de syntaxe sont les plus fr�quentes (en
  particulier l'oubli de virgules). Lors de la d�finition d'une
  fonction, une v�rification de la syntaxe est effectu�e.
\item Lorsqu'une fonction ne retourne pas le r�sultat attendu, placer
  des commandes \fonction{print} � l'int�rieur de la fonction, de
  fa�on � pouvoir suivre les valeurs prises par les diff�rentes
  variables.

  Par exemple, la modification suivante � la boucle de la fonction
  \code{fp} permet d'afficher les valeurs successives de la variable
  \code{i} et de d�tecter une proc�dure it�rative divergente:
\begin{verbatim}
repeat
{
    it <- i
    i <- (1 - (1 + it)^(-n))/k
    print(i)
    if (abs((i - it)/it < TOL))
        break
}
\end{verbatim}
\item Avec Emacs\index{Emacs} et le mode ESS\index{ESS}, la principale
  technique de d�bogage consiste � s'assurer que toutes les variables
  pass�es en arguments � une fonction existent dans l'espace de
  travail, puis � ex�cuter successivement les lignes de la fonction
  avec \ess{C-c C-n}. Les interfaces graphiques de S-Plus et
  \textsf{R} ne permettent pas une telle proc�dure puisque la fen�tre
  d'�dition de fonctions bloque l'acc�s � l'interface de commande.
\end{itemize}


\section{Styles de codage}
\index{style}
\label{fonctions:style}

Si tous s'entendent que l'adoption qu'un style propre et uniforme
favorise le d�veloppement et la lecture de code, il existe plusieurs
chapelles dans le monde des programmeurs quant � la �bonne fa�on� de
pr�senter et, surtout, d'indenter du code informatique.

Par exemple, Emacs\index{Emacs} reconna�t et supporte les styles de
codage suivants, entre autres:
\begin{center}
  \begin{minipage}[t]{9cm}
    C++/Stroustrup
    \hfill
    \begin{minipage}[t]{5cm}
\begin{verbatim}
for (i in 1:10)
{
    expression
}
\end{verbatim}
    \end{minipage}
  \end{minipage}
  \vspace{\baselineskip}

  \begin{minipage}[t]{9cm}
    K\&R (1TBS)
    \hfill
    \begin{minipage}[t]{5cm}
\begin{verbatim}
for (i in 1:10){
     expression
}
\end{verbatim}
    \end{minipage}
  \end{minipage}
  \vspace{\baselineskip}

  \begin{minipage}[t]{9cm}
    Whitesmith
    \hfill
    \begin{minipage}[t]{5cm}
\begin{verbatim}
for (i in 1:10)
     {
     expression
     }
\end{verbatim}
    \end{minipage}
  \end{minipage}
  \vspace{\baselineskip}

  \begin{minipage}[t]{9cm}
    GNU
    \hfill
    \begin{minipage}[t]{5cm}
\begin{verbatim}
for (i in 1:10)
  {
    expression
  }
\end{verbatim}
    \end{minipage}
  \end{minipage}
\end{center}

\begin{itemize}
\item Pour des raisons g�n�rales de lisibilit� et de popularit�, le
  style C$++$, avec les accolades sur leurs propres lignes et une
  indentation de quatre (4) espaces est consid�r� standard en
  programmation en S.
\item Pour utiliser ce style dans Emacs\index{Emacs}, faire
\begin{verbatim}
    M-x ess-set-style RET C++ RET
\end{verbatim}
  une fois qu'un fichier de script est ouvert.
\item Pour �viter de devoir r�p�ter cette commande � chaque session de
  travail, cr�er ou �diter le fichier de configuration \texttt{.emacs}
  dans le dossier vers lequel pointe la variable d'environnement
  \texttt{HOME} et y placer les lignes suivantes:
\begin{verbatim}
(add-hook 'ess-mode-hook
          (lambda () (ess-set-style 'C++)))
\end{verbatim}
\end{itemize}


\section{Exemples}
\label{fonctions:exemples}

\lstinputlisting{fonctions.R}


\section{Exercices}
\label{fonctions:exercices}

\Opensolutionfile{reponses}[reponses-fonctions]
\Writetofile{reponses}{\protect\section*{Chapitre \protect\ref{fonctions}}}

\begin{exercice}
  La fonctions \texttt{var} calcule l'estimateur sans biais de la
  variance d'une population � partir de l'�chantillon donn� en
  argument. �crire une fonction \code{variance} qui calculera
  l'estimateur biais� ou sans biais selon que l'argument \code{biased}
  sera \code{TRUE} ou \code{FALSE}, respectivement. Le comportement
  par d�faut de \code{variance} devrait �tre le m�me que celui de
  \fonction{var}.  L'estimateur sans biais de la variance � partir
  d'un �chantillon $X_1, \dots, X_n$ est
  \begin{align*}
    S^2_{n-1}
    &= \frac{1}{n-1} \sum_{i=1}^n (X_i - \bar{X})^2, \\
    \intertext{alors que l'estimateur biais� est}
    S^2_n
    &= \frac{1}{n} \sum_{i=1}^n (X_i - \bar{X})^2,
  \end{align*}
  o� $\bar{X} = n^{-1}(X_1 + \dots + X_n)$.
  \begin{rep}
\begin{verbatim}
variance <- function(x, biased=FALSE)
{
    if (biased)
    {
        n <- length(x)
        (n - 1)/n * var(x)
    }
    else
        var(x)
}
\end{verbatim}
  \end{rep}
\end{exercice}

\begin{exercice}
  \index{matrice}
  �crire une fonction \code{matrix2} qui, contrairement � la fonction
  \fonction{matrix}, remplira par d�faut la matrice par ligne.  La
  fonction \emph{ne doit pas} utiliser \code{matrix}. Les arguments
  de la fonction \code{matrix2} seront les m�mes que ceux de
  \code{matrix}, sauf que l'argument \code{byrow} sera remplac�
  par \code{bycol}.
  \begin{rep}
    Une premi�re solution utilise la transpos�e. La premi�re
    expression de la fonction s'assure que la longueur de
    \code{data} est compatible avec le nombre de lignes et de
    colonnes de la matrice demand�e.
\begin{verbatim}
matrix2 <- function(data=NA, nrow=1, ncol=1,
                    bycol=FALSE, dimnames=NULL)
{
    data <- rep(data, length=nrow * ncol)

    if (bycol)
        dim(data) <- c(nrow, ncol)
    else
    {
        dim(data) <- c(ncol, nrow)
        data <- t(data)
    }

    dimnames(data) <- dimnames
    data
}
\end{verbatim}
    La seconde solution n'a pas recours � la transpos�e. Si l'on doit
    remplir la matrice par ligne, l'id�e consiste � r�ordonner les
    �l�ments du vecteur \code{data} en utilisant la formule obtenue
    � l'exercice \ref{operateurs}.\ref{exercice:operateurs:ijk}.
\begin{verbatim}
matrix2 <- function(data=NA, nrow=1, ncol=1,
                    bycol=FALSE, dimnames=NULL)
{
    data <- rep(data, length=nrow * ncol)

    if (!bycol)
    {
        i <- 1:nrow
        j <- rep(1:ncol, each=nrow)
        data <- data[(i - 1)*ncol + j]
    }
    dim(data) <- c(nrow, ncol)
    dimnames(data) <- dimnames
    data
}
\end{verbatim}
  \end{rep}
\end{exercice}

\begin{exercice}
  \index{distribution!normale}
  \label{exercice:fonctions:phi}
  �crire une fonction \code{phi} servant � calculer la fonction de
  densit� de probabilit� d'une loi normale centr�e r�duite, soit
  \begin{displaymath}
    \phi(x) = \frac{1}{\sqrt{2 \pi}} e^{-x^2/2},
    \quad -\infty < x < \infty.
  \end{displaymath}
  La fonction devrait prendre en argument un vecteur de valeurs de
  $x$. Comparer les r�sultats avec ceux de la fonction \fonction{dnorm}.
  \begin{rep}
\begin{verbatim}
phi <- function(x)
{
    exp(-x^2/2) / sqrt(2 * pi)
}
\end{verbatim}
  \end{rep}
\end{exercice}

\begin{exercice}
  \index{distribution!normale}
  \label{exercice:fonctions:Phi}
  �crire une fonction \code{Phi} servant � calculer la fonction de
  r�partition d'une loi normale centr�e r�duite, soit
  \begin{displaymath}
    \Phi(x) = \int_{-\infty}^x \frac{1}{\sqrt{2 \pi}} e^{-y^2/2}\, dy,
     \quad -\infty < x < \infty.
  \end{displaymath}
  Supposer, pour le moment, que $x \geq 0$. L'�valuation num�rique de
  l'int�grale ci-dessus peut se faire avec l'identit�
  \begin{displaymath}
    \Phi(x) = \frac{1}{2} + \phi(x) \sum_{n=0}^\infty
    \frac{x^{2n + 1}}{1 \cdot 3 \cdot 5 \cdots (2n + 1)},
    \quad x \geq 0.
  \end{displaymath}
  Utiliser la fonction \code{phi} de l'exercice
  \ref{fonctions}.\ref{exercice:fonctions:phi} et tronquer la somme
  infinie � une grande valeur, 50 par exemple. La fonction ne doit pas
  utiliser de boucles, mais peut ne prendre qu'une seule valeur de $x$
  � la fois.  Comparer les r�sultats avec ceux de la fonction
  \fonction{pnorm}.
  \begin{rep}
\begin{verbatim}
Phi <- function(x)
{
    n <- 1 + 2 * 0:50
    0.5 + phi(x) * sum(x^n / cumprod(n))
}
\end{verbatim}
  \end{rep}
\end{exercice}

\begin{exercice}
  \index{distribution!normale}
  \label{exercice:fonctions:Phi2}
  Modifier la fonction \code{Phi} de l'exercice
  \ref{fonctions}.\ref{exercice:fonctions:Phi} afin qu'elle admette
  des valeurs de $x$ n�gatives. Lorsque $x < 0$, $\Phi(x) = 1 -
  \Phi(-x)$. La solution simple consiste � utiliser une structure de
  contr�le \code{if ...  else}, mais les curieux chercheront � s'en
  passer. Les plus ambitieux regarderont m�me du c�t� de la fonction
  \fonction{Recall} \cite[page 49]{Sprogramming}.
  \begin{rep}
    Premi�re solution utilisant une fonction interne et une structure
    de contr�le \code{if ... else}.
\begin{verbatim}
Phi <- function(x)
{
    fun <- function(x)
    {
        n <- 1 + 2 * 0:50
        0.5 + phi(x) * sum(x^n / cumprod(n))
    }

    if (x < 0)
        1 - fun(-x)
    else
        fun(x)
}
\end{verbatim}
    Seconde solution r�cursive, c'est-�-dire que si $x < 0$, la
    fonction s'appelle elle-m�me avec un argument positif.
\begin{verbatim}
Phi <- function(x)
{
    if (x < 0)
        1 - Recall(-x)
    else
    {
        n <- 1 + 2 * 0:50
        0.5 + phi(x) * sum(x^n / cumprod(n))
    }
}
\end{verbatim}
    Troisi�me solution sans structure de contr�le \code{if ...
      else}. Rappelons que dans des calculs alg�briques,
    \code{FALSE} vaut 0 et \code{TRUE} vaut 1.
\begin{verbatim}
Phi <- function(x)
{
    n <- 1 + 2 * 0:50
    neg <- x < 0
    x <- abs(x)
    neg + (-1)^neg * (0.5 + phi(x) *
                      sum(x^n / cumprod(n)))
}
\end{verbatim}
  \end{rep}
\end{exercice}

\begin{exercice}
  \index{distribution!normale}
  G�n�raliser maintenant la fonction de l'exercice
  \ref{fonctions}.\ref{exercice:fonctions:Phi2} pour qu'elle prenne
  en argument un vecteur de valeurs de $x$. Ne pas utiliser de boucle.
  Comparer les r�sultats avec ceux de la fonction \fonction{pnorm}.
  \begin{rep}
\begin{verbatim}
Phi <- function(x)
{
    n <- 1 + 2 * 0:30
    0.5 + phi(x) * colSums(t(outer(x, n, "^")) /
                           cumprod(n))
}
\end{verbatim}
  \end{rep}
\end{exercice}

\begin{exercice}
  \index{matrice}
  Sans utiliser l'op�rateur \fonction{\%*\%}, �crire une fonction
  \code{prod.mat} qui effectuera le produit matriciel de deux
  matrices seulement si les dimensions de celles-ci le permettent.
  Cette fonction aura deux arguments (\code{mat1} et \code{mat2})
  et devra tout d'abord v�rifier si le produit matriciel est possible.
  Si celui-ci est impossible, la fonction retourne un message
  d'erreur.
  \begin{enumerate}
  \item Utiliser une structure de contr�le \code{if ... else} et deux
    boucles.
  \item Utiliser une structure de contr�le \code{if ... else} et une
    seule boucle.
  \end{enumerate}
  Dans chaque cas, comparer le r�sultat avec l'op�rateur \code{\%*\%}.
  \begin{rep}
    \begin{enumerate}
    \item
\begin{verbatim}
prod.mat <- function(mat1, mat2)
{
    if (ncol(mat1) == nrow(mat2))
    {
        res <- matrix(0, nrow=nrow(mat1),
                      ncol=ncol(mat2))
        for (i in 1:nrow(mat1))
        {
            for (j in 1:ncol(mat2))
            {
                res[i, j] <- sum(mat1[i,] * mat2[,j])
            }
        }
        res
    }
    else
        stop("Les dimensions des matrices ne
permettent pas le produit matriciel.")
}
\end{verbatim}
    \item
\begin{verbatim}
prod.mat<-function(mat1, mat2)
{
    if (ncol(mat1) == nrow(mat2))
    {
        res <- matrix(0, nrow=nrow(mat1),
                      ncol=ncol(mat2))
        for (i in 1:nrow(mat1))
            res[i,] <- colSums(mat1[i,] * mat2)
        res
    }
    else
        stop("Les dimensions des matrices ne
permettent pas le produit matriciel.")
}
\end{verbatim}
    \end{enumerate}
    Solutions bonus: deux fa�ons de faire �quivalentes qui cachent la
    boucle dans un \fonction{sapply}.
\begin{verbatim}
prod.mat<-function(mat1, mat2)
{
    if (ncol(mat1) == nrow(mat2))
        t(sapply(1:nrow(mat1),
                 function(i) colSums(mat1[i,] * mat2)))
    else
        stop("Les dimensions des matrices ne permettent
pas le produit matriciel.")
}

prod.mat<-function(mat1, mat2)
{
    if (ncol(mat1) == nrow(mat2))
        t(sapply(1:ncol(mat2),
                 function(j) colSums(t(mat1) * mat2[,j])))
    else
        stop("Les dimensions des matrices ne permettent
pas le produit matriciel.")
}
\end{verbatim}
  \end{rep}
\end{exercice}

\begin{exercice}
  \index{matrice}
  Vous devez calculer la note finale d'un groupe d'�tudiants � partir
  de deux informations:
  \begin{inparaenum}[(1)]
  \item une matrice contenant la note sur 100 de chacun des �tudiants
    � chacune des �valuations, et
  \item un vecteur contenant la pond�ration de chacune des
    �valuations.
  \end{inparaenum}
  Un coll�gue a compos� la fonction \code{notes.finales} ci-dessous
  afin de calculer la note finale de chacun des �tudiants.  Votre
  coll�gue vous mentionne toutefois que sa fonction est plut�t lente
  et inefficace pour de grands groupes d'�tudiants.  Vous d�cidez donc
  de modifier la fonction afin d'en r�duire le nombre d'op�rations et
  qu'elle n'utilise aucune boucle.
\begin{verbatim}
notes.finales <- function(notes, p)
{
    netud <- nrow(notes)
    neval <- ncol(notes)
    final <- (1:netud) * 0
    for(i in 1:netud)
    {
        for(j in 1:neval)
        {
            final[i] <- final[i] + notes[i, j] * p[j]
        }
    }
    final
}
\end{verbatim}
  \begin{rep}
\begin{verbatim}
notes.finales <- function(notes, p) notes %*% p
\end{verbatim}
  \end{rep}
\end{exercice}

\begin{exercice}
  Trouver les erreurs qui emp�chent la d�finition de la fonction
  ci-dessous.
\begin{verbatim}
AnnuiteFinPeriode <- function(n, i)
{{
    v <- 1/1 + i)
    ValPresChaquePmt <- v^(1:n)
    sum(ValPresChaquepmt)
}
\end{verbatim}
\end{exercice}

\begin{exercice}
  \index{distribution!normale}
  \index{distribution!gamma}
  \index{distribution!Pareto}
  La fonction ci-dessous calcule la valeur des param�tres d'une loi
  normale, gamma ou Pareto � partir de la moyenne et de la variance,
  qui sont connues par l'utilisateur.
\begin{verbatim}
param <- function(moyenne, variance, loi)
{
    loi <- tolower(loi)
    if (loi == "normale")
        param1 <- moyenne
        param2 <- sqrt(variance)
        return(list(mean=param1, sd=param2))
    if (loi == "gamma")
        param2 <- moyenne/variance
        param1 <- moyenne * param2
        return(list(shape=param1, scale=param2))
    if (loi == "pareto")
        cte <- variance/moyenne^2
        param1 <- 2 * cte/(cte-1)
        param2 <- moyenne * (param1 - 1)
        return(list(alpha=param1, lambda=param2))
    stop("La loi doit etre une de \"normale\",
\"gamma\" ou \"pareto\"")
}
\end{verbatim}
  L'utilisation de la fonction pour diverses lois donne les r�sultats
  suivants:
\begin{Schunk}
\begin{Sinput}
> param(2, 4, "normale")
\end{Sinput}
\begin{Soutput}
$mean
[1] 2

$sd
[1] 2
\end{Soutput}
\end{Schunk}
\begin{Schunk}
\begin{Sinput}
> param(50, 7500, "gamma")
\end{Sinput}
\end{Schunk}
\begin{Schunk}
\begin{Soutput}
Erreur dans param(50, 7500, "gamma") : Objet "param1"
non trouv�
\end{Soutput}
\end{Schunk}
\begin{Schunk}
\begin{Sinput}
> param(50, 7500, "pareto")
\end{Sinput}
\end{Schunk}
\begin{Schunk}
\begin{Soutput}
Erreur dans param(50, 7500, "pareto") : Objet "param1"
non trouv�
\end{Soutput}
\end{Schunk}
  \begin{enumerate}
  \item Expliquer pour quelle raison la fonction se comporte ainsi.
  \item Appliquer les corrections n�cessaires � la fonction pour que
    celle-ci puisse calculer les bonnes valeurs. (Les erreurs ne sont
    pas contenues dans les math�matiques de la fonction.) \emph{Astuce}:
    tirer profit du moteur d'indentation de Emacs.
  \end{enumerate}
  \begin{rep}
\begin{verbatim}
param <- function (moyenne, variance, loi)
{
    loi <- tolower(loi)
    if (loi == "normale")
    {
        param1 <- moyenne
        param2 <- sqrt(variance)
        return(list(mean=param1, sd=param2))
    }
    if (loi == "gamma")
    {
        param2 <- moyenne/variance
        param1 <- moyenne * param2
        return(list(shape=param1, scale=param2))
    }
    if (loi == "pareto")
    {
        cte <- variance/moyenne^2
        param1 <- 2 * cte/(cte-1)
        param2 <- moyenne * (param1 - 1)
        return(list(alpha=param1, lambda=param2))
    }
    stop("La loi doit etre une de \"normale\",
\"gamma\" ou \"pareto\"")
}
\end{verbatim}
  \end{rep}
\end{exercice}

\Closesolutionfile{reponses}

%%% Local Variables:
%%% mode: latex
%%% TeX-master: "introduction_programmation_S"
%%% End:

\chapter{Concepts avanc�s}
\label{avance}


Ce chapitre traite de divers concepts et fonctions un peu plus avanc�s
du langage S. Le lecteur int�ress� � approfondir ses connaissances de
ce langage pourra consulter \citet{Sprogramming}, en particulier les
chapitre 3 et 4.


\section{L'argument `\code{...}'}
\label{avance:dots}

La mention `\code{...}' appara�t dans la d�finition de plusieurs
fonction en S. Il ne faut pas voir l� de la paresse de la part des
r�dacteurs des rubriques d'aide, mais bel et bien un argument formel
dont `\Argument{...}' est le nom.
\begin{itemize}
\item Cet argument signifie qu'une fonction peut accepter un ou
  plusieurs autres arguments autres que ceux faisant partie de sa
  d�finition.
\item Le contenu de l'argument `\code{...}' n'est ni pris en compte,
  ni modifi� par la fonction.
\item Il est g�n�ralement simplement pass� tel quel � une autre
  fonction.
\item Voir les d�finitions des fonctions \fonction{apply},
  \fonction{lapply} et \fonction{sapply}, ci-dessous, pour des
  exemples.
\end{itemize}


\section{Fonction \code{apply}}
\label{avance:apply}

La fonction \Fonction{apply} sert � appliquer une fonction quelconque
sur une partie d'une matrice\index{matrice} ou, plus g�n�ralement,
d'un tableau\index{tableau}. La syntaxe de la fonction est la
suivante:
\begin{center}
  \code{apply(X, MARGIN, FUN, ...)},
\end{center}
o�
\begin{itemize}
\item \code{X} est une matrice ou un tableau (\emph{array});
\item \code{MARGIN} est un vecteur d'entiers contenant la ou les
  dimensions de la matrice ou du tableau sur lesquelles la fonction
  doit s'appliquer;
\item \code{FUN} est la fonction � appliquer;
\item `\code{...}' est un ensemble d'arguments suppl�mentaires,
  s�par�s par des virgules, � passer � la fonction \code{FUN}.
\end{itemize}

Lorsque \code{X} est une matrice, \fonction{apply} sert principalement
� calculer des sommaires par ligne (dimension 1) ou par colonne
(dimension 2) autres que la somme ou la moyenne (puisque les fonctions
\fonction{rowSums}, \fonction{colSums}, \fonction{rowMeans} et
\fonction{colMeans} existent pour ce faire).
\begin{itemize}
\item Utiliser la fonction \fonction{apply} plut�t que des boucles
  puisque celle-ci est plus efficace.
\item Consid�rer les exemples suivants.
\begin{Schunk}
\begin{Sinput}
> (m <- matrix(sample(1:100, 20, rep = TRUE), 
+     5, 4))
\end{Sinput}
\begin{Soutput}
     [,1] [,2] [,3] [,4]
[1,]   56   63   94   73
[2,]   62   30    6   25
[3,]   96    8   28   70
[4,]   62   56   78   35
[5,]   97   93   18   61
\end{Soutput}
\begin{Sinput}
> apply(m, 1, var)
\end{Sinput}
\begin{Soutput}
[1]  273.6667  540.9167 1587.6667  316.2500
[5] 1337.5833
\end{Soutput}
\begin{Sinput}
> apply(m, 2, min)
\end{Sinput}
\begin{Soutput}
[1] 56  8  6 25
\end{Soutput}
\begin{Sinput}
> apply(m, 1, mean, trim = 0.2)
\end{Sinput}
\begin{Soutput}
[1] 71.50 30.75 50.50 57.75 67.25
\end{Soutput}
\end{Schunk}
\end{itemize}

Puisqu'il n'existe pas de fonctions internes pour effectuer des
sommaires sur des tableaux, il faut toujours utiliser la fonction
\fonction{apply}. Si \code{X} est un tableau de plus de deux
dimensions, alors l'argument pass� � \code{FUN} peut �tre une matrice
ou un tableau.
\begin{itemize}
\item D�terminants des cinq sous-matrices $4 \times 4$ d'un tableau $4
  \times 4 \times 5$:
\begin{Schunk}
\begin{Sinput}
> arr <- array(sample(1:100, 80, rep = TRUE), 
+     c(4, 4, 5))
> apply(arr, 3, det)
\end{Sinput}
\begin{Soutput}
[1]   7987847   5940297 -34638968  -2330760
[5]  14753016
\end{Soutput}
\end{Schunk}
\end{itemize}



\section{Fonctions \code{lapply} et \code{sapply}}
\label{avance:lapply}

Les fonctions \Fonction{lapply} et \Fonction{sapply} sont similaires �
la fonction \fonction{apply} en ce qu'elles permettent d'appliquer une
fonction aux �l�ments d'une structure --- le vecteur ou la liste en
l'occurence. Leur syntaxe est similaire:
\begin{center}
  \code{lapply(X, FUN, ...)} \\
  \code{sapply(X, FUN, ...)}
\end{center}
\begin{itemize}
\item La fonction \fonction{lapply} applique une fonction \code{FUN} �
  tous les �l�ments d'un vecteur ou d'une liste \code{X} et retourne
  le r�sultat sous forme de liste.
\begin{Schunk}
\begin{Sinput}
> (v <- lapply(5:8, sample, x = 1:100))
\end{Sinput}
\begin{Soutput}
[[1]]
[1] 39 57 91  5 20

[[2]]
[1] 50 37 80 84 34  3

[[3]]
[1] 55 37 90  9 76 98 69

[[4]]
[1] 64 39 43 40 96 19 70 86
\end{Soutput}
\begin{Sinput}
> lapply(v, mean)
\end{Sinput}
\begin{Soutput}
[[1]]
[1] 42.4

[[2]]
[1] 48

[[3]]
[1] 62

[[4]]
[1] 57.125
\end{Soutput}
\end{Schunk}
\item La fonction \fonction{sapply} est similaire � \fonction{lapply},
  sauf que le r�sultat est retourn� sous forme de vecteur, si
  possible.
\begin{Schunk}
\begin{Sinput}
> sapply(v, mean)
\end{Sinput}
\begin{Soutput}
[1] 42.400 48.000 62.000 57.125
\end{Soutput}
\end{Schunk}
\item Si le r�sultat de chaque application de la fonction est un
  vecteur, alors \fonction{sapply} retourne une matrice, remplie comme
  toujours par colonne.
\begin{Schunk}
\begin{Sinput}
> (v <- lapply(rep(5, 3), sample, x = 1:100))
\end{Sinput}
\begin{Soutput}
[[1]]
[1] 70 55 21 98 67

[[2]]
[1] 21 15 45 18 47

[[3]]
[1] 68 51 97 59 32
\end{Soutput}
\begin{Sinput}
> sapply(v, sort)
\end{Sinput}
\begin{Soutput}
     [,1] [,2] [,3]
[1,]   21   15   32
[2,]   55   18   51
[3,]   67   21   59
[4,]   70   45   68
[5,]   98   47   97
\end{Soutput}
\end{Schunk}
\item Dans un grand nombre de cas, il est possible de remplacer les
  boucles \fonction{for} par l'utilisation de \fonction{lapply} ou
  \fonction{sapply}. On ne saurait donc trop insister sur l'importance
  de ces fonctions.
\end{itemize}


\section{Fonction \code{mapply}}
\label{avance:mapply}

La fonction \Fonction{mapply} est une version multidimensionnelle de
\code{sapply}. Sa syntaxe est, essentiellement,
\begin{center}
  \code{mapply(FUN, ...)}
\end{center}
\begin{itemize}
\item Le r�sultat de \code{mapply} est l'application de la fonction
  \code{FUN} aux premiers �l�ments de tous les arguments contenus dans
  `\code{...}', puis � tous les seconds �l�ments, et ainsi de suite.
\item Ainsi, si \code{v} et \code{w} sont des vecteurs,
  \code{mapply(FUN, v, w)} retourne sous forme de liste, de vecteur
  ou de matrice, selon le cas, \code{FUN(v[1], w[1])}, \code{FUN(v[2],
    w[2])}, etc.
\begin{Schunk}
\begin{Sinput}
> mapply(rep, 1:4, 4:1)
\end{Sinput}
\begin{Soutput}
[[1]]
[1] 1 1 1 1

[[2]]
[1] 2 2 2

[[3]]
[1] 3 3

[[4]]
[1] 4
\end{Soutput}
\end{Schunk}
\item Les �l�ments de `\code{...}' sont recycl�s au besoin.
\begin{Schunk}
\begin{Sinput}
> mapply(seq, 1:6, 6:8)
\end{Sinput}
\begin{Soutput}
[[1]]
[1] 1 2 3 4 5 6

[[2]]
[1] 2 3 4 5 6 7

[[3]]
[1] 3 4 5 6 7 8

[[4]]
[1] 4 5 6

[[5]]
[1] 5 6 7

[[6]]
[1] 6 7 8
\end{Soutput}
\end{Schunk}
\end{itemize}


\section{Fonction \code{replicate}}
\label{avance:replicate}

La fonction \Fonction{replicate}, propre � \textsf{R}, \R est une
fonction enveloppante de \fonction{sapply} simplifiant la syntaxe pour
l'ex�cution r�p�t�e d'une expression.

\begin{itemize}
\item Son usage est particuli�rement indiqu� pour les simulations.
  Ainsi, on peut construire une fonction \code{fun} qui fait tous
  les calculs d'une simulation, puis obtenir les r�sultats pour,
  disons, \nombre{10000} simulations avec
\begin{Schunk}
\begin{Sinput}
> replicate(10000, fun(...))
\end{Sinput}
\end{Schunk}
\item L'annexe \ref{simulation} pr�sente en d�tail diff�rentes
  strat�gies --- dont l'utilisation de \code{replicate} --- pour
  la r�alisation d'�tudes de simulation en S.
\end{itemize}


\section{Classes et fonctions g�n�riques}
\label{avance:classes}

Tous les objets dans le langage S ont une classe. La classe est
parfois implicite ou d�riv�e du mode de l'objet (consulter la rubrique
d'aide de \fonction{class} pour de plus amples d�tails).

\begin{itemize}
\item Certaines fonctions, dites fonctions
  \emph{g�n�riques}\index{fonction!g�n�rique}, se comportent
  diff�remment selon la classe de l'objet donn� en argument. Les
  fonctions g�n�riques les plus fr�quemment employ�es sont
  \fonction{print}, \fonction{plot} et \fonction{summary}.
\item Une fonction g�n�rique poss�de une \emph{m�thode} correspondant
  � chaque classe qu'elle reconna�t et, g�n�ralement, une m�thode
  \code{default} pour les autres objets. La liste des m�thodes
  existant pour une fonction g�n�rique s'obtient avec
  \Fonction{methods}:
\begin{Schunk}
\begin{Sinput}
> methods(plot)
\end{Sinput}
\begin{Soutput}
 [1] plot.acf*           plot.data.frame*   
 [3] plot.Date*          plot.decomposed.ts*
 [5] plot.default        plot.dendrogram*   
 [7] plot.density        plot.ecdf          
 [9] plot.factor*        plot.formula*      
[11] plot.hclust*        plot.histogram*    
[13] plot.HoltWinters*   plot.isoreg*       
[15] plot.lm             plot.medpolish*    
[17] plot.mlm            plot.POSIXct*      
[19] plot.POSIXlt*       plot.ppr*          
[21] plot.prcomp*        plot.princomp*     
[23] plot.profile.nls*   plot.spec          
[25] plot.spec.coherency plot.spec.phase    
[27] plot.stepfun        plot.stl*          
[29] plot.table*         plot.ts            
[31] plot.tskernel*      plot.TukeyHSD      

   Non-visible functions are asterisked
\end{Soutput}
\end{Schunk}
\item � chaque m�thode \code{methode} d'une fonction g�n�rique
  \code{fun} correspond une fonction \code{fun.methode}. C'est donc la
  rubrique d'aide de cette derni�re fonction qu'il faut consulter au
  besoin, et non celle de la fonction g�n�rique, qui contient en
  g�n�ral peu d'informations.
\item Il est int�ressant de savoir que lorsque l'on tape le nom d'un
  objet � la ligne de commande pour voir son contenu, c'est la
  fonction g�n�rique \fonction{print} qui est appel�e. On peut donc
  compl�tement modifier la repr�sentation � l'�cran du contenu d'un
  objet est cr�ant une nouvelle classe et une nouvelle m�thode pour la
  fonction \code{print}.
\end{itemize}


\section{Exemples}
\label{avance:exemples}

\lstinputlisting{avance.R}


\section{Exercices}
\label{avance:exercices}

\Opensolutionfile{reponses}[reponses-avance]
\Writetofile{reponses}{\protect\section*{Chapitre \protect\ref{avance}}}

\begin{exercice}
  \index{moyenne!pond�r�e}
  � l'exercice \ref{exercice:exemples:moyennes} du chapitre
  \ref{exemples}, on a calcul� la moyenne pond�r�e d'un vecteur
  d'observations
  \begin{displaymath}
    X_w = \sum_{i=1}^n \frac{w_i}{w_\pt}\, X_i,
  \end{displaymath}
  o� $w_\pt = \sum_{i=1}^n w_i$. Si l'on a plut�t une matrice $n
  \times p$ d'observations $X_{ij}$, on peut d�finir les moyennes
  pond�r�es
  \begin{align*}
    X_{iw}
    &= \sum_{j=1}^p \frac{w_{ij}}{w_{i\pt}}\, X_{ij}, \quad
    w_{i\pt} = \sum_{j=1}^p w_{ij} \\
    X_{wj}
    &= \sum_{i=1}^n \frac{w_{ij}}{w_{\pt j}}\, X_{ij}, \quad
    w_{\pt j} = \sum_{i=1}^n w_{ij} \\
    \intertext{et}
    X_{ww}
    &= \sum_{i=1}^n \sum_{j=1}^p \frac{w_{ij}}{w_{\pt\pt}}\, X_{ij}, \quad
    w_{\pt\pt} = \sum_{i=1}^n \sum_{j=1}^p w_{ij}.
  \end{align*}
  De m�me, on peut d�finir des moyennes pond�r�es calcul�es � partir
  d'un tableau de donn�es $X_{ijk}$ de dimensions $n \times p \times
  r$ dont la notation suit la m�me logique que ci-dessus. �crire des
  expressions S pour calculer, sans boucle, les moyennes pond�r�es
  suivantes.
  \begin{enumerate}
  \item $X_{iw}$ en supposant une matrice de donn�es $n \times p$.
  \item $X_{wj}$ en supposant une matrice de donn�es $n \times p$.
  \item $X_{ww}$ en supposant une matrice de donn�es $n \times p$.
  \item $X_{ijw}$ en supposant un tableau de donn�es $n \times p
    \times r$.
  \item $X_{iww}$ en supposant un tableau de donn�es $n \times p
    \times r$.
  \item $X_{wjw}$ en supposant un tableau de donn�es $n \times p
    \times r$.
  \item $X_{www}$ en supposant un tableau de donn�es $n \times p
    \times r$.
  \end{enumerate}
  \begin{rep}
    Soit \code{Xij} et \code{wij} des matrices, et \code{Xijk}
    et \code{wijk} des tableaux � trois dimensions.
    \begin{enumerate}
\item
\begin{Schunk}
\begin{Sinput}
> rowSums(Xij * wij)/rowSums(wij)
\end{Sinput}
\end{Schunk}
\item
\begin{Schunk}
\begin{Sinput}
> colSums(Xij * wij)/colSums(wij)
\end{Sinput}
\end{Schunk}
\item
\begin{Schunk}
\begin{Sinput}
> sum(Xij * wij)/sum(wij)
\end{Sinput}
\end{Schunk}
\item
\begin{Schunk}
\begin{Sinput}
> apply(Xijk * wijk, c(1, 2), sum)/apply(wijk, 
+     c(1, 2), sum)
\end{Sinput}
\end{Schunk}
\item
\begin{Schunk}
\begin{Sinput}
> apply(Xijk * wijk, 1, sum)/apply(wijk, 1, 
+     sum)
\end{Sinput}
\end{Schunk}
\item
\begin{Schunk}
\begin{Sinput}
> apply(Xijk * wijk, 2, sum)/apply(wijk, 2, 
+     sum)
\end{Sinput}
\end{Schunk}
\item
\begin{Schunk}
\begin{Sinput}
> sum(Xijk * wijk)/sum(wijk)
\end{Sinput}
\end{Schunk}
    \end{enumerate}
  \end{rep}
\end{exercice}

\begin{exercice}
  G�n�rer les suites de nombres suivantes � l'aide d'une expression S.
  (�videmment, il faut trouver un moyen de g�n�rer les suites sans
  simplement concat�ner les diff�rentes sous suites.)
  \begin{enumerate}
  \item $0, 0, 1, 0, 1, 2, \dots, 0, 1, 2, 3, \dots, 10$.
  \item $10, 9, 8, \dots, 2, 1, 10, 9, 8, \dots 3, 2, \dots, 10, 9, 10$
  \item $10, 9, 8, \dots, 2, 1, 9, 8, \dots, 2, 1, \dots, 2, 1, 1$
  \end{enumerate}
  \begin{rep}
    \begin{enumerate}
\item
\begin{Schunk}
\begin{Sinput}
> unlist(lapply(0:10, seq, from = 0))
\end{Sinput}
\end{Schunk}
\item
\begin{Schunk}
\begin{Sinput}
> unlist(lapply(1:10, seq, from = 10))
\end{Sinput}
\end{Schunk}
\item
\begin{Schunk}
\begin{Sinput}
> unlist(lapply(10:1, seq, to = 1))
\end{Sinput}
\end{Schunk}
    \end{enumerate}
  \end{rep}
\end{exercice}

\begin{exercice}
  \index{distribution!Pareto}
  \label{exercice:avance:pareto}
  La fonction de densit� de probabilit� et la fonction de r�partition
  de la loi de Pareto de param�tres $\alpha$ et $\lambda$ sont,
  respectivement,
  \begin{align*}
    f(x)
    &= \frac{\alpha \lambda^\alpha}{(x + \lambda)^{\alpha + 1}} \\
    \intertext{et}
    F(x)
    &= 1 - \left( \frac{\lambda}{x + \lambda} \right)^\alpha.
  \end{align*}
  La fonction suivante simule un �chantillon al�atoire de taille $n$
  issu d'une distribution de Pareto de param�tres $\alpha$ et
  $\lambda$:
\begin{verbatim}
rpareto <- function(n, alpha, lambda)
    lambda * (runif(n)^(-1/alpha) - 1)
\end{verbatim}
  \begin{enumerate}
  \item �crire une expression S permettant de simuler, en utilisant la
    fonction \code{rpareto} ci-dessus, cinq �chantillons al�atoires
    de tailles 100, 150, 200, 250 et 300 d'une loi de Pareto avec
    $\alpha = 2$ et $\lambda = \nombre{5000}$. Les �chantillons
    al�atoires devraient �tre stock�s dans une liste.
  \item On vous donne l'exemple suivant d'utilisation de la fonction
    \fonction{paste}:
\begin{Schunk}
\begin{Sinput}
> paste("a", 1:5, sep = "")
\end{Sinput}
\begin{Soutput}
[1] "a1" "a2" "a3" "a4" "a5"
\end{Soutput}
\end{Schunk}
    Nommer les �l�ments de la liste cr��e en (a)
    \code{echantillon1}, ..., \code{echantillon5}.
  \item Calculer la moyenne de chacun des �chantillons al�atoires
    obtenus en (a). Retourner le r�sultat dans un vecteur.
  \item �valuer la fonction de r�partition de la loi de Pareto$(2,
    \nombre{5000})$ en chacune des valeurs de chacun des �chantillons
    al�atoires obtenus en (a). Retourner les valeurs de la fonction de
    r�partition en ordre croissant.
  \item Faire l'histogramme des donn�es du cinqui�me �chantillon
    al�atoire � l'aide de la fonction \fonction{hist}.
  \item Ajouter \nombre{1000} � toutes les valeurs de tous les
    �chantillons simul�s en (a), ceci afin d'obtenir des observations
    d'une distribution de Pareto \emph{translat�e}.
  \end{enumerate}
  \begin{rep}
    \begin{enumerate}
\item
\begin{Schunk}
\begin{Sinput}
> ea <- lapply(seq(100, 300, by = 50), rpareto, 
+     alpha = 2, lambda = 5000)
\end{Sinput}
\end{Schunk}
\item
\begin{Schunk}
\begin{Sinput}
> names(ea) <- paste("echantillon", 1:5, sep = "")
\end{Sinput}
\end{Schunk}
\item
\begin{Schunk}
\begin{Sinput}
> sapply(ea, mean)
\end{Sinput}
\end{Schunk}
\item
\begin{Schunk}
\begin{Sinput}
> lapply(ea, function(x) sort(ppareto(x, 2, 
+     5000)))
> lapply(lapply(ea, sort), ppareto, alpha = 2, 
+     lambda = 5000)
\end{Sinput}
\end{Schunk}
\item
\begin{Schunk}
\begin{Sinput}
> hist(ea$echantillon5)
\end{Sinput}
\end{Schunk}
\item
\begin{Schunk}
\begin{Sinput}
> lapply(ea, "+", 1000)
\end{Sinput}
\end{Schunk}
    \end{enumerate}
  \end{rep}
\end{exercice}

\begin{exercice}
  Une base de donn�es contenant toutes les informations sur les
  assur�s est stock�e dans une liste de la fa�on suivante:
\begin{Schunk}
\begin{Sinput}
> x[[1]]
\end{Sinput}
\begin{Soutput}
$num.police
[1] 1001

$franchise
[1] 250

$nb.acc
 [1] 3 1 0 0 0 1 2 4 0 0

$montants
 [1] 1426.5918 2688.4555 6470.1165 8762.0631
 [5] 3011.0889 1253.5112 9490.0217 4383.2323
 [9] 1304.6549  912.9694 1134.0616
\end{Soutput}
\begin{Sinput}
> x[[2]]
\end{Sinput}
\begin{Soutput}
$num.police
[1] 1002

$franchise
[1] 1000

$nb.acc
[1] 2 1 0 2 4

$montants
[1]  1376.457  2885.352  3010.319  4043.079
[5]  4861.903  7404.045 13672.752  2332.587
[9]  2972.889
\end{Soutput}
\end{Schunk}
  Ainsi, \code{x[[i]]} contient les informations relatives � l'assur�
  $i$. Sans utiliser de boucles, �crire une expression ou une fonction S
  qui permettra de calculer les quantit�s suivantes.
  \begin{enumerate}
  \item La franchise moyenne dans le portefeuille.
  \item Le nombre annuel moyen de r�clamations par assur�.
  \item Le nombre total de r�clamations dans le portefeuille.
  \item Le montant moyen par accident dans le portefeuille.
  \item Le nombre d'assur�s n'ayant eu aucune r�clamation.
  \item Le nombre d'assur�s ayant eu une seule r�clamation dans leur
    premi�re ann�e.
  \item La variance du nombre total de sinistres.
  \item La variance du nombre de sinistres pour chaque assur�.
  \item La probabilit� empirique qu'une r�clamation soit inf�rieure �
    $x$ (un scalaire) dans le portefeuille.
  \item La probabilit� empirique qu'une r�clamation soit inf�rieure
    � $\mat{x}$ (un vecteur) dans le portefeuille.
  \end{enumerate}
  \begin{rep}
    \begin{enumerate}
\item
\begin{Schunk}
\begin{Sinput}
> mean(sapply(x, function(liste) liste$franchise))
\end{Sinput}
\end{Schunk}
Les crochets utilis�s pour l'indi�age constituent en fait un
op�rateur dont le �nom� est \fonction{[[}. On peut donc utiliser cet
op�rateur dans la fonction \code{sapply}:
\begin{Schunk}
\begin{Sinput}
> mean(sapply(x, "[[", "franchise"))
\end{Sinput}
\end{Schunk}
\item
\begin{Schunk}
\begin{Sinput}
> sapply(x, function(x) mean(x$nb.acc))
\end{Sinput}
\end{Schunk}
\item
\begin{Schunk}
\begin{Sinput}
> sum(sapply(x, function(x) sum(x$nb.acc)))
\end{Sinput}
\end{Schunk}
ou
\begin{Schunk}
\begin{Sinput}
> sum(unlist(sapply(x, "[[", "nb.acc")))
\end{Sinput}
\end{Schunk}
\item
\begin{Schunk}
\begin{Sinput}
> mean(unlist(lapply(x, "[[", "montants")))
\end{Sinput}
\end{Schunk}
\item
\begin{Schunk}
\begin{Sinput}
> sum(sapply(x, function(x) sum(x$nb.acc) == 
+     0))
\end{Sinput}
\end{Schunk}
\item
\begin{Schunk}
\begin{Sinput}
> sum(sapply(x, function(x) x$nb.acc[1] == 
+     1))
\end{Sinput}
\end{Schunk}
\item
\begin{Schunk}
\begin{Sinput}
> var(unlist(lapply(x, function(x) sum(x$nb.acc))))
\end{Sinput}
\end{Schunk}
\item
\begin{Schunk}
\begin{Sinput}
> sapply(x, function(x) var(x$nb.acc))
\end{Sinput}
\end{Schunk}
\item
\begin{Schunk}
\begin{Sinput}
> y <- unlist(lapply(x, "[[", "montants"))
> sum(y <= x)/length(y)
\end{Sinput}
\end{Schunk}
La fonction \fonction{ecdf} retourne une fonction permettant
de calculer la fonction de r�partition empirique en tout point:
\begin{Schunk}
\begin{Sinput}
> ecdf(unlist(lapply(x, "[[", "montants")))(x)
\end{Sinput}
\end{Schunk}
\item
\begin{Schunk}
\begin{Sinput}
> y <- unlist(lapply(x, "[[", "montants"))
> colSums(outer(y, x, "<="))/length(y)
\end{Sinput}
\end{Schunk}
La fonction retourn�e par \fonction{ecdf} accepte un vecteur
de points en argument:
\begin{Schunk}
\begin{Sinput}
> ecdf(unlist(lapply(x, "[[", "montants")))(x)
\end{Sinput}
\end{Schunk}
    \end{enumerate}
  \end{rep}
\end{exercice}

\Closesolutionfile{reponses}


%%% Local Variables:
%%% mode: latex
%%% TeX-master: "introduction_programmation_S"
%%% End:

\chapter{Fonctions d'optimisation}
\label{optimisation}



Les m�thodes de bissection, du point fixe, de Newton--Raphson et
consorts permettent de r�soudre des �quations � une variable de la
forme $f(x) = 0$ ou $g(x) = x$. Il existe �galement des versions de
ces m�thodes pour les syst�mes � plusieurs variables de la forme
\begin{align*}
  f_1(x_1, x_2, x_3) &= 0 \\
  f_2(x_1, x_2, x_3) &= 0 \\
  f_3(x_1, x_2, x_3) &= 0.
\end{align*}

De tels syst�mes d'�quations surviennent plus souvent qu'autrement
lors de l'optimisation d'une fonction. Par exemple, en recherchant le
maximum ou le minimum d'une fonction $f(x, y)$, on souhaitera r�soudre
le syst�me d'�quations
\begin{align*}
  \frac{\partial}{\partial x}\, f(x, y) &= 0 \\
  \frac{\partial}{\partial y}\, f(x, y) &= 0.
\end{align*}

En statistique, les fonctions d'optimisation sont fr�quemment
employ�es pour calculer num�riquement des estimateurs du maximum de
vraisemblance.

La grande majorit� des suites logicielles de calcul comportent des
outils d'optimisation de fonctions. Ce chapitre passe en revue les
fonctions disponibles dans S-Plus et \textsf{R}.


\section{Le package \texttt{MASS}}
\label{optimisation:MASS}

L'offre en fonctions d'optimisation est un des domaines o� S-Plus et
\textsf{R} diff�rent passablement. Il existe toutefois une option
commune avec le package \texttt{MASS}\index{package@\texttt{MASS}}.

Le package \texttt{MASS} \citep{MASS} contient plusieurs fonctions
utiles et de grande qualit�.  Les auteurs de ces fonctions contribuent
activement au d�veloppement de \textsf{R} et de S-Plus et, tel que
mentionn� au chapitre \ref{presentation}, leurs livres sur le langage
S \citep{Sprogramming,MASS} constituent des r�f�rences de choix.

Le package \texttt{MASS} est distribu� autant avec S-Plus (depuis au
moins la version 6.1) que \textsf{R}.  On peut aussi le t�l�charger
gratuitement depuis l'URL
\begin{quote}
  \url{http://www.stats.ox.ac.uk/pub/MASS4/Software.html}
\end{quote}
Pour acc�der aux fonctions du package, il suffit de le charger en
m�moire avec la commande
\begin{Schunk}
\begin{Sinput}
> library(MASS)
\end{Sinput}
\end{Schunk}


\section{Fonctions d'optimisation disponibles}
\label{optimisation:fonctions}

Les fonctions d'optimisation disponibles dans S-Plus et \textsf{R}
sont les suivantes.

\medskip

\subsection{\code{uniroot}}
\label{optimisation:uniroot}

La fonction \Fonction{uniroot} recherche la
racine\index{racine!d'une fonction}\index{fonction!racine} d'une fonction
\code{f} entre les points \code{lower} et \code{upper}. C'est la
fonction de base pour trouver la solution (unique) de l'�quation $f(x)
= 0$.
\begin{ex}
  Trouver la racine de la fonction $f(x) = x - 2^{-x}$ dans
  l'intervalle $[0, 1]$.
\end{ex}
\begin{sol}
  \mbox{}
\begin{Schunk}
\begin{Sinput}
> uniroot(function(x) x - 2^(-x), lower = 0, 
+     upper = 1)
\end{Sinput}
\begin{Soutput}
$root
[1] 0.6411922

$f.root
[1] 9.310346e-06

$iter
[1] 3

$estim.prec
[1] 6.103516e-05
\end{Soutput}
\end{Schunk}
\end{sol}

\subsection{\code{polyroot}}
\label{optimisation:polyroot}

La fonction \Fonction{polyroot} calcule toutes les
racines\index{racine!d'un polyn�me} (complexes) du polyn�me $\sum_{i=0}^n
a_i x^i$.  Le premier argument est le vecteur des coefficients $a_0,
a_1, \dots, a_n$, dans cet ordre.
\begin{ex}
  Trouver les racines du polyn�me $x^3 + 4x^2 - 10$.
\end{ex}
\begin{sol}
  \mbox{}
\begin{Schunk}
\begin{Sinput}
> polyroot(c(-10, 0, 4, 1))
\end{Sinput}
\begin{Soutput}
[1]  1.365230-0.000000i -2.682615+0.358259i
[3] -2.682615-0.358259i
\end{Soutput}
\end{Schunk}
\end{sol}

\subsection{\code{optimize}}
\label{optimisation:optimize}

La fonction \Fonction{optimize} recherche le
maximum\index{maximum!local}\index{fonction!maximum local} ou
minimum\index{minimum!local}\index{fonction!minimum local} local d'une
fonction \code{f} entre les points \code{lower} et \code{upper}.
\begin{ex}
  Trouver l'extremum de la fonction de densit� de la loi b�ta de
  param�tres $\alpha = 3$ et $\beta = 2$.
\end{ex}
\begin{sol}
  On sait que l'extremum se trouve dans l'intervalle $[0, 1]$.
\begin{Schunk}
\begin{Sinput}
> f <- function(x) dbeta(x, 3, 2)
> optimize(f, lower = 0, upper = 1, maximum = TRUE)
\end{Sinput}
\begin{Soutput}
$maximum
[1] 0.6666795

$objective
[1] 1.777778
\end{Soutput}
\end{Schunk}
\end{sol}

\subsection{\code{ms}}
\label{optimisation:ms}

La \Splus fonction \Fonction{ms}, minimise une
somme\index{minimum!d'une somme}. C'est une des principales fonction
d'optimisation de S-Plus. Elles est utile, par exemple, pour minimiser
la valeur n�gative d'une fonction de
log-vraisemblance\index{vraisemblance}, $-l(\theta) = - \sum_{i=1}^n
\ln f(x_i; \theta)$.  Son utilisation est toutefois compliqu�e par
l'usage de formules\index{formule} (voir la section
\ref{regression:formules}) et de \emph{data frames}\index{data frame}
(section \ref{bases:dataframes}).
\begin{ex}
  \label{ex:optimisation:ms}
  Calculer les estimateurs du maximum de vraisemblance des param�tres
  $\alpha$ et $\lambda$ de la distribution gamma dont la densit� est
  donn�e � l'�quation \eqref{eq:exemples:gamma:fdp} � la page
  \pageref{eq:exemples:gamma:fdp} � partir de l'�chantillon al�atoire
\begin{Schunk}
\begin{Sinput}
> x
\end{Sinput}
\begin{Soutput}
 [1] 2.2557923 2.6291918 2.1579953 5.2925777
 [5] 0.8625360 0.6744605 1.5091443 1.0829637
 [9] 2.5340812 1.9135480
\end{Soutput}
\end{Schunk}
\end{ex}
\begin{sol}
  On cherche � minimiser $-l(\alpha, \lambda) = -\sum_{i=1}^n \ln
  f(x_i; \alpha, \lambda)$, donc l'argument de \code{ms} doit �tre
  $- \ln f(x_i; \alpha, \lambda)$.
\begin{Schunk}
\begin{Sinput}
> x <- rgamma(10, shape=5, rate=2)
> ms(~-log(dgamma(x, a, l)),
           data=as.data.frame(x),
           start=list(a=1, l=1))
\end{Sinput}
\begin{Soutput}
value: 14.60445
parameters:
        a        l
 3.217898 1.538759
formula:  ~   - log(dgamma(x, a, l))
100 observations
call: ms(formula =  ~  - log(dgamma(x, a, l)), data =
as.data.frame(x), start = list(         a = 1, l = 1))
\end{Soutput}
\end{Schunk}
\end{sol}

\subsection{\code{nlmin}}
\label{optimisation:nlmin}

La fonction \Fonction{nlmin}, \Splus propre � S-Plus, minimise une
fonction non lin�aire\index{minimum!fonction non
  lin�aire}\index{fonction!minimum}.  La fonction que \code{nlmin}
minimisera ne peut avoir qu'un seul argument, soit le vecteur des
param�tres � trouver.
\begin{ex}
  R�p�ter l'exemple \ref{ex:optimisation:ms} � l'aide de
  \code{nlmin} dans S-Plus.
\end{ex}
\begin{sol}
  Il faut cette fois passer en argument la fonction $-l(\alpha,
  \lambda)$. Le second argument, \code{c(1, 1)}, contient des
  valeurs de d�part.
\begin{Schunk}
\begin{Sinput}
> f <- function(p) -sum(log(dgamma(x, p[1], p[2])))
> nlmin(f, c(1, 1))
\end{Sinput}
\begin{Soutput}
$x:
[1] 3.217898 1.538759

$converged:
[1] T

$conv.type:
[1] "relative function convergence"
\end{Soutput}
\end{Schunk}
%$
\end{sol}

\subsection{\code{nlminb}}
\Indexfonction{nlminb}
\label{optimisation:nlminb}

Minimisation \Splus d'une fonction non lin�aire avec des bornes
inf�rieure et/ou sup�rieure pour les param�tres (S-Plus seulement).


\subsection{\code{nlm}}
\label{optimisation:nlm}

La fonction \Fonction{nlm}, \R propre � \textsf{R}, minimise aussi une
fonction non lin�aire\index{minimum!fonction non
  lin�aire}\index{fonction!minimum}.  La principale diff�rence entre
la fonction \fonction{nlmin} de S-Plus et \code{nlm} est que cette
derni�re peut passer des arguments � la fonction � minimiser, ce qui
en facilite l'utilisation.
\begin{ex}
  \label{ex:optimisation:nlm}
  R�p�ter l'exemple \ref{ex:optimisation:ms} � l'aide de
  \code{nlm} dans \textsf{R}.
\end{ex}
\begin{sol}
  Remarquer comment on peut passer le vecteur de donn�es � la fonction
  de log-vraisemblance � optimiser.
\begin{Schunk}
\begin{Sinput}
> f <- function(p, x) -sum(dgamma(x, p[1], 
+     p[2], log = TRUE))
> nlm(f, c(1, 1), x = x)
\end{Sinput}
\begin{Soutput}
$minimum
[1] 14.60445

$estimate
[1] 3.217881 1.538752

$gradient
[1] -1.057352e-05  2.737923e-05

$code
[1] 2

$iterations
[1] 15
\end{Soutput}
\end{Schunk}
\end{sol}

\subsection{\code{optim}}
\label{optimisation:optim}

La fonction \Fonction{optim}\index{fonction!optimisation} est un outil
d'optimisation tout usage, souvent utilis�e par d'autres fonctions.
Elle permet, selon l'algorithme utilis�, de fixer des seuils minimum
et/ou maximum aux param�tres � optimiser.  Dans S-Plus, il faut
charger la section \texttt{MASS}\index{package@\texttt{MASS}} de la
biblioth�que.
\begin{ex}
  R�p�ter l'exemple \ref{ex:optimisation:ms} � l'aide de
  \code{optim}.
\end{ex}
\begin{sol}
  En r�utilisant la fonction \code{f} d�finie dans la solution de
  l'exemple \ref{ex:optimisation:nlm}:
\begin{Schunk}
\begin{Sinput}
> optim(c(1, 1), f, x = x)
\end{Sinput}
\begin{Soutput}
$par
[1] 3.217098 1.538413

$value
[1] 14.60445

$counts
function gradient 
      65       NA 

$convergence
[1] 0

$message
NULL
\end{Soutput}
\end{Schunk}
\end{sol}

\begin{rem}
  L'option \R \code{log = TRUE} de la fonction \fonction{dgamma} (et
  de toutes les autres fonctions de densit�) permet de calculer plus
  pr�cis�ment le logarithme de la densit�.  Cette option n'est
  disponible que dans \textsf{R}.
\end{rem}

\begin{rem}
  L'estimation par le maximum de vraisemblance\index{vraisemblance}
  est beaucoup simplifi�e par l'utilisation de la fonction
  \fonction{fitdistr} du package
  \texttt{MASS}\index{package@\texttt{MASS}}.
\end{rem}


\section{Exemples}
\label{optimisation:exemples}

\lstinputlisting{optimisation.R}



\section{Exercices}
\label{optimisation:exercices}

\Opensolutionfile{reponses}[reponses-optimisation]
\Writetofile{reponses}{\protect\section*{Chapitre \protect\ref{optimisation}}}


\begin{exercice}
  Trouver la solution des �quations suivantes � l'aide des fonctions S
  appropri�es.
  \begin{enumerate}
  \item $x^3 - 2 x^2 - 5 = 0$ pour $1 \leq x \leq 4$
  \item $x^3 + 3 x^2 - 1 = 0$ pour $-4 \leq x \leq 0$
  \item $x - 2^{-x} = 0$ pour $0 \leq x \leq 1$
  \item $e^x + 2^{-x} + 2 \cos x - 6 = 0$ pour $1 \leq x \leq 2$
  \item $e^x - x^2 + 3x - 2 = 0$ pour $0 \leq x \leq 1$
  \end{enumerate}
  \begin{rep}
    \begin{enumerate}
\item
\begin{Schunk}
\begin{Sinput}
> f <- function(x) x^3 - 2 * x^2 - 5
> uniroot(f, lower = 1, upper = 4)
\end{Sinput}
\end{Schunk}
\item
\begin{Schunk}
\begin{Sinput}
> f <- function(x) x^3 + 3 * x^2 - 1
> uniroot(f, lower = -4, upper = -1)
\end{Sinput}
\end{Schunk}
\item
\begin{Schunk}
\begin{Sinput}
> f <- function(x) x - 2^(-x)
> uniroot(f, lower = 0, upper = 1)
\end{Sinput}
\end{Schunk}
\item
\begin{Schunk}
\begin{Sinput}
> f <- function(x) exp(x) + 2^(-x) + 2 * cos(x) - 
+     6
> uniroot(f, lower = 1, upper = 2)
\end{Sinput}
\end{Schunk}
\item
\begin{Schunk}
\begin{Sinput}
> f <- function(x) exp(x) - x^2 + 3 * x - 
+     2
> uniroot(f, lower = 0, upper = 1)
\end{Sinput}
\end{Schunk}
    \end{enumerate}
  \end{rep}
\end{exercice}

\begin{exercice}
  En th�orie de la cr�dibilit�, l'estimateur d'un param�tre $a$ est
  donn� sous forme de point fixe
  \begin{displaymath}
    \hat{a} = \frac{1}{n - 1} \sum_{i=1}^n z_i (X_i - \bar{X}_z)^2,
  \end{displaymath}
  o�
  \begin{align*}
    z_i &= \frac{\hat{a} w_i}{\hat{a} w_i + s^2} \\
    \bar{X}_z &= \sum_{i=1}^n \frac{z_i}{z_\pt} X_i
  \end{align*}
  et $X_1, \dots, X_n$, $w_1, \dots, w_n$ et $s^2$ sont des donn�es.
  Calculer la valeur de $\hat{a}$ si $s^2 = \nombre{140 000 000}$ et
  que les valeurs de $X_i$ et $w_i$ sont telles que donn�es dans le
  tableau ci-dessous.
  \begin{center}
    \begin{tabular}{crrrrr}
      \toprule
      $i$ & 1 & 2 & 3 & 4 & 5 \\
      \midrule
      $X_i$ &
      \nombre{2061} & \nombre{1511} &
      \nombre{1806} & \nombre{1353} & \nombre{1600} \\
      $w_i$ &
      \nombre{100155} & \nombre{19895} &
      \nombre{13735}  & \nombre{4152}  & \nombre{36110} \\
      \bottomrule
    \end{tabular}
  \end{center}
  \begin{rep}
\begin{Schunk}
\begin{Sinput}
> X <- c(2061, 1511, 1806, 1353, 1600)
> w <- c(100155, 19895, 13735, 4152, 36110)
> g <- function(a, X, w, s2) {
+     z <- 1/(1 + s2/(a * w))
+     Xz <- sum(z * X)/sum(z)
+     sum(z * (X - Xz)^2)/(length(X) - 1)
+ }
> uniroot(function(x) g(x, X, w, 1.4e+08) - 
+     x, c(50000, 80000))
\end{Sinput}
\end{Schunk}
  \end{rep}
\end{exercice}


\begin{exercice}
  Les fonctions de densit� de probabilit� et de r�partition de la
  distribution de Pareto sont donn�es � l'exercice
  \ref{avance}.\ref{exercice:avance:pareto}. Calculer les estimateurs
  du maximum de vraisemblance des param�tres de la Pareto � partir d'un
  �chantillon al�atoire obtenu par simulation avec la commande
\begin{Schunk}
\begin{Sinput}
> x <- lambda * (runif(100)^(-1/alpha) - 1)
\end{Sinput}
\end{Schunk}
  pour des valeurs de \texttt{alpha} et \texttt{lambda} choisies.
  \begin{rep}
\begin{Schunk}
\begin{Sinput}
> dpareto <- function(x, alpha, lambda) {
+     (alpha * lambda^alpha)/(x + lambda)^(alpha + 
+         1)
+ }
> f <- function(par, x) -sum(log(dpareto(x, 
+     par[1], par[2])))
> optim(c(1, 1000), f, x = x)
\end{Sinput}
\end{Schunk}
    ou
\begin{Schunk}
\begin{Sinput}
> dpareto <- function(x, logAlpha, logLambda) {
+     alpha <- exp(logAlpha)
+     lambda <- exp(logLambda)
+     (alpha * lambda^alpha)/(x + lambda)^(alpha + 
+         1)
+ }
> optim(c(log(2), log(1000)), f, x = x)
> exp(optim(c(log(2), log(1000)), f, x = x)$par)
\end{Sinput}
\end{Schunk}
  \end{rep}
\end{exercice}

\Closesolutionfile{reponses}


%%% Local Variables:
%%% mode: latex
%%% TeX-master: "introduction_language_S"
%%% End:

\chapter{G�n�rateurs de nombres al�atoires}
\label{rng}


Avant d'utiliser pour quelque t�che de simulation moindrement
importante un g�n�rateur de nombres al�atoires inclus dans un
logiciel, il importe de s'assurer de la qualit� de celui-ci. On
trouvera en g�n�ral relativement facilement de l'information dans
Internet.

On pr�sente ici, sans entrer dans les d�tails, les g�n�rateurs de
nombres uniformes utilis�s dans S-Plus et \textsf{R} ainsi que la
liste des diff�rentes fonctions de simulation de variables al�atoires.


\section{G�n�rateurs de nombres al�atoires}
\index{simulation!nombres uniformes}
\label{rng:generateurs}

On obtient des nombres uniformes sur un intervalle quelconque (par
d�faut $[0, 1]$) avec la fonction \Fonction{runif} dans S-Plus et
\textsf{R}. L'amorce du g�n�rateur al�atoire est d�termin�e avec la
fonction \Fonction{set.seed}.

Dans S-Plus, \Splus le g�n�rateur utilis� est une version modifi�e de
\emph{Super Duper}. Sa p�riode est $2^{30} \times \nombre{4292868097}
\approx 4,6 \times 10^{18}$.

Dans \textsf{R}, \R on a la possibilit� de choisir entre six
g�n�rateurs de nombres al�atoires diff�rents, ou encore de sp�cifier
son propre g�n�rateur. Par d�faut, \textsf{R} utilise le g�n�rateur
Marsenne--Twister, consid�r� comme le plus avanc� au moment d'�crire
ces lignes.  La p�riode de ce g�n�rateur est $2^{\nombre{19937}} - 1$,
rien de moins!

Consulter les rubriques d'aide des fonctions \code{.Random.seed}
\index{Random.seed@\code{.Random.seed}} et \code{set.seed} pour de
plus amples d�tails.


\section{Fonctions de simulation de variables al�atoires}
\index{simulation!variables al�atoires}
\label{rng:va}

Les caract�ristiques de plusieurs lois de probabilit� sont directement
accessibles dans S-Plus et \textsf{R} par un large �ventail de
fonctions. La logique r�gne dans les noms de fonctions: pour chaque
racine \code{\textit{loi}}, il existe quatre fonctions diff�rentes:
\begin{enumerate}
\item \code{d\textit{loi}} calcule la fonction de densit� de
  probabilit� (lois continues) ou la fonction de masse de probabilit�
  (lois discr�tes);
\item \code{p\textit{loi}} calcule la fonction de r�partition;
\item \code{q\textit{loi}} calcule la fonction de quantile;
\item \code{r\textit{loi}} simule des observations de cette loi.
\end{enumerate}

Les diff�rentes lois de probabilit� disponibles dans S-Plus et
\textsf{R}, leur racine et le nom de leurs param�tres sont rassembl�es
au tableau \ref{tab:rng:lois}

\begin{table}
  \index{distribution!b�ta}
  \index{distribution!binomiale}
  \index{distribution!binomiale n�gative}
  \index{distribution!Cauchy}
  \index{distribution!exponentielle}
  \index{distribution!F}
  \index{distribution!gamma}
  \index{distribution!geometrique@g�om�trique}
  \index{distribution!hyperg�om�trique}
  \index{distribution!khi carr�}
  \index{distribution!logistique}
  \index{distribution!log-normale}
  \index{distribution!normale}
  \index{distribution!Poisson}
  \index{distribution!t}
  \index{distribution!uniforme}
  \index{distribution!Weibull}
  \index{distribution!Wilcoxon}
  \centering
  \begin{tabular}{lll}
    \toprule
    Loi de probabilit� & Racine dans S & Noms des param�tres \\
    \midrule
    B�ta & \texttt{beta} & \texttt{shape1}, \texttt{shape2} \\
    Binomiale & \texttt{binom} & \texttt{size}, \texttt{prob} \\
    Binomiale n�gative & \texttt{nbinom} & \texttt{size},
    \texttt{prob} ou \texttt{mu} \\
    Cauchy & \texttt{cauchy} & \texttt{location}, \texttt{scale} \\
    Exponentielle & \texttt{exp} & \texttt{rate} \\
    \emph{F} (Fisher) & \texttt{f} & \texttt{df1}, \texttt{df2} \\
    Gamma & \texttt{gamma} & \texttt{shape}, \texttt{rate} ou
    \texttt{scale} \\
    G�om�trique & \texttt{geom} & \texttt{prob} \\
    Hyperg�om�trique & \texttt{hyper} & \texttt{m}, \texttt{n},
    \texttt{k} \\
    Khi carr� & \texttt{chisq} & \texttt{df} \\
    Logistique & \texttt{logis} & \texttt{location}, \texttt{scale} \\
    Log-normale & \texttt{lnorm} & \texttt{meanlog}, \texttt{sdlog} \\
    Normale & \texttt{norm} & \texttt{mean}, \texttt{sd} \\
    Poisson & \texttt{pois} & \texttt{lambda} \\
    \emph{t} (Student) & \texttt{t} & \texttt{df} \\
    Uniforme & \texttt{unif} & \texttt{min}, \texttt{max} \\
    Weibull & \texttt{weibull} & \texttt{shape}, \texttt{scale} \\
    Wilcoxon & \texttt{wilcox} & \texttt{m}, \texttt{n} \\
    \bottomrule
  \end{tabular}
  \caption{Lois de probabilit� pour lesquelles existent des fonctions dans S-Plus et \textsf{R}}
  \label{tab:rng:lois}
\end{table}

Toutes les fonctions du tableau \ref{tab:rng:lois} sont vectorielles,
c'est-�-dire qu'elles acceptent en argument un vecteur de points o� la
fonction (de densit�, de r�partition ou de quantile) doit �tre �valu�e
et m�me un vecteur de param�tres. Par exemple,
\begin{Schunk}
\begin{Sinput}
> dpois(c(3, 0, 8), lambda = c(1, 4, 10))
\end{Sinput}
\begin{Soutput}
[1] 0.06131324 0.01831564 0.11259903
\end{Soutput}
\end{Schunk}
retourne la probabilit� que des lois de Poisson de param�tre 1, 4, et
10 prennent les valeurs 3, 0 et 8, respectivement.

Le premier argument des fonctions de simulation est la quantit� de
nom\-bres al�atoires d�sir�e.  Ainsi,
\begin{Schunk}
\begin{Sinput}
> rpois(3, lambda = c(1, 4, 10))
\end{Sinput}
\begin{Soutput}
[1] 0 4 9
\end{Soutput}
\end{Schunk}
retourne trois nombres al�atoires issus de distributions de Poisson
de param�tre 1, 4 et 10, respectivement.  �videmment, passer un
vecteur comme premier argument n'a pas tellement de sens, mais, si
c'est fait, S retournera une quantit� de nombres al�atoires �gale � la
\emph{longueur} du vecteur (sans �gard aux valeurs contenues dans le
vecteur).

La fonction \Fonction{sample} permet de simuler des nombres d'une
distribution discr�te quelconque. Sa syntaxe est
\begin{center}
  \code{sample(x, size, replace = FALSE, prob = NULL)},
\end{center}
o� \code{x} est un vecteur des valeurs possibles de l'�chantillon �
simuler (le support de la distribution), \code{size} est la quantit�
de nombres � simuler et \code{prob} est un vecteur de probabilit�s
associ�es � chaque valeur de \code{x} (\code{1/length(x)} par d�faut).
Enfin, si \code{replace} est \code{TRUE}, l'�chantillonnage se fait
avec remise.

\section{Exercices}
\label{rng:exercices}

\Opensolutionfile{reponses}[reponses-rng]
\Writetofile{reponses}{\protect\section*{Annexe \protect\ref{rng}}}

\begin{exercice}
  La loi log-normale\index{distribution!log-normale} est obtenue par
  transformation de la loi normale: si la distribution de la variable
  alatoire $X$ est une normale de param�tres $\mu$ et $\sigma^2$,
  alors la distribution de $e^X$ est une log-normale. Simuler
  \nombre{1000} observations d'une loi log-normale de param�tres $\mu
  = \ln 5000 - \frac{1}{2}$ et $\sigma^2 = 1$, puis tracer
  l'histogramme de l'�chantillon al�atoire obtenu.
  \begin{rep}
\begin{Schunk}
\begin{Sinput}
> x <- rlnorm(1000, meanlog = log(5000) - 
+     0.5, sdlog = 1)
> hist(x)
\end{Sinput}
\end{Schunk}
  \end{rep}
\end{exercice}

\begin{exercice}
  Simuler \nombre{10000} observations d'un m�lange continu
  Poisson/gamma\index{distribution!m�lange Poisson/gamma} o� les
  param�tres de la loi gamma sont $\alpha = 5$ et $\lambda = 4$, puis
  tracer la distribution de fr�quence de l'�chantillon al�atoire
  obtenu � l'aide des fonctions \fonction{plot} et \fonction{table}.
  Superposer � ce graphique la fonction de probabilit� d'une binomiale
  n�gative de param�tres $r = 5$ et $\theta = 0,8$.
  \begin{rep}
\begin{Schunk}
\begin{Sinput}
> x <- rpois(10000, lambda = rgamma(10000, 
+     shape = 5, rate = 4))
> xx <- seq(min(x), max(x))
> px <- table(x)
> plot(xx, dnbinom(xx, size = 5, prob = 0.8), 
+     type = "h", lwd = 5, col = "blue")
> points(xx, px/length(x), pch = 16)
\end{Sinput}
\end{Schunk}
  \end{rep}
\end{exercice}

\begin{exercice}
  Simuler \nombre{10000} observations d'un m�lange
  discret\index{distribution!m�lange discret} de deux distributions
  log-normales, l'une de param�tres $(\mu=3,5, \sigma^2=0,6)$ et
  l'autre de param�tres $(\mu=4,6, \sigma^2=0,3)$.  Utiliser un
  param�tre de m�lange $p = 0,55$. Tracer ensuite l'histogramme de
  l'�chantillon al�atoire obtenu.
  \begin{rep}
\begin{Schunk}
\begin{Sinput}
> w <- rbinom(1, 10000, 0.55)
> x <- c(rlnorm(w, 3.5, 0.6), rlnorm(10000 - 
+     w, 4.6, 0.3))
> hist(x)
\end{Sinput}
\end{Schunk}
  \end{rep}
\end{exercice}

\Closesolutionfile{reponses}

%%% Local Variables:
%%% mode: latex
%%% TeX-master: "introduction_programation_S"
%%% End:

\chapter{Planification d'une simulation en S}
\index{simulation!planification|(}
\label{simulation}



\section{Introduction}
\label{simulation:intro}

La simulation est de plus en plus utilis�e pour r�soudre des probl�mes
complexes. Il existe de multiples fa�ons de r�aliser la mise en
{\oe}uvre informatique d'une simulation, mais certaines sont plus
efficaces que d'autres.  Ce document passe en revue diverses fa�ons de
faire des simulations avec S-Plus et \textsf{R} � l'aide d'un exemple
simple de nature statistique.

Soit $X_1, \dots, X_n$ un �chantillon al�atoire tir� d'une population
distribu�e selon une loi uniforme sur l'intervalle $(\theta -
\frac{1}{2}, \theta + \frac{1}{2})$. On consid�re les trois
estimateurs suivants du param�tre inconnu $\theta$:
\begin{enumerate}
\item la moyenne arithm�tique
  \begin{displaymath}
    \hat{\theta}_1 = \frac{1}{n} \sum_{i=1}^n X_i\,;
  \end{displaymath}
\item la m�diane empirique
  \begin{displaymath}
    \hat{\theta}_2 =
    \begin{cases}
      X_{(\frac{n+1}{2})}, & \text{$n$ impair} \\
      \frac{1}{2}(X_{(\frac{n}{2})} + X_{(\frac{n}{2} + 1)}), & \text{$n$ pair},
    \end{cases}
  \end{displaymath}
  o� $X_{(k)}$ est la $k$\ieme{} statistique d'ordre de l'�chantillon
  al�atoire;
\item la mi-�tendue
  \begin{displaymath}
    \hat{\theta}_3 = \frac{X_{(1)} + X_{(n)}}{2}.
  \end{displaymath}
\end{enumerate}

� l'aide de la simulation on veut, d'une part, v�rifier si les trois
estimateurs sont bel et bien sans biais et, d'autre part, d�terminer
lequel a la plus faible variance. Pour ce faire, on doit d'abord
simuler un grand nombre $N$ d'�chantillons al�atoires de taille $n$
d'une distribution $U(\theta - \frac{1}{2}, \theta + \frac{1}{2})$
pour une valeur de $\theta$ choisie. Pour chaque �chantillon, on
calculera ensuite les trois estimateurs ci-dessus, puis la moyenne et
la variance, par type d'estimateur, de tous les estimateurs obtenus.
Si la moyenne des $N$ estimateurs $\hat{\theta}_i$, $i = 1, 2, 3$ est
pr�s de $\theta$, alors on pourra conclure que $\hat{\theta}_i$ est
sans biais. De m�me, on d�terminera lequel des trois estimateurs a la
plus faible variance selon le classement des variances empiriques.



\section{Premi�re approche: avec une boucle}
\index{boucle}
\label{simulation:boucle}

La fa�on la plus intuitive de mettre en {\oe}uvre cette �tude de
simulation en S consiste � utiliser une boucle \fonction{for}. Avec
cette approche, il est n�cessaire d'initialiser une matrice de 3
lignes et $N$ colonnes (ou l'inverse) dans laquelle seront stock�es
les valeurs des trois estimateurs pour chaque simulation. Une fois la
matrice remplie dans la boucle, il ne reste plus qu'� calculer la
moyenne et la variance par ligne pour obtenir les r�sultats souhait�s.

La figure \ref{fig:boucle} pr�sente un exemple de code ad�quat pour
r�aliser la simulation � l'aide d'une boucle.

\begin{figure}
  \centering
  \begin{framed}
\begin{lstlisting}
### Bonne habitude � prendre: stocker les constantes dans
### des variables faciles � modifier au lieu de les �crire
### explicitement dans le code.
size <- 100                # taille de chaque �chantillon
nsimul <- 10000            # nombre de simulations
theta <- 0                 # la valeur du param�tre

### Les lignes ci-dessous �viteront de faire deux additions
### 'nsimul' fois.
a <- theta - 0.5           # borne inf�rieure de l'uniforme
b <- theta + 0.5           # borne sup�rieure de l'uniforme

### Initialisation de la matrice dans laquelle seront
### stock�es les valeurs des estimateurs. On donne �galement
### des noms aux lignes de la matrice afin de facilement
### identifier les estimateurs.
x <- matrix(0, nrow=3, ncol=nsimul)
rownames(x) <- c("Moyenne", "Mediane", "Mi-etendue")

### Simulation comme telle.
for (i in 1:nsimul)
{
    u <- runif(size, a, b)
    x[1, i] <- mean(u)     # moyenne
    x[2, i] <- median(u)   # m�diane
    x[3, i] <- mean(range(u)) # mi-�tendue
}

### On peut maintenant calculer la moyenne et la variance
### par ligne.
rowMeans(x) - theta        # v�rification du biais
apply(x, 1, var)           # comparaison des variances
\end{lstlisting}
  \end{framed}
  \caption{Code pour la simulation utilisant une boucle \texttt{for}}
  \label{fig:boucle}
\end{figure}

\begin{figure}[htbp]
  \centering
  \begin{framed}
\begin{lstlisting}
simul1 <- function(nsimul, size, theta)
{
    a <- theta - 0.5
    b <- theta + 0.5

    x <- matrix(0, nrow=3, ncol=nsimul)
    rownames(x) <- c("Moyenne","Mediane","Mi-etendue")

    for (i in 1:nsimul)
    {
        u <- runif(size, a, b)
        x[1, i] <- mean(u)
        x[2, i] <- median(u)
        x[3, i] <- mean(range(u))
    }

    list(biais=rowMeans(x) - theta,
         variances=apply(x, 1, var))
}
\end{lstlisting}
  \end{framed}
  \caption{D�finition de la fonction \texttt{simul1}}
  \label{fig:simul1}
\end{figure}

Si l'on souhaite pouvoir ex�cuter le code de la figure
\ref{fig:boucle} facilement � l'aide d'une seule expression, il suffit
de placer l'ensemble du code dans une fonction. La fonction
\code{simul1} de la figure \ref{fig:simul1} reprend le code de la
figure \ref{fig:boucle}, sans les commentaires. On a alors:

\begin{Schunk}
\begin{Sinput}
> simul1(10000, 100, 0)
\end{Sinput}
\begin{Soutput}
$biais
      Moyenne       Mediane    Mi-etendue 
 4.067055e-04  6.074806e-04 -6.295365e-05 

$variances
     Moyenne      Mediane   Mi-etendue 
8.384350e-04 2.436228e-03 4.812775e-05 
\end{Soutput}
\end{Schunk}

\section{Seconde approche: avec \texttt{sapply}}
\label{simulation:sapply}

On le sait, les boucles sont inefficaces en S --- tout
particuli�rement dans S-Plus. Il est en g�n�ral plus efficace de
d�l�guer les boucles aux fonctions \fonction{lapply} et
\fonction{sapply} (section \ref{avance:lapply}), dont la syntaxe est
\begin{center}
  \code{lapply(x, FUN, ...)} \quad et \quad \code{sapply(x, FUN, ...)}.
\end{center}
Celles-ci appliquent la fonction \texttt{FUN} � tous les �l�ments de
la liste ou du vecteur \texttt{x} et retournent les r�sultats sous
forme de liste (\fonction{lapply}) ou, lorsque c'est possible, de
vecteur ou de matrice (\fonction{sapply}). Il est important de noter
que les valeurs successives de \code{x} seront pass�es comme
\emph{premier} argument � la fonction \code{FUN}. Le cas �ch�ant, les
autres arguments de \code{FUN} sont sp�cifi�s dans le
champ~`\argument{...}'.

Pour pouvoir utiliser ces fonctions dans le cadre d'une simulation
comme celle dont il est question ici, il s'agit de d�finir une
fonction qui fera tous les calculs pour une simulation, puis de la
passer � \fonction{sapply} pour obtenir les r�sultats de $N$
simulations. La figure \ref{fig:fun1} pr�sente une premi�re version
d'une telle fonction. On remarquera que l'argument \code{i} ne joue
aucun r�le dans la fonction.  Voici un exemple d'utilisation pour un
petit nombre (4) de simulations:

\begin{figure}[tbp]
  \centering
  \begin{framed}
\begin{lstlisting}
fun1 <- function(i, size, a, b)
{
    u <- runif(size, a, b)
    c(Moyenne=mean(u),
      Mediane=median(u),
      "Mi-etendue"=mean(range(u)))
}
\end{lstlisting}
  \end{framed}
  \caption{D�finition de la fonction \texttt{fun1}}
  \label{fig:fun1}
\end{figure}

\begin{Schunk}
\begin{Sinput}
> sapply(1:4, fun1, size = 10, a = -0.5, b = 0.5)
\end{Sinput}
\begin{Soutput}
                   [,1]          [,2]        [,3]      [,4]
Moyenne     0.053791370 -0.0621231376 -0.01328870 0.1545880
Mediane     0.137352108 -0.1115397780 -0.06042433 0.2528814
Mi-etendue -0.007453277  0.0002247883  0.04237688 0.1174229
\end{Soutput}
\end{Schunk}

On remarque donc que les r�sultats de chaque simulation se trouvent
dans les colonnes de la matrice obtenue avec \code{sapply}.

Pour compl�ter l'analyse, on englobe le tout dans une fonction
\code{simul2}, dont le code se trouve � la figure \ref{fig:simul2}:

\begin{figure}[tbp]
  \centering
  \begin{framed}
\begin{lstlisting}
simul2 <- function(nsimul, size, theta)
{
    a <- theta - 0.5
    b <- theta + 0.5

    x <- sapply(1:nsimul, fun1, size, a, b)

    list(biais=rowMeans(x) - theta,
         variances=apply(x, 1, var))
}
\end{lstlisting}
  \end{framed}
  \caption{D�finition de la fonction \texttt{simul2}}
  \label{fig:simul2}
\end{figure}

\begin{Schunk}
\begin{Sinput}
> simul2(10000, 100, 0)
\end{Sinput}
\begin{Soutput}
$biais
     Moyenne      Mediane   Mi-etendue 
3.172877e-04 2.759724e-04 3.247261e-05 

$variances
     Moyenne      Mediane   Mi-etendue 
8.029618e-04 2.390275e-03 4.881379e-05 
\end{Soutput}
\end{Schunk}

Il est g�n�ralement plus facile de d�boguer le code avec cette
approche.


\section{Variante de la seconde approche}
\label{simulation:replicate}

Une chose manque d'�l�gance dans la seconde approche: l'obligation
d'inclure un argument factice dans la fonction \code{fun1}.  La
fonction \R \fonction{replicate} (section \ref{avance:replicate}),
disponible dans \textsf{R} seulement, permet toutefois de passer outre
cette contrainte. En effet, cette fonction ex�cute un nombre donn� de
fois une expression quelconque.  Les fonctions \code{fun2} et
\code{simul3} des figures \ref{fig:fun2} et \ref{fig:simul3},
respectivement, sont des versions l�g�rement modifi�es de \code{fun1}
et \code{simul2} pour utilisation avec \fonction{replicate}.

On a alors
\begin{Schunk}
\begin{Sinput}
> simul3(10000, 100, 0)
\end{Sinput}
\begin{Soutput}
$biais
      Moyenne       Mediane    Mi-etendue 
-3.101468e-04 -2.104813e-04 -3.392078e-05 

$variances
     Moyenne      Mediane   Mi-etendue 
8.202952e-04 2.420604e-03 4.830223e-05 
\end{Soutput}
\end{Schunk}

\begin{figure}[tbp]
  \centering
  \begin{framed}
\begin{lstlisting}
fun2 <- function(size, a, b)
{
    u <- runif(size, a, b)
    c(Moyenne=mean(u),
      Mediane=median(u),
      "Mi-etendue"=mean(range(u)))
}
\end{lstlisting}
  \end{framed}
  \caption{D�finition de la fonction \texttt{fun2}}
  \label{fig:fun2}
\end{figure}

\begin{figure}[tbp]
  \centering
  \begin{framed}
\begin{lstlisting}
simul3 <- function(nsimul, size, theta)
{
    a <- theta - 0.5
    b <- theta + 0.5

    x <- replicate(nsimul, fun2(size, a, b))

    list(biais=rowMeans(x) - theta,
         variances=apply(x, 1, var))
}
\end{lstlisting}
  \end{framed}
  \caption{D�finition de la fonction \texttt{simul3}}
  \label{fig:simul3}
\end{figure}


\section{Comparaison des temps de calcul}
\label{simulation:temps}

A-t-on gagn� quoi que ce soit en termes de temps de calcul d'une
approche � l'autre? La fonction \fonction{system.time} de \textsf{R}
(ou \fonction{sys.time} de S-Plus) permet de mesurer le temps requis
pour l'ex�cution d'une expression.  Le premier r�sultat de
\fonction{system.time} est le temps CPU utilis� et le troisi�me, le
temps total �coul�. Sous Windows, les quatri�me et cinqui�me r�sultats
sont \texttt{NA}.

\begin{Schunk}
\begin{Sinput}
> system.time(simul1(10000, 100, 0))
\end{Sinput}
\begin{Soutput}
[1] 8.16 0.00 8.30 0.00 0.00
\end{Soutput}
\begin{Sinput}
> system.time(simul2(10000, 100, 0))
\end{Sinput}
\begin{Soutput}
[1] 7.85 0.05 8.04 0.00 0.00
\end{Soutput}
\begin{Sinput}
> system.time(simul3(10000, 100, 0))
\end{Sinput}
\begin{Soutput}
[1] 7.54 0.02 7.66 0.00 0.00
\end{Soutput}
\end{Schunk}

Les diff�rences, petites ici, peuvent �tre plus importantes lors de
grosses simulations et favoriser d'autant plus l'utilisation de la
fonction \code{replicate}.


\section{Gestion des fichiers}
\label{simulation:fichiers}

Pour un petit projet comme celui utilis� en exemple ici, il est simple
et pratique de placer tout le code informatique dans un seul fichier
de script. Pour un plus gros projet, cependant, il vaut souvent mieux
avoir recours � plusieurs fichiers diff�rents. Le pr�sent auteur
utilise pour sa part un fichier par fonction.

� des fins d'illustration, supposons que l'on utilise l'approche
de la section \ref{simulation:replicate} avec la fonction
\code{replicate} en \textsf{R} et que le code des fonctions
\code{fun2} et \code{simul3} est sauvegard� dans des fichiers
\code{fun2.R} et \code{simul3.R}, respectivement. Si l'on cr�e un
autre fichier, \texttt{go.R}, ne contenant que des expressions
\fonction{source} pour lire les autres fichiers, il est alors possible
de d�marrer des simulations en ex�cutant ce seul fichier. Dans notre
exemple, le fichier \texttt{go.R} contiendrait les lignes suivantes:
\begin{verbatim}
source("fun2.R")
source("simul3.R")
simul3(10000, 100, 0)
\end{verbatim}
Une simple commande
\begin{Sinput}
> source("go.R")
\end{Sinput}
ex�cutera alors une simulation compl�te.


\section{Ex�cution en lot}
\label{simulation:batch}


On peut acc�l�rer le traitement d'une simulation en l'ex�cutant en lot
--- ou mode \emph{batch} --- et ce, avec S-Plus comme avec \textsf{R}.
Dans ce mode, aucune interface graphique n'est d�marr�e et tous les
r�sultats sont redirig�s vers un fichier pour consultation ult�rieure.
Pour les simulations demandant un long temps de calcul, c'est tr�s
pratique.

Pour ex�cuter S-Plus ou \textsf{R} en lot sous Windows, ouvrir une
invite de commande (dans le menu Accessoires du menu D�marrer) puis se
d�placer (� l'aide de la commande \texttt{cd}) dans le dossier o� sont
sauvegard�s les fichiers de script. Avec S-Plus, il faut par la suite
ex�cuter la commande suivante:
\begin{verbatim}
C:\> Splus /BATCH go.S go.Sout
\end{verbatim}
Le troisi�me �l�ment de cette commande est le nom du fichier de script
contenant les expressions � ex�cuter et le quatri�me, le nom du
fichier dans lequel seront sauvegard�s les r�sultats. Ils peuvent
�videmment �tre diff�rents de ceux ci-dessus.

Avec \textsf{R}, la syntaxe est plut�t
\begin{verbatim}
C:\> R CMD BATCH go.R
\end{verbatim}
et les r�sultats seront plac�s par d�faut dans le fichier
\texttt{go.Rout}. Si Windows ne trouve pas l'ex�cutable de \textsf{R},
il faut sp�cifier le chemin d'acc�s complet, par exemple:
\begin{verbatim}
C:\> "c:\program files\R\R-2.2.0\bin\R" CMD BATCH go.R
\end{verbatim}

Depuis la version 6.2, S-Plus sous Windows contient un outil BATCH dans
le dossier S-Plus du menu D�marrer facilitant l'utilisation en lot. Il
suffit de remplir les champs appropri�s dans la bo�te de dialogue.


\section{Quelques remarques}
\label{simulation:rem}

\begin{enumerate}
\item La \Splus fonction \fonction{rownames} utilis�e dans la figure
  \ref{fig:boucle} existe seulement dans \textsf{R}. Dans S-Plus, on
  utilisera plut�t \fonction{row.names} ou \fonction{dimnames}.

\item Dans \Splus S-Plus, on peut calculer la variance par ligne ou
  par colonne d'une matrice avec les fonctions \fonction{rowVars} et
  \fonction{colVars}.

\item Le nombre de simulations, $N$, et la taille de l'�chantillon,
  $n$, ont tous deux un impact sur la qualit� des r�sultats, mais de
  mani�re diff�rente. Quand $n$ augmente, la pr�cision des estimateurs
  augmente. Ainsi, dans l'exemple ci-dessus, le biais et la variance
  des estimateurs de $\theta$ seront plus faibles. D'autre part,
  l'augmentation du nombre de simulations diminue l'impact des
  �chantillons al�atoires individuels et, de ce fait, am�liore la
  fiabilit� des conclusions de l'�tude.

\item Conclusion de l'�tude de simulation sur le biais et la variance
  des trois estimateurs de la moyenne d'une loi uniforme: les trois
  estimateurs sont sans biais et la mi-�tendue a la plus faible
  variance. On peut d'ailleurs prouver que, pour $n$ impair,
  \begin{align*}
    \var{\hat{\theta}_1} &= \frac{1}{12 n} \\
    \var{\hat{\theta}_2} &= \frac{1}{4n + 2} \\
    \var{\hat{\theta}_3} &= \frac{1}{2(n + 1)(n + 2)}
  \end{align*}
  et donc
  \begin{displaymath}
    \var{\hat{\theta}_3} \leq \var{\hat{\theta}_1} \leq \var{\hat{\theta}_2}
  \end{displaymath}
  pour tout $n \geq 2$.
\end{enumerate}

\index{simulation!planification|)}

%%% Local Variables:
%%% mode: latex
%%% TeX-master: "introduction_programation_S"
%%% End:


\appendix
\chapter{GNU Emacs et ESS: la base}
\index{Emacs|(}
\label{ess}

Emacs est l'�diteur de texte des �diteurs de texte. Bien que
d'abord et avant tout un �diteur pour programmeurs (avec des modes
sp�ciaux pour une multitude de langages diff�rents), c'est �galement
un environnement id�al pour travailler sur des documents \LaTeX,
interagir avec \textsf{R}, S-Plus, SAS ou SQL, ou m�me pour lire son
courrier �lectronique.

Le pr�sent auteur distribue une version simple � installer et
augment�e de quelques ajouts de la plus r�cente version de GNU Emacs
pour Windows.  Consulter le site Internet
\begin{quote}
  \url{http://vgoulet.act.ulaval.ca/ressources/#Emacs}
\end{quote}

Cette annexe passe en revue les quelques commandes essentielles �
con\-na�tre pour commencer � travailler avec GNU Emacs et le mode ESS.
L'ouvrage de \cite{LearningEmacs} constitue une excellente r�f�rence
pour l'apprentissage plus pouss� de l'�diteur.


\section{Mise en contexte}

Emacs est le logiciel �tendard du projet GNU (�\emph{GNU is not
  Unix}�), dont le principal commanditaire est la \emph{Free Software
  Foundation}.

\begin{itemize}
\item Distribu� sous la GNU \emph{General Public License} (GPL),
  donc gratuit, ou �libre�.
\item Le nom provient de �\emph{Editing MACroS}�.
\item La premi�re version de Emacs a �t� �crite par Richard M.\ 
  Stallman, pr�sident de la FSF.
\end{itemize}

\section{Configuration de l'�diteur}
\index{Emacs!configuration}

Une des grandes forces de Emacs est d'�tre configurable � l'envi.

\begin{itemize}
\item Depuis la version 21, le menu \texttt{Customize} rend la
  configuration ais�e.
\item Une grande part de la configuration provient du fichier
  \texttt{.emacs}:
  \begin{itemize}
  \item nomm� \texttt{.emacs} sous Linux et Unix, Windows 2000 et
    Windows XP;
  \item sous Windows 95/98/Me, utiliser plut�t \texttt{\_emacs}.
  \end{itemize}
\end{itemize}


\section{\emph{Emacs-ismes} et \emph{Unix-ismes}}

\begin{itemize}
\item Un \emph{buffer} contient un fichier ouvert (�\emph{visited}�).
  �quivalent � une fen�tre dans Windows.
\item Le \emph{minibuffer} est la r�gion au bas de l'�cran Emacs o�
  l'on entre des commandes et re�oit de l'information de Emacs.
\item La ligne de mode (�\emph{mode line}�) est le s�parateur
  horizontal contenant diverses informations sur le fichier ouvert et
  l'�tat de Emacs.
\item Toutes les fonctionnalit�s de Emacs correspondent � une commande
  pouvant �tre tap�e dans le \emph{minibuffer}.  \texttt{M-x} d�marre
  l'interpr�teur (ou invite) de commandes.
\item Dans les d�finitions de raccourcis claviers:
  \begin{itemize}
  \item \texttt{C} est la touche \texttt{Ctrl} (\texttt{Control});
  \item \texttt{M} est la touche \texttt{Meta}, qui correspond � la
    touche \texttt{Alt} de gauche sur un PC;
  \item \texttt{ESC} est la touche \texttt{�chap} (\texttt{Esc}) et
    est �quivalente � \texttt{Meta};
  \item \texttt{SPC} est la barre d'espacement;
  \item \texttt{RET} est la touche Entr�e.
  \end{itemize}
\item Le caract�re \verb=~= repr�sente le dossier vers lequel pointe
  la variable d'environnement \texttt{\$HOME} (Unix) ou
  \texttt{\%HOME\%} (Windows).
\item La barre oblique (\texttt{/}) est utilis�e pour s�parer les
  dossiers dans les chemins d'acc�s aux fichiers, m�me sous Windows.
\item En g�n�ral, il est possible d'appuyer sur \texttt{TAB} dans le
  \emph{minibuffer} pour compl�ter les noms de fichiers ou de
  commandes.
\end{itemize}


\section{Commandes d'�dition de base}

Il n'est pas vain de lire le tutoriel de Emacs, que l'on d�marre avec
\begin{quote}
  \texttt{C-h t}
\end{quote}
  
Pour une liste plus exhaustive des commandes Emacs les plus
importantes, consulter la \emph{GNU Emacs Reference Card}, dans le
fichier
\begin{quote}
  \url{.../emacs-21.x/etc/refcard.ps}
\end{quote}

\enlargethispage{5mm}
\begin{itemize}
\item Pour cr�er un nouveau fichier, ouvrir un fichier n'existant
  pas.\index{Emacs!nouveau fichier}
\item Principales commandes d'�dition avec, entre parenth�ses, le nom
  de la commande correspondant au raccourci clavier:
  \begin{ttscript}{C-x C-w}
    \raggedright
  \item[\emacs{C-x C-f}] ouvrir un fichier (\texttt{find-file})
  \item[\emacs{C-x C-s}] sauvegarder
    (\texttt{save-buffer})\index{Emacs!sauvegarder}
  \item[\emacs{C-x C-w}] sauvegarder sous
    (\texttt{write-file})\index{Emacs!sauvegarder sous}
  \item[\emacs{C-x k}] fermer un fichier (\texttt{kill-buffer}).
  \item[\emacs{C-x C-c}] quitter Emacs
    (\texttt{save-buffers-kill-emacs}) \\[\baselineskip]
  \item[\emacs{C-g}] bouton de panique: quitter!
    (\texttt{keyboard-quit})
  \item[\emacs{C-\_}] annuler (pratiquement illimit�); aussi
    \emacs{C-x u} (\texttt{undo}) \\[\baselineskip]
  \item[\emacs{C-s}] recherche incr�mentale avant
    (\texttt{isearch-forward})
  \item[\emacs{C-r}] Recherche incr�mentale arri�re
    (\texttt{isearch-backward})
  \item[\emacs{M-\%}] rechercher et remplacer
    (\texttt{query-replace})\index{Emacs!rechercher et remplacer}
    \\[\baselineskip]
  \item[\emacs{C-x b}] changer de \emph{buffer}
    (\texttt{switch-buffer})
  \item[\emacs{C-x 2}] s�parer l'�cran en deux fen�tres
    (\texttt{split-window-vertically})
  \item[\emacs{C-x 1}] conserver uniquement la fen�tre courante
    (\texttt{delete-other-windows})
  \item[\emacs{C-x 0}] fermer la fen�tre courante
    (\texttt{delete-window})
  \item[\emacs{C-x o}] aller vers une autre fen�tre lorsqu'il y en a
    plus d'une (\texttt{other-window})
  \end{ttscript}
\end{itemize}


\section{S�lection de texte}
\index{Emacs!s�lection}  
\label{ess:selection}

La s�lection de texte fonctionne diff�remment du standard Windows.

\begin{itemize}
\item Les raccourcis clavier standards sous Emacs sont:
  \begin{ttscript}{C-SPC}
    \raggedright
  \item[\emacs{C-SPC}] d�bute la s�lection (\texttt{set-mark-command})
  \item[\emacs{C-w}] couper la s�lection (\texttt{kill-region})
  \item[\emacs{M-w}] copier la s�lection (\texttt{kill-ring-save})
  \item[\emacs{C-y}] coller (\texttt{yank})
  \item[\emacs{M-y}] remplacer le dernier texte coll� par la
    s�lection pr�c�dente (\texttt{yank-pop})
  \end{ttscript}
\item Il existe quelques extensions de Emacs permettant d'utiliser les
  raccourcis clavier usuels de Windows (\texttt{C-c}, \texttt{C-x},
  \texttt{C-v}); voir
  \url{http://www.emacswiki.org/cgi-bin/wiki/CuaMode}.
\end{itemize}
\index{Emacs|)}

\section{Mode ESS}
\index{ESS|(}

Le mode ESS (\emph{Emacs Speaks Statistics}) est un mode pour
interagir avec des logiciels statistiques (S-Plus, \textsf{R}, SAS,
etc.) depuis Emacs. Ce mode est install� dans la version modifi�e de
GNU Emacs distribu�e dans le site Internet
\url{http://vgoulet.act.ulaval.ca/pub/emacs/}

\begin{itemize}
\item Voir le fichier
  \begin{quote}
    \texttt{.../emacs-21.x/site-lisp/ess/doc/html/index.html}
  \end{quote}
  pour la documentation compl�te.
\item Deux modes mineurs: \texttt{ESS} pour les fichiers de script
  (code source) et \texttt{iESS} pour l'invite de commande.
\item Une fois install�, le mode mineur \texttt{ESS} s'active
  automatiquement en �ditant des fichiers avec l'extension \texttt{.S}
  ou \texttt{.R}.
\item Commandes les plus fr�quemment employ�es lors de l'�dition d'un
  fichier de script (mode \texttt{ESS}):
  \begin{ttscript}{C-c C-c}
    \raggedright
  \item[\ess{C-c C-n}] �value la ligne sous le curseur dans le
    processus S (\texttt{ess-eval-line-and-step})
  \item[\ess{C-c C-r}] �value la r�gion s�lectionn�e dans le
    processus S (\texttt{ess-eval-region})
  \item[\ess{C-c C-f}] �value le code de la fonction courante dans
    le processus S (\texttt{ess-eval-function})
  \item[\ess{C-c C-l}] �value le code du fichier courant dans le
    processus S (\texttt{ess-load-file})
  \item[\ess{C-c C-v}] aide sur une commande S
    (\texttt{ess-display-help-on-object})
  \item[\ess{C-c C-s}] changer de processus (utile si l'on a plus
    d'un processus S actif)
  \end{ttscript}
\item Pour d�marrer un processus S et activer le mode mineur
  \texttt{iESS}, entrer l'une des commandes \texttt{S}, \texttt{Sqpe}
  ou \texttt{R} dans l'invite de commande de Emacs (voir aussi
  l'annexe \ref{s-plus_windows}). Par exemple, pour d�marrer un
  processus \textsf{R} � l'int�rieur m�me de Emacs, on fera
  \begin{quote}
    \ttfamily M-x R RET
  \end{quote}
\item Commandes le plus fr�quemment employ�es � la ligne de commande
  (mode \texttt{iESS}):
  \begin{ttscript}{C-c C-c}
    \raggedright
  \item[\ess{C-c C-e}] replacer la derni�re ligne au bas de la
    fen�tre (\texttt{comint-show-maximum-output})
  \item[\ess{M-h}] s�lectionner le r�sultat de la derni�re commande
    (\texttt{mark-paragraph})
  \item[\ess{C-c C-o}] effacer le r�sultat de la derni�re commande
    (\texttt{comint-delete-output})
  \item[\ess{C-c C-v}] aide sur une commande S
    (\texttt{ess-display-help-on-object})
  \item[\ess{C-c C-q}] terminer le processus S
    (\texttt{ess-quit})
  \end{ttscript}
\end{itemize}
\index{ESS|)}


\section{Session de travail type}
\label{ess:session_travail}

On d�crit, dans cette section, les diff�rentes �tapes et principaux
raccourcis claviers d'une session de travail type avec S-Plus ou
\textsf{R}, Emacs et ESS.

\begin{enumerate}
\item D�terminer le dossier de travail et le cr�er au besoin.
  Normalement, le dossier de travail pour S-Plus ou \textsf{R} sera le
  m�me que celui o� les fichiers de script sont sauvegard�s. Si l'on
  pr�voit sauvegarder des objets S, il est important de choisir un
  nouveau dossier pour le projet.
\item Lancer Emacs et d�marrer un processus S dans le dossier de
  travail d�termin� ci-dessus.
  \begin{itemize}
  \item \code{M-x R RET} ou \code{M-x Sqpe RET}
  \end{itemize}
\item Ouvrir un fichier de script dans lequel on sauvegardera le code
  source.
  \begin{itemize}
  \item \emacs{C-x C-f} pour ouvrir un fichier existant ou un nouveau
    fichier.
  \end{itemize}
\item Composer le code. Lors de cette �tape, on se d�placera souvent
  du fichier de script � la ligne de commande afin d'essayer diverses
  expressions. On ex�cutera �galement des parties seulement du code se
  trouvant dans le fichier de script.
  \begin{itemize}
  \item \emacs{C-x o} pour se d�placer de la fen�tre de script � la
    ligne de commande et vice-versa.
  \item \ess{C-c C-e} pour replacer le curseur � la ligne de commande
    et au bas de la fen�tre.
  \item \ess{C-c C-o} pour effacer le r�sultat de la derni�re
    commande, surtout s'il est tr�s long.
  \item \ess{C-c C-n} pour ex�cuter une ligne du fichier de script.
  \item \ess{C-c C-r} pour ex�cuter une r�gion du fichier de script
    (voir la section \ref{ess:selection} pour s�lectionner une r�gion).
  \item \ess{C-c C-f} pour d�finir une fonction.
  \end{itemize}
\item Sauvegarder le fichier de script. Les quatri�me et cinqui�me
  caract�res de la ligne de mode changent de \verb|**| � \verb|--|.
  \begin{itemize}
  \item \emacs{C-x C-s} pour sauvegarder le fichier de script.
  \end{itemize}
\item Il est �galement possible de sauvegarder le texte de la session
  interactive (la ligne de commande). Il est recommand� de nommer de
  telles transcriptions de la session de travail avec une extension
  \texttt{.St} ou \texttt{.Rt}. En effet, ESS poss�de un mode sp�cial
  pour les transcriptions. Consulter � cet effet le chapitre 5 de la
  documentation de ESS.
  \begin{itemize}
  \item \emacs{C-x C-s} pour sauvegarder le texte de la session.
  \end{itemize}
\item Quitter le processus S. Utiliser la commande ESS pour ce faire
  puisque ESS se chargera de fermer tous les fichiers associ�s au
  processus S. � moins qu'il ne contienne des objets importants,
  l'espace de travail \R en \textsf{R} n'est habituellement pas
  sauvegard�.
  \begin{itemize}
  \item \ess{C-c C-q} pour quitter le processus S.
  \end{itemize}
\end{enumerate}


%%% Local Variables: 
%%% mode: latex
%%% TeX-master: "introduction_programmation_S"
%%% End: 

\chapter{Utilisation de ESS et S-Plus sous Windows}
\index{Emacs!et S-Plus|(}
\label{s-plus_windows}

L'utilisation de \textsf{R} et S-Plus avec ESS dans Emacs est
virtuellement identique sous Unix. Sous Windows, la proc�dure est
exactement la m�me que sous Unix pour \textsf{R}, mais l'interface
avec S-Plus est l�g�rement plus compliqu�e.

Avant toute chose, il faut s'assurer d'avoir une installation de
Emacs, ESS et S-Plus fonctionnelle. L'installation de la version
modifi�e de Emacs distribu�es dans le site Internet
\begin{quote}
  \url{http://vgoulet.act.ulaval.ca/pub/emacs/}
\end{quote}
devrait permettre de satisfaire cette exigence rapidement.

Il y a deux fa�ons de travailler avec S-Plus depuis Emacs sous
Windows: tout dans Emacs ou une combinaison de Emacs et de l'interface
graphique de S-Plus.


\section{Tout dans Emacs}

Cette approche est similaire � celle favoris�e sous Unix ainsi qu'avec
\textsf{R}. Un processus S-Plus est d�marr� � l'int�rieur m�me de
Emacs, un fichier de script (habituellement avec une extension
\texttt{.S}) est ouvert dans Emacs et les lignes de ce fichier sont
ex�cut�es dans le processus S-Plus. La fen�tre Emacs est alors scind�e
en deux. C'est l'approche pr�n�e � la section
\ref{presentation:strategies}.

Le truc consiste ici � utiliser non pas l'ex�cutable
\texttt{splus.exe} (qui est l'interface graphique), mais plut�t
l'interface en ligne de commande, plus simple et rapide. L'ex�cutable
est \texttt{sqpe.exe}. Pour d�marrer une session S-Plus dans Emacs, on
fera donc

\begin{quote}
  \ttfamily M-x Sqpe RET
\end{quote}
Lorsque demand�, on sp�cifie le dossier de travail. Une fois l'invite
de commande S-Plus obtenue, on pourra ex�cuter des lignes du fichier
de script dans le processus S-Plus avec \texttt{C-c C-n}, \texttt{C-c
  C-f}, etc.

Il y a toutefois un os avec cette approche: aucun p�riph�rique
graphique n'est disponible. Sauf depuis la version 6.1 de S-Plus: on
peut utiliser un p�riph�rique graphique Java. Afin de pouvoir
l'utiliser, il faut ex�cuter les deux lignes suivantes \emph{avant} de
cr�er un graphique:
\begin{Schunk}
  \begin{Sinput}
> library(winjava)
> java.graph()
  \end{Sinput}
\end{Schunk}

Il est possible d'automatiser ce processus en sauvegardant ces deux
lignes dans un fichier nomm� \texttt{S.init} dans le dossier de
travail. Le contenu de ce fichier sera ex�cut� � chaque fois que
S-Plus sera d�marr� dans ce dossier.


\section[Combinaison Emacs et S-Plus GUI]{Combinaison Emacs et interface graphique de S-Plus}

Cette option est moins �l�gante que la pr�c�dente, mais certains
pourraient lui voir comme avantage d'utiliser l'interface graphique
(GUI) de S-Plus. En fin de compte, la proc�dure ci-dessous revient �
remplacer par Emacs la fen�tre d'�dition de script incluse dans
S-Plus.

En faisant
\begin{quote}
  \ttfamily M-x S RET
\end{quote}
� l'int�rieur de Emacs, une nouvelle session graphique de S-Plus sera
d�marr�e (il faut �tre patient, les n�gociations entre les deux
logiciels peuvent prendre du temps). On se retrouve donc avec deux
fen�tres: une pour Emacs et une pour S-Plus.

Ouvrir un fichier de script dans Emacs et ex�cuter les lignes de comme
ci-dessus. Les lignes de code seront ex�cut�es dans l'interface
graphique. En d'autres mots, le code source se trouve dans une fen�tre
(Emacs) et les r�sultats de ce code source dans une autre (S-Plus). Il
faut bien disposer les fen�tres c�te � c�te pour que cette strat�gie
se r�v�le minimalement efficace.

L'information ci-dessus se trouve dans la documentation de ESS.

\index{Emacs!et S-Plus|)}

%%% Local Variables: 
%%% mode: latex
%%% TeX-master: "introduction_programmation_S"
%%% End: 

\selectlanguage{english}        % retour au fran�ais � la fin

\chapter{GNU Free Documentation License}
\label{fdl}

 \begin{center}

       Version 1.2, November 2002


 Copyright \copyright 2000, 2001, 2002  Free Software Foundation, Inc.
 
 \bigskip
 
     51 Franklin Street, Fifth Floor, Boston, MA  02110-1301, USA
  
 \bigskip
 
 Everyone is permitted to copy and distribute verbatim copies
 of this license document, but changing it is not allowed.
\end{center}


\section*{Preamble}

The purpose of this License is to make a manual, textbook, or other
functional and useful document "free" in the sense of freedom: to
assure everyone the effective freedom to copy and redistribute it,
with or without modifying it, either commercially or noncommercially.
Secondarily, this License preserves for the author and publisher a way
to get credit for their work, while not being considered responsible
for modifications made by others.

This License is a kind of "copyleft", which means that derivative
works of the document must themselves be free in the same sense.  It
complements the GNU General Public License, which is a copyleft
license designed for free software.

We have designed this License in order to use it for manuals for free
software, because free software needs free documentation: a free
program should come with manuals providing the same freedoms that the
software does.  But this License is not limited to software manuals;
it can be used for any textual work, regardless of subject matter or
whether it is published as a printed book.  We recommend this License
principally for works whose purpose is instruction or reference.


\section{APPLICABILITY AND DEFINITIONS}

This License applies to any manual or other work, in any medium, that
contains a notice placed by the copyright holder saying it can be
distributed under the terms of this License.  Such a notice grants a
world-wide, royalty-free license, unlimited in duration, to use that
work under the conditions stated herein.  The \textbf{"Document"}, below,
refers to any such manual or work.  Any member of the public is a
licensee, and is addressed as \textbf{"you"}.  You accept the license if you
copy, modify or distribute the work in a way requiring permission
under copyright law.

A \textbf{"Modified Version"} of the Document means any work containing the
Document or a portion of it, either copied verbatim, or with
modifications and/or translated into another language.

A \textbf{"Secondary Section"} is a named appendix or a front-matter section of
the Document that deals exclusively with the relationship of the
publishers or authors of the Document to the Document's overall subject
(or to related matters) and contains nothing that could fall directly
within that overall subject.  (Thus, if the Document is in part a
textbook of mathematics, a Secondary Section may not explain any
mathematics.)  The relationship could be a matter of historical
connection with the subject or with related matters, or of legal,
commercial, philosophical, ethical or political position regarding
them.

The \textbf{"Invariant Sections"} are certain Secondary Sections whose titles
are designated, as being those of Invariant Sections, in the notice
that says that the Document is released under this License.  If a
section does not fit the above definition of Secondary then it is not
allowed to be designated as Invariant.  The Document may contain zero
Invariant Sections.  If the Document does not identify any Invariant
Sections then there are none.

The \textbf{"Cover Texts"} are certain short passages of text that are listed,
as Front-Cover Texts or Back-Cover Texts, in the notice that says that
the Document is released under this License.  A Front-Cover Text may
be at most 5 words, and a Back-Cover Text may be at most 25 words.

A \textbf{"Transparent"} copy of the Document means a machine-readable copy,
represented in a format whose specification is available to the
general public, that is suitable for revising the document
straightforwardly with generic text editors or (for images composed of
pixels) generic paint programs or (for drawings) some widely available
drawing editor, and that is suitable for input to text formatters or
for automatic translation to a variety of formats suitable for input
to text formatters.  A copy made in an otherwise Transparent file
format whose markup, or absence of markup, has been arranged to thwart
or discourage subsequent modification by readers is not Transparent.
An image format is not Transparent if used for any substantial amount
of text.  A copy that is not "Transparent" is called \textbf{"Opaque"}.

%% Le mot PostScript d�passe largement de la marge
\begin{sloppypar}
  Examples of suitable formats for Transparent copies include plain
  ASCII without markup, Texinfo input format, LaTeX input format, SGML
  or XML using a publicly available DTD, and standard-conforming
  simple HTML, PostScript or PDF designed for human modification.
  Examples of transparent image formats include PNG, XCF and JPG.
  Opaque formats include proprietary formats that can be read and
  edited only by proprietary word processors, SGML or XML for which
  the DTD and/or processing tools are not generally available, and the
  machine-generated HTML, PostScript or PDF produced by some word
  processors for output purposes only.
\end{sloppypar}

The \textbf{"Title Page"} means, for a printed book, the title page itself,
plus such following pages as are needed to hold, legibly, the material
this License requires to appear in the title page.  For works in
formats which do not have any title page as such, "Title Page" means
the text near the most prominent appearance of the work's title,
preceding the beginning of the body of the text.

A section \textbf{"Entitled XYZ"} means a named subunit of the Document whose
title either is precisely XYZ or contains XYZ in parentheses following
text that translates XYZ in another language.  (Here XYZ stands for a
specific section name mentioned below, such as \textbf{"Acknowledgements"},
\textbf{"Dedications"}, \textbf{"Endorsements"}, or \textbf{"History"}.)  
To \textbf{"Preserve the Title"}
of such a section when you modify the Document means that it remains a
section "Entitled XYZ" according to this definition.

The Document may include Warranty Disclaimers next to the notice which
states that this License applies to the Document.  These Warranty
Disclaimers are considered to be included by reference in this
License, but only as regards disclaiming warranties: any other
implication that these Warranty Disclaimers may have is void and has
no effect on the meaning of this License.


\section{VERBATIM COPYING}

You may copy and distribute the Document in any medium, either
commercially or noncommercially, provided that this License, the
copyright notices, and the license notice saying this License applies
to the Document are reproduced in all copies, and that you add no other
conditions whatsoever to those of this License.  You may not use
technical measures to obstruct or control the reading or further
copying of the copies you make or distribute.  However, you may accept
compensation in exchange for copies.  If you distribute a large enough
number of copies you must also follow the conditions in section 3.

You may also lend copies, under the same conditions stated above, and
you may publicly display copies.


\section{COPYING IN QUANTITY}

If you publish printed copies (or copies in media that commonly have
printed covers) of the Document, numbering more than 100, and the
Document's license notice requires Cover Texts, you must enclose the
copies in covers that carry, clearly and legibly, all these Cover
Texts: Front-Cover Texts on the front cover, and Back-Cover Texts on
the back cover.  Both covers must also clearly and legibly identify
you as the publisher of these copies.  The front cover must present
the full title with all words of the title equally prominent and
visible.  You may add other material on the covers in addition.
Copying with changes limited to the covers, as long as they preserve
the title of the Document and satisfy these conditions, can be treated
as verbatim copying in other respects.

If the required texts for either cover are too voluminous to fit
legibly, you should put the first ones listed (as many as fit
reasonably) on the actual cover, and continue the rest onto adjacent
pages.

If you publish or distribute Opaque copies of the Document numbering
more than 100, you must either include a machine-readable Transparent
copy along with each Opaque copy, or state in or with each Opaque copy
a computer-network location from which the general network-using
public has access to download using public-standard network protocols
a complete Transparent copy of the Document, free of added material.
If you use the latter option, you must take reasonably prudent steps,
when you begin distribution of Opaque copies in quantity, to ensure
that this Transparent copy will remain thus accessible at the stated
location until at least one year after the last time you distribute an
Opaque copy (directly or through your agents or retailers) of that
edition to the public.

It is requested, but not required, that you contact the authors of the
Document well before redistributing any large number of copies, to give
them a chance to provide you with an updated version of the Document.


\section{MODIFICATIONS}

You may copy and distribute a Modified Version of the Document under
the conditions of sections 2 and 3 above, provided that you release
the Modified Version under precisely this License, with the Modified
Version filling the role of the Document, thus licensing distribution
and modification of the Modified Version to whoever possesses a copy
of it.  In addition, you must do these things in the Modified Version:

\begin{itemize}
\item[A.] 
   Use in the Title Page (and on the covers, if any) a title distinct
   from that of the Document, and from those of previous versions
   (which should, if there were any, be listed in the History section
   of the Document).  You may use the same title as a previous version
   if the original publisher of that version gives permission.
   
\item[B.]
   List on the Title Page, as authors, one or more persons or entities
   responsible for authorship of the modifications in the Modified
   Version, together with at least five of the principal authors of the
   Document (all of its principal authors, if it has fewer than five),
   unless they release you from this requirement.
   
\item[C.]
   State on the Title page the name of the publisher of the
   Modified Version, as the publisher.
   
\item[D.]
   Preserve all the copyright notices of the Document.
   
\item[E.]
   Add an appropriate copyright notice for your modifications
   adjacent to the other copyright notices.
   
\item[F.]
   Include, immediately after the copyright notices, a license notice
   giving the public permission to use the Modified Version under the
   terms of this License, in the form shown in the Addendum below.
   
\item[G.]
   Preserve in that license notice the full lists of Invariant Sections
   and required Cover Texts given in the Document's license notice.
   
\item[H.]
   Include an unaltered copy of this License.
   
\item[I.]
   Preserve the section Entitled "History", Preserve its Title, and add
   to it an item stating at least the title, year, new authors, and
   publisher of the Modified Version as given on the Title Page.  If
   there is no section Entitled "History" in the Document, create one
   stating the title, year, authors, and publisher of the Document as
   given on its Title Page, then add an item describing the Modified
   Version as stated in the previous sentence.
   
\item[J.]
   Preserve the network location, if any, given in the Document for
   public access to a Transparent copy of the Document, and likewise
   the network locations given in the Document for previous versions
   it was based on.  These may be placed in the "History" section.
   You may omit a network location for a work that was published at
   least four years before the Document itself, or if the original
   publisher of the version it refers to gives permission.
   
\item[K.]
   For any section Entitled "Acknowledgements" or "Dedications",
   Preserve the Title of the section, and preserve in the section all
   the substance and tone of each of the contributor acknowledgements
   and/or dedications given therein.
   
\item[L.]
   Preserve all the Invariant Sections of the Document,
   unaltered in their text and in their titles.  Section numbers
   or the equivalent are not considered part of the section titles.
   
\item[M.]
   Delete any section Entitled "Endorsements".  Such a section
   may not be included in the Modified Version.
   
\item[N.]
   Do not retitle any existing section to be Entitled "Endorsements"
   or to conflict in title with any Invariant Section.
   
\item[O.]
   Preserve any Warranty Disclaimers.
\end{itemize}

If the Modified Version includes new front-matter sections or
appendices that qualify as Secondary Sections and contain no material
copied from the Document, you may at your option designate some or all
of these sections as invariant.  To do this, add their titles to the
list of Invariant Sections in the Modified Version's license notice.
These titles must be distinct from any other section titles.

You may add a section Entitled "Endorsements", provided it contains
nothing but endorsements of your Modified Version by various
parties--for example, statements of peer review or that the text has
been approved by an organization as the authoritative definition of a
standard.

You may add a passage of up to five words as a Front-Cover Text, and a
passage of up to 25 words as a Back-Cover Text, to the end of the list
of Cover Texts in the Modified Version.  Only one passage of
Front-Cover Text and one of Back-Cover Text may be added by (or
through arrangements made by) any one entity.  If the Document already
includes a cover text for the same cover, previously added by you or
by arrangement made by the same entity you are acting on behalf of,
you may not add another; but you may replace the old one, on explicit
permission from the previous publisher that added the old one.

The author(s) and publisher(s) of the Document do not by this License
give permission to use their names for publicity for or to assert or
imply endorsement of any Modified Version.


\section{COMBINING DOCUMENTS}

You may combine the Document with other documents released under this
License, under the terms defined in section 4 above for modified
versions, provided that you include in the combination all of the
Invariant Sections of all of the original documents, unmodified, and
list them all as Invariant Sections of your combined work in its
license notice, and that you preserve all their Warranty Disclaimers.

The combined work need only contain one copy of this License, and
multiple identical Invariant Sections may be replaced with a single
copy.  If there are multiple Invariant Sections with the same name but
different contents, make the title of each such section unique by
adding at the end of it, in parentheses, the name of the original
author or publisher of that section if known, or else a unique number.
Make the same adjustment to the section titles in the list of
Invariant Sections in the license notice of the combined work.

In the combination, you must combine any sections Entitled "History"
in the various original documents, forming one section Entitled
"History"; likewise combine any sections Entitled "Acknowledgements",
and any sections Entitled "Dedications".  You must delete all sections
Entitled "Endorsements".


\section{COLLECTIONS OF DOCUMENTS}

You may make a collection consisting of the Document and other documents
released under this License, and replace the individual copies of this
License in the various documents with a single copy that is included in
the collection, provided that you follow the rules of this License for
verbatim copying of each of the documents in all other respects.

You may extract a single document from such a collection, and distribute
it individually under this License, provided you insert a copy of this
License into the extracted document, and follow this License in all
other respects regarding verbatim copying of that document.


\section{AGGREGATION WITH INDEPENDENT WORKS}

A compilation of the Document or its derivatives with other separate
and independent documents or works, in or on a volume of a storage or
distribution medium, is called an "aggregate" if the copyright
resulting from the compilation is not used to limit the legal rights
of the compilation's users beyond what the individual works permit.
When the Document is included in an aggregate, this License does not
apply to the other works in the aggregate which are not themselves
derivative works of the Document.

If the Cover Text requirement of section 3 is applicable to these
copies of the Document, then if the Document is less than one half of
the entire aggregate, the Document's Cover Texts may be placed on
covers that bracket the Document within the aggregate, or the
electronic equivalent of covers if the Document is in electronic form.
Otherwise they must appear on printed covers that bracket the whole
aggregate.


\section{TRANSLATION}

Translation is considered a kind of modification, so you may
distribute translations of the Document under the terms of section 4.
Replacing Invariant Sections with translations requires special
permission from their copyright holders, but you may include
translations of some or all Invariant Sections in addition to the
original versions of these Invariant Sections.  You may include a
translation of this License, and all the license notices in the
Document, and any Warranty Disclaimers, provided that you also include
the original English version of this License and the original versions
of those notices and disclaimers.  In case of a disagreement between
the translation and the original version of this License or a notice
or disclaimer, the original version will prevail.

If a section in the Document is Entitled "Acknowledgements",
"Dedications", or "History", the requirement (section 4) to Preserve
its Title (section 1) will typically require changing the actual
title.


\section{TERMINATION}

You may not copy, modify, sublicense, or distribute the Document except
as expressly provided for under this License.  Any other attempt to
copy, modify, sublicense or distribute the Document is void, and will
automatically terminate your rights under this License.  However,
parties who have received copies, or rights, from you under this
License will not have their licenses terminated so long as such
parties remain in full compliance.


\section{FUTURE REVISIONS OF THIS LICENSE}

The Free Software Foundation may publish new, revised versions
of the GNU Free Documentation License from time to time.  Such new
versions will be similar in spirit to the present version, but may
differ in detail to address new problems or concerns.  See
http://www.gnu.org/copyleft/.

Each version of the License is given a distinguishing version number.
If the Document specifies that a particular numbered version of this
License "or any later version" applies to it, you have the option of
following the terms and conditions either of that specified version or
of any later version that has been published (not as a draft) by the
Free Software Foundation.  If the Document does not specify a version
number of this License, you may choose any version ever published (not
as a draft) by the Free Software Foundation.


\section*{ADDENDUM: How to use this License for your documents}
\addcontentsline{toc}{section}{ADDENDUM: How to use this License for
  your documents}

To use this License in a document you have written, include a copy of
the License in the document and put the following copyright and
license notices just after the title page:

\bigskip
\begin{quote}
    Copyright \copyright  YEAR  YOUR NAME.
    Permission is granted to copy, distribute and/or modify this document
    under the terms of the GNU Free Documentation License, Version 1.2
    or any later version published by the Free Software Foundation;
    with no Invariant Sections, no Front-Cover Texts, and no Back-Cover Texts.
    A copy of the license is included in the section entitled "GNU
    Free Documentation License".
\end{quote}
\bigskip
    
If you have Invariant Sections, Front-Cover Texts and Back-Cover Texts,
replace the "with...Texts." line with this:

\bigskip
\begin{quote}
    with the Invariant Sections being LIST THEIR TITLES, with the
    Front-Cover Texts being LIST, and with the Back-Cover Texts being LIST.
\end{quote}
\bigskip
    
If you have Invariant Sections without Cover Texts, or some other
combination of the three, merge those two alternatives to suit the
situation.

If your document contains nontrivial examples of program code, we
recommend releasing these examples in parallel under your choice of
free software license, such as the GNU General Public License,
to permit their use in free software.

\selectlanguage{francais}

%%% Local Variables:
%%% mode: latex
%%% TeX-master: "introduction_programmation_S"
%%% End:

\chapter{Réponses des exercices}
\label{chap:reponses}

\begingroup

%% Environnement Schunk simplifié pour l'affichage des réponses
\renewenvironment{Schunk}{%
  \setlength{\topsep}{0pt}
  \colorlet{shadecolor}{codebg}
  \begin{snugshade*}}%
  {\end{snugshade*}}

\input{solutions-informatique}
\input{solutions-bases}
\input{solutions-implementation}
\input{solutions-donnees}
\input{solutions-application}
\input{solutions-internes}

\endgroup

%%% Local Variables:
%%% mode: latex
%%% TeX-engine: xetex
%%% TeX-master: "programmer-avec-r"
%%% coding: utf-8
%%% End:


\bibliographystyle{francais}
\bibliography{stat,informatique}

\cleardoublepage
\printindex

\cleardoublepage
\cleartoverso

\pagestyle{empty}
\pagestyle{empty}
\renewcommand{\ttdefault}{hlst}

\bandeverso
\begin{textblock*}{71mm}(145mm, -40mm)
  \large\ttfamily\raggedright
  \textblockcolor{}
  ISBN \\ \ISBN
\end{textblock*}

\end{document}

%%% Local Variables: 
%%% mode: latex
%%% TeX-master: t
%%% End: 
